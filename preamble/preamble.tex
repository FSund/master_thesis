% ----  biblatex ---- %
\usepackage[english]{babel}
\usepackage{csquotes} % required by {babel} (or biblatex?)
\usepackage[backend=biber, sorting=nyt, style=numeric,%
    firstinits=true % render all first and middle names as initials
]{biblatex} % Note that "sorting=none" is NOT the same as leaving the field blank, "sorting=none" means "sorting=citeorder".
% About DATES:
% The date fields date, origdate, eventdate, and urldate require a date specification in yyyy-mm-dd format. Date ranges are given as yyyy-mm-dd/yyyy-mm-dd. Partial dates are valid provided that date components are omitted at the end only. You may specify an open ended date range by giving the range separator and omitting the end date (e. g., yyyy/).
\bibliography{bibliography/JabRef_database,bibliography/Master}
% Example from here: http://codydunne.blogspot.no/2012/01/suppressing-bibtex-fields-for-specific.html
\AtEveryBibitem{% Clean up the bibtex rather than editing it
    \clearlist{address}
%  \clearfield{date}
%     \ifentrytype{www}{}{ % url is alias for ``online''
%         \clearfield{url}
%     }
    \ifentrytype{online}{}{
        \clearfield{url}
    }
    \clearfield{doi}
    \clearfield{eprint}
    \clearfield{isbn}
    \clearfield{issn}
    \clearlist{location}
    \clearfield{month}
    \clearfield{series}
    
%     \ifentrytype{book}{}{% Remove publisher and editor except for books
%         \clearlist{publisher}
%         \clearname{editor}
%     }
}


% ---- Code syntax highlighting ---- %
% \usepackage[scaled=0.8]{beramono}  % Better monospaced font, nice ~ and ^
\usepackage[scale=0.8]{droidmono}

% Order: float - (fix \listoflistings) - hyperref - minted
\usepackage{float}

% Fix listoflistings (need to do this before loading hyperref)
% Fix \listoflistings
\usepackage{etoolbox}
% \newfloat{listing}{h}{lol} % Not necessary
\makeatletter
\patchcmd{\@chapter}%
 {\addtocontents{lof}}%
 {\addtocontents{lol}{\protect\addvspace{10pt}}%  % <-- "lol" is the extension minted use
  \addtocontents{lof}}%
\makeatother

% FIX LIST OF LISTINGS: 
% http://tex.stackexchange.com/a/58498/31078
% http://tex.stackexchange.com/questions/14856/change-spacing-between-elements-in-newfloat-listof/14867#14867
% http://tex.stackexchange.com/questions/139533/using-listing-with-minted-gets-wrong-listoflistings


% Load hyperref after fixing listoflistings
\usepackage{hyperref}
\input{preamble/hyperref.tex}

% Finally load minted
\usepackage[chapter]{minted} % Option [chapter] has to do with \listoflistings numbering. Alternative is ``section''
% Need to load the float package before hyperref, so hyperref detects and patches it, so when we load minted later it uses the patched \newfloat (minted loads float). 
% More info here: http://tex.stackexchange.com/questions/19371/minteds-listing-environment-with-hyperref-and-caption-package-together

\usemintedstyle{colorful}
\definecolor{codebg}{rgb}{0.95,0.95,0.95}
\newminted[cppcode]{mdcpp}{ % can use \newminted{cpp} to replace default ``cpp'', gives the same result
    mathescape,
%     frame=lines,
%     framesep=2mm,
    bgcolor = codebg,
    fontsize = \small,
%     fontfamily = courier
} 
% usage: \begin{cppcode}
% use \begin{cppcode*}{<extra options>} if you want to add extra options on the fly

% More options:
% \renewcommand\listingscaption{Program code}
% \renewcommand\listoflistingscaption{List of program codes}

% FIX LIST OF LISTINGS: 
% http://tex.stackexchange.com/a/58498/31078
% http://tex.stackexchange.com/questions/14856/change-spacing-between-elements-in-newfloat-listof/14867#14867
% http://tex.stackexchange.com/questions/139533/using-listing-with-minted-gets-wrong-listoflistingscaption



% ----  Draft stuff (remove before final version) ---- %
\usepackage[colorinlistoftodos,color=black!20]{todonotes}
\usepackage{xcolor}
\newcommand{\orangebox}[1]{
    \fcolorbox{black}{orange}{
        \begin{minipage}{0.96\textwidth}
            #1
        \end{minipage}
    }
}
\usepackage{soulutf8}       % To highlight stuff (when using utf8), using \hl{}. Also has \st{} for strikethrough. Doesn't work that well with equations...
\usepackage{lipsum}

% ---- Images ---- %
\usepackage{graphicx}
\usepackage{svg}            % To include .svg vector graphics directly using \includesvg (will automatically compile/convert the images to .pdf+.pdf_tex using Inkscape). 
                            % NEEDS ``pdflatex --shell-escape'' !!!
                            
\setsvg{%                   % conversion options for svg package
    inkscape = inkscape -z -D,%
    pretex = \footnotesize%
}

% The subfigure and subfig packages are deprecated and shouldn't be used any more: 
% http://tex.stackexchange.com/questions/144782/subfigure-and-subfig-packages-deprecated
% the svg-package originally uses subfig, but replacing subfig with subcaption seems to work
\usepackage{subcaption}     % \begin{subfigure}, \subcaption{}, and \subcaptionbox{}

% ---- Fix unicode stuff ---- %
\DeclareUnicodeCharacter{2212}{-}   % Unicode 'MINUS SIGN' (U+2212) that matplotlib uses for minus on axis tick labes

% ---- Formatting ---- %
% skip line instead of indent on new paragraph
% \setlength{\parskip}{11pt}
% \setlength{\parindent}{0mm}
\newlength{\oldparindent}
\setlength{\oldparindent}{\parindent} % doesn't work?
\usepackage[parfill]{parskip} % The option parfill is a useful addition: it avoids that a paragraph ends almost flush right. So, paragraphs could easier be distinguished.
