% ---- Figures ---- %
% (so we can just use \setlength later)
\newlength{\myfigwidth}
\newlength{\mycaptionwidth}

% \let\oldincludesvg\includesvg
% \renewcommand{\includesvg}[1]{\small{\includesvg{#1}}}
% \renewcommand{\includesvg}[1]{\begin{small}\includesvg{#1}\end{small}}


% ---- Custom commands and symbols ---- %

% ----- Vectors ---- %
% \newcommand{\bvec}[1]{\mathbf{#1}}
\newcommand{\oldvec}{\vec}
\newcommand{\bvec}[1]{\boldsymbol{#1}}      % Using amsmath's boldsymbol
\renewcommand{\vec}{\bvec}
% \newcommand{\bvec}[1]{\bm{#1}}
\newcommand{\rvec}{\vec{r}}
\newcommand{\vvec}{\vec{v}}
\newcommand{\avec}{\vec{a}}
\newcommand{\Fvec}{\vec{F}}
\newcommand{\rvecij}{\rvec_{ij}}

% ---- Math commands and symbols ---- %
% \newcommand\diff{\mathop{}\!\mathrm{d}}   % Already defined as \dif by commath! But maybe this way is better? Commath uses ``\DeclareMathOperator{\dif}{d \!}''
\DeclareMathOperator{\diff}{d \!}           % See top comment on this reply: http://tex.stackexchange.com/a/95681/31078
\newcommand{\drvec}{\dif \rvec}
% \DeclareMathOperator{\nablaop}{\nabla}
\renewcommand\div{\mathop{}\!\vec\nabla}
% \newcommand\Ham{\mathop{}\!\mathrm{\mathcal{H}}}
\DeclareMathOperator{\Ham}{\mathcal{H}}
\DeclareMathOperator{\Lag}{\mathcal{L}}
% \newcommand\bigint{\mathop{\mathlarger{\int}}}
\DeclareMathOperator{\bigint}{\mathlarger{\int}}
% \newcommand\deltaop{\mathop{\!\mathrm{\delta}}}
% \DeclareMathOperator{\deltaop}{\updelta \!}
\DeclareMathOperator{\deltaop}{\updelta}
% \newcommand\biggerint{\mathop{\mathlarger{\mathlarger{\int}}}}
\DeclareMathOperator{\biggerint}{\mathlarger{\mathlarger{\int}}}

% ---- Text ---- %
\newcommand{\Ang}{\AA ngstr\"om}
\newcommand{\Schr}{Schr\"odinger}
% \newcommand{\cpp}{\Verb!C++!} % Doesn't work in captions... Can be fixed with ``\protect'' before the verb, and ``\SaveVerb'' before list of figures, or the cprotect package. More info here: http://tex.stackexchange.com/questions/8810/how-to-include-verbatim-in-a-figure-caption
\newcommand{\cpp}{\texttt{C++}}

% ---- Todo colors ---- %
\newcommand{\todoa}[1]{\todo[inline,color=red!100]{#1}}
\newcommand{\todob}[1]{\todo[inline,color=red!60]{#1}}
\newcommand{\todoc}[1]{\todo[inline,color=red!30]{#1}}
\newcommand{\todod}[1]{\todo[inline,color=red!10]{#1}}
% \newcommand{\todod}[1]{\todo[inline,color="rgb:-green!40!yellow,3;green!40!yellow,2;red,1"]{#1}}
% rgb:-green!40!yellow,3;green!40!yellow,2;red,1

\newcommand{\todoao}[1]{\todo[color=red!100]{#1}}
\newcommand{\todobo}[1]{\todo[color=red!60]{#1}}
\newcommand{\todoco}[1]{\todo[color=red!30]{#1}}
\newcommand{\tododo}[1]{\todo[color=red!10]{#1}}
