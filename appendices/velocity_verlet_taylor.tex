\section{Deriving the Verlet algorithm using Taylor expansions\label{appendix:verlet_taylor}}
To derive the algorithm we first let
\begin{align*}
    \dod{\rvec(t)}{t} &= \vvec(t),
\end{align*}
and
\begin{align*}
    \dod{\vvec(t)}{t} &= \avec(t) = \frac{\Fvec(t)}{m}.
\end{align*}

We then do a Taylor expansion of $\rvec(t \pm \Delta t)$ around time $t$
\begin{align}
    \rvec(t + \Delta t) &= \rvec(t) + \vvec(t)\Delta t + \avec(t)\frac{\Delta t^2}{2} + \dod[3]{\rvec(0)}{t}\frac{\Delta t^3}{6} + \mathcal{O}(\Delta t^4), \label{eq:verlet_plus}\\
    \rvec(t - \Delta t) &= \rvec(t) - \vvec(t)\Delta t + \avec(t)\frac{\Delta t^2}{2} - \dod[3]{\rvec(0)}{t}\frac{\Delta t^3}{6} + \mathcal{O}(\Delta t^4).\label{eq:verlet_minus}
\end{align}
By summing these two equations we get
\begin{align*}
    \rvec(t + \Delta t) + \rvec(t - \Delta t) = 2\rvec(t) + \avec(t)\Delta t^2 + \mathcal{O}(\Delta t^4),
\end{align*}
which by rearranging can be written as
\begin{align*}
    \rvec(t + \Delta t) \approx 2\rvec(t) - \rvec(t - \Delta t) + \avec(t)\Delta t^2.
\end{align*}
This is the equation used to update the positions in the regular Verlet algorithm. We see that the estimate of the new position contains an truncation error for one timestep $\Delta t$ of the order $\mathcal{O}(\Delta t^4)$.

The Verlet algorithm does not use the velocity to compute the new position, but we can find an estimate of the velocity by taking the difference between \cref{eq:verlet_plus,eq:verlet_minus}
\begin{align*}
    \rvec(t + \Delta t) - \rvec(t - \Delta t) = 2\vvec(t)\Delta t + \mathcal{O}(\Delta t^3),
\end{align*}
which by rearranging can be written as
\begin{align*}
    \vvec(t) = \frac{\rvec(t + \Delta t) - \rvec(t - \Delta t)}{2\Delta t} + \mathcal{O}(\Delta t^2).
\end{align*}
We see that this estimate of the velocity has a truncation error of the order $\mathcal{O}(\Delta t^2)$, compared to the error in the position $\mathcal{O}(\Delta t^4)$.
% \todo{Something about lower precision}

\subsection{Velocity Verlet}
A modification of the Verlet algorithm usually called the velocity Verlet algorithm % \cite{swope1982computer} %\todo{something about why velocity Verlet is good} %
can be derived in a similar way. We have the same Taylor expansion of $\rvec(t+\Delta t)$ around $t$ as before
\begin{align}
    \rvec(t + \Delta t) &= \rvec(t) + \vvec(t)\Delta t + \avec(t)\frac{\Delta t^2}{2} + \mathcal{O}(\Delta t^3), \label{eq:position_taylor}
\end{align}
and now we also expand $\vvec(t + \Delta t)$ around $t$
\begin{align}
    \vvec(t + \Delta t) 
%     &= \vvec(t) + \dod{\vvec(t)}{t}\Delta t + \dod[2]{\vvec(t)}{t}\frac{\Delta t^2}{t} + \mathcal{O}(\Delta t^3) \\
    &= \vvec(t) + \avec(t)\Delta t + \dod[2]{\vvec(t)}{t}\frac{\Delta t^2}{2} + \mathcal{O}(\Delta t^3). \label{eq:velocity_taylor}
\end{align}
We now need an expression for $\od[2]{\vvec(t)}{t}$, which can be found by a Taylor expansion of $\od{\vvec(t+\Delta t)}{t}$
\begin{align*}
    \dod{\vvec(t+\Delta t)}{t} = \dod{\vvec(t)}{t} + \dod[2]{\vvec(t)}{t}\Delta t + \mathcal{O}(\Delta t^2),
\end{align*}
which by rearranging and multiplying with $\frac{\Delta t}{2}$ gives
\begin{align*}
    \dod[2]{\vvec}{t}\frac{\Delta t^2}{2} 
    &= \left( \dod{\vvec(t+\Delta t)}{t} - \dod{\vvec(t)}{t}\right)\frac{\Delta t}{2} + \mathcal{O}(\Delta t^3) \\
    &= \big[\avec(t + \Delta t) - \avec(t)\big] \frac{\Delta t}{2} + \mathcal{O}(\Delta t^3).
\end{align*}
Inserting this into \cref{eq:velocity_taylor} we get
\begin{align}
    \vvec(t + \Delta t) 
    &= \vvec(t) + \avec(t)\Delta t + \big[\avec(t + \Delta t) - \avec(t)\big] \frac{\Delta t}{2} + \mathcal{O}(\Delta t^3) \nonumber\\
    &= \vvec(t) + \big[\avec(t) + \avec(t + \Delta t)\big] \frac{\Delta t}{2} + \mathcal{O}(\Delta t^3). \label{eq:verlet_velocity_taylor_insertion}
\end{align}
So the total velocity Verlet algorithm with truncation of the higher-order terms is
\begin{align}
    \rvec(t + \Delta t) &= \rvec(t) + \vvec(t)\Delta t + \avec(t)\frac{\Delta t^2}{2} %\label{eq:velocity_verlet_position}
    \\
    \vvec(t + \Delta t) &= \vvec(t) + \big[\avec(t) + \avec(t + \Delta t)\big] \frac{\Delta t}{2}, %\label{eq:velocity_verlet_velocity}
\end{align}
with the truncation error for one timestep $\Delta t$ being of the order $\mathcal{O}(\Delta t^3)$ for both the position and the velocity. 
