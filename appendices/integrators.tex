\chapter{Verlet integrators\label{appendix:verlet_integrators}}

\fcolorbox{black}{orange}{
\begin{minipage}{\textwidth}
{\bf TODO:}
\begin{itemize}
    \item Truncation error Verlet/velociy Verlet
    \item Numerical stability?
    \item Memory?
    \item Self starting, symplectic, reversible
\end{itemize}
\end{minipage}
}
\todo{why do we use velocity Verlet}

The Verlet algorithm\cite{verlet1967computer} is a simple method for \hl{numerically?} integrating second order differential equations of the form 
\begin{align*}
    \dod[2]{\rvec(t)}{t} = \Fvec\big[\rvec(t), t\big] = \Fvec(t).
\end{align*}
The algorithm has several \hl{(equivalent?)} forms, and the form \hl{used/developed?} by Verlet is
\begin{align*}
    \rvec(t + \Delta t) \approx 2\rvec(t) - \rvec(t - \Delta t) + \avec(t)\Delta t^2,
\end{align*}
where $\Delta t$ is the timestep\todo{define timestep}, and $\avec(t)$ is the velocity at time $t$. An equivalent formulation, usually called the velocity Verlet algorithm, has the form
\begin{align*}
    \rvec(t + \Delta t) &= \rvec(t) + \vvec(t)\Delta t + \frac{\Fvec(t)}{2m}\Delta t^2 \\
    \vvec(t + \Delta t) &= \vvec(t) + \frac{\Fvec(t + \Delta t) + \Fvec(t)}{2m}\Delta t.
\end{align*}
The velocity Verlet algorithm is the most used form \hl{why?}, and \hl{something about errors, references to below}.

This algorithm has a \hl{glocal/accumulated} error of $\mathcal{O}(\Delta t^2)$\todo{either \cite{thijssen1999computational} sec. 8.4.1-8.4.3 or \cite{frenkel2001understanding} sec. 4.3.3}.

% \section{Verlet integrators}
% \begin{align*}
%     \rvec(t) = r_n \\
%     \vvec(t) = v_n \\
%     \bvec a(t) = a_n.
% \end{align*}
% \begin{align*}
%     \vvec_{n+1/2} = \vvec_n + \frac{1}{2}\bvec a_n \delta t \\
%     \rvec_{n+1} = \rvec_n + \vvec_{n+\frac{1}{2}}\delta t
% \end{align*}

% \begin{align*}
%     \vvec(t + \Delta t/2) &= \vvec(t) + \frac{1}{2}\bvec a(t) \Delta t \\
%     \rvec(t + \Delta t)   &= \rvec(t) + \vvec(t + \Delta t/2)\Delta t \\
%     \bvec a(t + \Delta t)   &= \frac{1}{m}\Fvec(\rvec(t + \Delta t)) \\
%     \vvec(t + \Delta t)   &= \vvec(t + \Delta t) + \frac{1}{2}\bvec a(t + \Delta t) \Delta t
% \end{align*}

We will now first derive the regular Verlet and the velocity Verlet algorithms using Taylor expansions, and then using the Liouville formulation of classical mechanics. \todo{why use two methods? what do we learn from them?} \todo{mention what errors we find?}

\section{Deriving the Verlet algorithm using Taylor expansions\label{appendix:verlet_taylor}}

To derive the algorithm we first let
\begin{align*}
    \dod{\rvec(t)}{t} &= \vvec(t),
\end{align*}
and
\begin{align*}
    \dod{\vvec(t)}{t} &= \avec(t) = \frac{\Fvec(t)}{m}.
\end{align*}

We then do a Taylor expansion of $\rvec(t \pm \Delta t)$ around time $t$
\begin{align}
    \rvec(t + \Delta t) &= \rvec(t) + \vvec(t)\Delta t + \avec(t)\frac{\Delta t^2}{2} + \dod[3]{\rvec(0)}{t}\frac{\Delta t^3}{6} + \mathcal{O}(\Delta t^4), \label{eq:verlet_plus}\\
    \rvec(t - \Delta t) &= \rvec(t) - \vvec(t)\Delta t + \avec(t)\frac{\Delta t^2}{2} - \dod[3]{\rvec(0)}{t}\frac{\Delta t^3}{6} + \mathcal{O}(\Delta t^4).\label{eq:verlet_minus}
\end{align}
By summing these two equations we get
\begin{align*}
    \rvec(t + \Delta t) + \rvec(t - \Delta t) = 2\rvec(t) + \avec(t)\Delta t^2 + \mathcal{O}(\Delta t^4),
\end{align*}
which by rearranging can be written as
\begin{align*}
    \rvec(t + \Delta t) \approx 2\rvec(t) - \rvec(t - \Delta t) + \avec(t)\Delta t^2.
\end{align*}
This is the equation used to update the positions in the regular Verlet algorithm. We see that the estimate of the new position contains an truncation error for one timestep $\Delta t$ of the order $\mathcal{O}(\Delta t^4)$.

The Verlet algorithm does not use the velocity to compute the new position, but we can find an estimate of the velocity by taking the difference between \cref{eq:verlet_plus,eq:verlet_minus}
\begin{align*}
    \rvec(t + \Delta t) - \rvec(t - \Delta t) = 2\vvec(t)\Delta t + \mathcal{O}(\Delta t^3),
\end{align*}
which by rearranging can be written as
\begin{align*}
    \vvec(t) = \frac{\rvec(t + \Delta t) - \rvec(t - \Delta t)}{2\Delta t} + \mathcal{O}(\Delta t^2).
\end{align*}
We see that this estimate of the velocity has a truncation error of the order $\mathcal{O}(\Delta t^2)$, compared to the error in the position $\mathcal{O}(\Delta t^4)$.
\todo{Something about lower precision}

\subsection{Velocity Verlet}
A modification of the Verlet algorithm usually called the velocity Verlet algorithm\cite{swope1982computer} \todo{something about why velocity Verlet is good} can be derived in a similar way. We have the same Taylor expansion of $\rvec(t+\Delta t)$ around $t$ as before
\begin{align}
    \rvec(t + \Delta t) &= \rvec(t) + \vvec(t)\Delta t + \avec(t)\frac{\Delta t^2}{2} + \mathcal{O}(\Delta t^3), \label{eq:position_taylor}
\end{align}
and now we also expand $\vvec(t + \Delta t)$ around $t$
\begin{align}
    \vvec(t + \Delta t) 
%     &= \vvec(t) + \dod{\vvec(t)}{t}\Delta t + \dod[2]{\vvec(t)}{t}\frac{\Delta t^2}{t} + \mathcal{O}(\Delta t^3) \\
    &= \vvec(t) + \avec(t)\Delta t + \dod[2]{\vvec(t)}{t}\frac{\Delta t^2}{2} + \mathcal{O}(\Delta t^3). \label{eq:velocity_taylor}
\end{align}
We now need an expression for $\od[2]{\vvec(t)}{t}$, which can be found by a Taylor expansion of $\od{\vvec(t+\Delta t)}{t}$
\begin{align*}
    \dod{\vvec(t+\Delta t)}{t} = \dod{\vvec(t)}{t} + \dod[2]{\vvec(t)}{t}\Delta t + \mathcal{O}(\Delta t^2),
\end{align*}
which by rearranging and multiplying with $\frac{\Delta t}{2}$ gives
\begin{align*}
    \dod[2]{\vvec}{t}\frac{\Delta t^2}{2} 
    &= \left( \dod{\vvec(t+\Delta t)}{t} - \dod{\vvec(t)}{t}\right)\frac{\Delta t}{2} + \mathcal{O}(\Delta t^3) \\
    &= \big[\avec(t + \Delta t) - \avec(t)\big] \frac{\Delta t}{2} + \mathcal{O}(\Delta t^3).
\end{align*}
Inserting this into \cref{eq:velocity_taylor} we get
\begin{align}
    \vvec(t + \Delta t) 
    &= \vvec(t) + \avec(t)\Delta t + \big[\avec(t + \Delta t) - \avec(t)\big] \frac{\Delta t}{2} + \mathcal{O}(\Delta t^3) \nonumber\\
    &= \vvec(t) + \big[\avec(t) + \avec(t + \Delta t)\big] \frac{\Delta t}{2} + \mathcal{O}(\Delta t^3). \label{eq:verlet_velocity_taylor_insertion}
\end{align}
So the total velocity Verlet algorithm with truncation of the higher-order terms is
\begin{align}
    \rvec(t + \Delta t) &= \rvec(t) + \vvec(t)\Delta t + \avec(t)\frac{\Delta t^2}{2} %\label{eq:velocity_verlet_position}
    \\
    \vvec(t + \Delta t) &= \vvec(t) + \big[\avec(t) + \avec(t + \Delta t)\big] \frac{\Delta t}{2}, %\label{eq:velocity_verlet_velocity}
\end{align}
with the truncation error for one timestep $\Delta t$ being of the order $\mathcal{O}(\Delta t^3)$ for both the position and the velocity. 

% \subsection{Implementation}
% \todo{Remove this subsec? Included in MD part}
% The algorithm is usually rewritten in the following way, to optimize the implementation on a computer. We see that the new velocities can be written as
% \begin{align}
%     \vvec(t+\Delta t) = \tilde\vvec(t + \tfrac{1}{2}\Delta t) + \avec(t+\Delta t)\frac{\Delta t}{2}, %\label{eq:verlet_velocity_with_halfstep}
% \end{align}
% where
% \begin{align}
%     \tilde\vvec(t + \tfrac{1}{2}\Delta t) = \vvec(t) + \avec(t)\frac{\Delta t}{2}. \label{eq:verlet_halfstep}
% \end{align}
% We see that \cref{eq:verlet_halfstep} can be used in updating the positions, so we rewrite \cref{eq:velocity_verlet_position} to
% \begin{align}
%     \rvec(t + \Delta t) &= \rvec(t) + \tilde\vvec(t+\tfrac{1}{2}\Delta t)\Delta t. %\label{eq:velocity_verlet_positions_halfstep}
% \end{align}
% Which leads us to the usual way of implementing the algorithm\cite{allen1989computer}:
% \begin{itemize}
%     \item Calculate the velocities at $t+\tfrac{1}{2}\Delta t$ using \cref{eq:verlet_halfstep} \hl{(repeated here)}
%     \begin{align*}
%         \tilde\vvec(t + \tfrac{1}{2}\Delta t) = \vvec(t) + \frac{\Fvec(t)}{m}\frac{\Delta t}{2}.
%     \end{align*}
%     \item Calculate the new positions at $t + \Delta t$ using \cref{eq:velocity_verlet_positions_halfstep} \hl{(repeated here)}
%     \begin{align*}
%         \rvec(t + \Delta t) &= \rvec(t) + \tilde\vvec(t+\tfrac{1}{2}\Delta t)\Delta t.
%     \end{align*}
%     \item Calculate the new forces $\Fvec(t+\Delta t)$.
%     \item Calculate the new velocities at $t+\Delta t$ using \cref{eq:verlet_velocity_with_halfstep} \hl{(repeated here)}
%     \begin{align*}
%         \vvec(t+\Delta t) = \vvec(t + \tfrac{1}{2}\Delta t) + \frac{\Fvec(t + \Delta t)}{m}\frac{\Delta t}{2}.
%     \end{align*}
% \end{itemize}
% This implementation minimizes the memory needs, as we only need to store one copy of $\rvec$, $\vvec$ and $\Fvec$ at all times, compared to implementing \cref{eq:velocity_verlet_position,eq:velocity_verlet_velocity} which needs to store the values of both $\Fvec(t)$ and $\Fvec(t+\Delta)$ to calculate the new velocities. 
% 
% Pseudocode: \todo{do we want this?}
% \begin{verbatim}
%     v += F*dt/(2*m);
%     r += v*dt;
%     F = calculate_forces(r, v);
%     v += F*dt/(2*m);
% \end{verbatim}



% The velocity Verlet algorithm can be implemented as follows\cite{allen1989computer}
% \begin{samepage}
% \begin{itemize}
%     \item Calculate the new positions at $t + \Delta t$ using \cref{eq:velocity_verlet_position}.
%     \item Calculate the velocities at $t + \frac{1}{2}\Delta t$ using
%     \begin{align*}
%         \vvec(t + \tfrac{1}{2}\Delta t) = \vvec(t) + \frac{\Fvec(t)}{m}\frac{\Delta t}{2}.
%     \end{align*}
%     \item Calculate the new forces $\Fvec(t + \Delta t)$.
%     \item Finally, calculate the new velocities at $t + \Delta t$ using
%     \begin{align*}
%         \vvec(t + \Delta t) = \vvec(t + \tfrac{1}{2}\Delta t) + \frac{\Fvec(t + \Delta t)}{m}\frac{\Delta t}{2}
%     \end{align*}
% \end{itemize}
% \end{samepage}
% But 
% This implementation uses $9N$ units of memory, 

\section{Global error in the Verlet algorithm - BAD}
Sources:
\begin{itemize}
    \item \url{http://math.stackexchange.com/questions/668707/verlet-method-global-error}
    \item \url{http://www.saylor.org/site/wp-content/uploads/2011/06/MA221-6.1.pdf} (same as Wikipedia)
\end{itemize}

\todo{I don't know if the calculations below are correct} 
The global error in position can be derived from the local error for one timestep $\Delta t$. We see from \cref{eq:position_taylor} that
\begin{align*}
    \text{error}\big(\rvec(t_0+\Delta t)\big) = \mathcal{O}(\Delta t^3),
\end{align*}
and from \cref{eq:verlet_velocity_taylor_insertion}
\begin{align*}
    \text{error}\big(\vvec(t_0+\Delta t)\big) = \mathcal{O}(\Delta t^3).
\end{align*}
The error for two timesteps is
\begin{align*}
    \rvec(t_0+2\Delta t) 
    &= \rvec(t_0+\Delta t) + \vvec(t_0+\Delta t)\Delta t + \avec(t_0+\Delta t)\frac{\Delta t^2}{2} + \mathcal{O}(\Delta t^3), \\
\end{align*}
which gives \todo{Error in $\avec(t_0 + \Delta t) = \mathcal{O}(\Delta t^3)$ ??}
\begin{align*}
    &\text{error}\big(\rvec(t_0+2\Delta t)\big) \\
    &~= \text{error}\big(\rvec(t_0+\Delta t)\big) + \text{error}\big(\vvec(t_0+\Delta t)\big)\Delta t + \text{error}\big(\avec(t_0+\Delta t)\big)\frac{\Delta t^2}{2} + \mathcal{O}(\Delta t^3)\\
    &~= \mathcal{O}(\Delta t^3) + \mathcal{O}(\Delta t^3)\Delta t + \mathcal{O}(\Delta t^3)\frac{\Delta t^2}{2} + \mathcal{O}(\Delta t^3)\\
    &~= 2\mathcal{O}(\Delta t^3).
\end{align*}
Similarly we find
\begin{align*}
    &\text{error}\big(\rvec(t_0+3\Delta t)\big) = 3\mathcal{O}(\Delta t^3) \\
    &\text{error}\big(\rvec(t_0+4\Delta t)\big) = 4\mathcal{O}(\Delta t^3) \\
    &\text{error}\big(\rvec(t_0+5\Delta t)\big) = 5\mathcal{O}(\Delta t^3),
\end{align*}
\begin{align*}
    \text{error}\big(\rvec(t_0 + n\Delta t)\big) = n\mathcal{O}(\Delta t^3) = \mathcal{O}(\Delta t^2)
\end{align*}

\section{Deriving velocity Verlet using Liouville operator\label{sec:liou}\label{appendix:liouville_verlet}}
\newcommand{\Liou}{i\hat{\vec L}}
\newcommand{\Lop}{\hat{\mathcal{U}}}
% \todo{Cite Tuckerman et al. ([71] in Frenkel) ?}
%
We will now derive the velocity Verlet algorithm in a more rigorous way, using the Liouville formulation of classical mechanics. This approach will give us better insight into why the algorithm is so powerful, and a good estimate of the global (or accumulated) error of this algorithm. The derivation closely follows section 4.3.3 in \cite{frenkel2001understanding}.

\subsection{Liouville operator}
We have a system consisting of $N$ particles, with positions $\rvec$ and momenta $\vec p$. We define a function of these variables $f(\rvec(t), \vec p(t)) = f(t)$, that has the time derivative %
%(the derivative is denoted by a dot)
(denoted by a dot)
\begin{align}
    \dot f(t) = \dot \rvec \dpd{f(t)}{\rvec} + \dot {\vec p} \dpd{f(t)}{\vec p} = \Liou f(t).
    \label{eq:liou_f}
\end{align}
where we have defined the \emph{Liouville operator}, $\Liou$, as
\begin{align*}
    \Liou = \dot \rvec \dpd{}{\rvec} + \dot {\vec p} \dpd{}{\vec p} = \Liou_r + \Liou_p
\end{align*}
where $\Liou_r$ and $\Liou_p$ denotes the left and right part if this operator, respectively. %
We can formally integrate \cref{eq:liou_f} to obtain
\begin{align}
    f(t) = e^{\Liou t} f(0),
    \label{eq:liou_integrate}
\end{align}
which allows us to define the time evolution operator $\Lop = \exp(\Liou t)$. We see that this integration doesn't get us any closer to finding $f(t)$%\todo{why do we want to find $f(t)$?}%
, since evaluating the right-hand side is equivalent to the exact integration of the classical equations of motion. To get around this we define the time evolution operator for positions $\Lop_r(t) = \exp(\Liou_r t)$, and try applying just this operator. If we do a Taylor expansion of the exponential we get
\begin{align*}
    \Lop_r(t)f(0)
    &= f(0) + \Liou_r t f(0) + \frac{(\Liou_r t)^2}{2!} f(0) + \dots \\
    &= \exp\left( \dot\rvec (0) t \dpd{}{\rvec} \right) f(0) \\
    &= \sum_{n=0}^\infty \frac{ \big(\dot\rvec(0) t\big)^n}{n!} \frac{\partial^n}{\partial \rvec^n} f(0) \\
    &= f\left\{ \left[ \rvec(0) + \dot\rvec(0)t \right], \vec p(0) \right\},
\end{align*}
where $\rvec(0)$ and $\vec p(0)$ are the positions and momenta at $t = 0$. We see that this has the effect of moving the positions $\rvec$ a step $t$ forward in time according to their derivative. It's easy to see that the equivalent momentum time evolution operator $\Lop_p(t) = \exp(\Liou_p t)$ has a similar effect on the momenta. 

\subsection{Velocity Verlet}
In a molecular dynamics simulation we would like to be able to apply these operators independently, since
\begin{align*}
    \Lop = e^{\Liou} = e^{\Liou_r + \Liou_p},
\end{align*}
but unfortunately, for two noncommuting operators $\hat {\vec A}$ and $\hat {\vec B}$ we have
\begin{align*}
    e^{\hat {\vec A} + \hat {\vec B}} \neq e^{\hat {\vec A}} e^{\hat {\vec B}}.
\end{align*}
To solve this we can use the following \emph{Trotter identity}
\begin{align*}
    e^{\hat {\vec A} + \hat {\vec B}} = \lim_{P\rightarrow \infty} \left( e^{\hat {\vec A}/2P} e^{\hat {\vec B}/P} e^{\hat {\vec A}/2P} \right)^P.
\end{align*}
Applying the operators an infinite number of times ($P\rightarrow \infty$) is unpractical, but fortunately the expression can be truncated for large but finite $P$ as
\begin{align}
    e^{\hat {\vec A} + \hat {\vec B}} = \left( e^{\hat {\vec A}/2P} e^{\hat {\vec B}/P} e^{\hat {\vec A}/2P} \right)^P e^{\mathcal{O}(1/P^2)}.
    \label{eq:liou_time_exact}
\end{align}

To derive the velocity Verlet scheme using this truncation we first identify the operators $\hat {\vec A}$ and $\hat {\vec B}$ as
\begin{align*}
    \frac{\hat {\vec A}}{P} &\equiv \frac{\Liou_p t}{P} \equiv \Delta t \dot {\vec p}(0) \dpd{}{\vec p} \\
    \frac{\hat {\vec B}}{P} &\equiv \frac{\Liou_r t}{P} \equiv \Delta t \dot \rvec (0) \dpd{}{\rvec}
\end{align*}
where $\Delta t = t/P$. We can now identify the \emph{truncated} time evolution operator $\Lop^*(t)$ as follows
\begin{align}
    \Lop(t) 
    &= \left( e^{\Liou_p\Delta t/2} e^{\Liou_r \Delta t} e^{\Liou_p \Delta t/2} \right)^P e^{\mathcal{O}(1/P^2)} \nonumber\\
    &\approx \left( e^{\Liou_p\Delta t/2} e^{\Liou_r \Delta t} e^{\Liou_p \Delta t/2} \right) = \Lop^*(t),
    \label{eq:liou_time_trunc}
\end{align}
% \begin{align*}
%     \Lop^*(t) &= \left( e^{\Liou_p\Delta t/2} e^{\Liou_r \Delta t} e^{\Liou_p \Delta t/2} \right)^P,
%     \label{eq:liou_time_trunc}
% \end{align*}
and the the truncated operator for moving \emph{one} timestep forward as
\begin{align}
    \Lop^*(\Delta t) = e^{\Liou_p\Delta t/2} e^{\Liou_r \Delta t} e^{\Liou_p \Delta t/2}.
    \label{eq:liou_onetime_trunc}
\end{align}

To see the effect of the operator $\Lop^*(\Delta t)$ on the coordinates and momenta of the particles we first apply $\exp(\Liou_p\Delta t/2)$ to $f(0)$, and get
\begin{align*}
    e^{\Liou_p \Delta t/2} f(0) = f \left\{ \rvec(0), \left[ \vec p(0) + \frac{\Delta t}{2} \dot{\vec p}(0) \right] \right\}.
\end{align*}
We then apply $\exp(\Liou_r\Delta t)$ and get
\begin{align*}
     e^{\Liou_r \Delta t} f \left\{ \rvec(0), \left[ \vec p(0) + \frac{\Delta t}{2}\dot{\vec p}(0) \right] \right\}
     &= f \left\{ \left[ \rvec(0) + \Delta t \dot\rvec\left(\frac{\Delta t}{2}\right) \right], \left[ \vec p(0) + \frac{\Delta t}{2}\dot{\vec p}(0) \right] \right\},
\end{align*}
and finally we apply $\exp(\Liou_p\Delta t/2)$ once more, and get
\begin{align*}
    f \left\{ 
        \left[ \rvec(0) + \Delta t \dot\rvec\left(\frac{\Delta t}{2}\right) \right], 
        \left[ \vec p(0) + \frac{\Delta t}{2}\dot{\vec p}(0) + \frac{\Delta t}{2}\dot{\vec p}(\Delta t) \right] 
    \right\}.
\end{align*}

If we look at the total effect of applying the operator we see that
\begin{align}
    \rvec(0)
    &\rightarrow \rvec(0) + \Delta t \dot\rvec\left(\frac{\Delta t}{2}\right) \label{eq:liou_pos_effect} \\
    \vec p(0) 
    &\rightarrow \vec p(0) + \big[\dot{\vec p}(0) + \dot{\vec p}(\Delta t) \big] \frac{\Delta t}{2} \label{eq:liou_momentum_effect}.
\end{align}
Using the relations $\vec p = m\vvec$, $\dot{\vec p} = \vec F$, and, if we assume that the forces only depend on the positions, $\Fvec(\rvec(t)) = \Fvec(t)$, the relation 
\begin{align*}
    &\dot\rvec(\Delta t/2) = \rvec(0) + \frac{\Fvec(0)}{m} \frac{\Delta t}{2},
\end{align*}
we can rewrite \cref{eq:liou_momentum_effect,eq:liou_pos_effect} to
\begin{align*}
    \vvec(\Delta t) &= \vvec(0) + \left[\frac{\Fvec(0)}{m} + \frac{\Fvec(\Delta t)}{m}\right] \frac{\Delta t}{2} \\
    \rvec(\Delta t) &= \rvec(0) + \vvec(0)\Delta t + \frac{\Fvec(0)}{m}\frac{\Delta t^2}{2},
\end{align*}
which is exactly the velocity Verlet algorithm, as we saw in \cref{eq:velocity_verlet_position,eq:velocity_verlet_velocity}.

\subsection{Error in velocity Verlet\label{subsec:verlet_error_liouville}}
\newcommand{\errop}{\hat{\epsilon}}%
When we approximate the exact time evolution operator for one timestep $\Lop (\Delta t)$ with $\Lop^*(\Delta t)$, going from \cref{eq:liou_time_exact} to \cref{eq:liou_time_trunc,eq:liou_onetime_trunc}, we do a truncation 
\begin{align*}
    \Lop(\Delta t) = \Lop^*(\Delta t) e^{\mathcal{O}(1/P^2)} \approx \Lop^*(\Delta t).
\end{align*}
To investigate the error introduced by this truncation we express it as an \emph{error operator} $\errop$
\begin{align*}
    e^{\Liou_p\Delta t/2} e^{\Liou_r \Delta t} e^{\Liou_p \Delta t/2} e^{\mathcal{O}(1/P^2)}
    = e^{\Liou \Delta t + \errop},
\end{align*}
which, using Campbell-Baker-Hausdorff expansion, can be expressed in terms of the commutators of $\vec L_p$ and $\vec L_r$ as
\begin{align}
    \errop = \sum_{n=1}^\infty (\Delta t)^{2n+1} c_{2n+1},\label{eq:campbell_expansion}
\end{align}
where $c_m$ denotes a combination of mth-order commutators. %

From this it can be shown that, if the expansion in \cref{eq:campbell_expansion} converges, Verlet style integrators will rigorously conserve a \emph{pseudo}-Hamiltonian, and that the difference between this pseudo-Hamiltonian and the actual Hamiltonian is of the order $(\Delta t)^{2n}$, where $n$ depends on the order of the algorithm\cite[section 4.3.3]{frenkel2001understanding}.




