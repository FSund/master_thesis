\chapter{Integrators}
Liouville operator, velocity Verlet, global error


\section{Integration}
% \begin{align*}
%     \rvec(t) = r_n \\
%     \vvec(t) = v_n \\
%     \bvec a(t) = a_n.
% \end{align*}
% \begin{align*}
%     \vvec_{n+1/2} = \vvec_n + \frac{1}{2}\bvec a_n \delta t \\
%     \rvec_{n+1} = \rvec_n + \vvec_{n+\frac{1}{2}}\delta t
% \end{align*}

% \begin{align*}
%     \vvec(t + \Delta t/2) &= \vvec(t) + \frac{1}{2}\bvec a(t) \Delta t \\
%     \rvec(t + \Delta t)   &= \rvec(t) + \vvec(t + \Delta t/2)\Delta t \\
%     \bvec a(t + \Delta t)   &= \frac{1}{m}\Fvec(\rvec(t + \Delta t)) \\
%     \vvec(t + \Delta t)   &= \vvec(t + \Delta t) + \frac{1}{2}\bvec a(t + \Delta t) \Delta t
% \end{align*}

\fcolorbox{black}{orange}{
\begin{minipage}{\textwidth}
{\bf TODO:}
\begin{itemize}
    \item Truncation error Verlet/velociy Verlet
    \item Numerical stability?
    \item Memory?
    \item Self starting, symplectic, reversible
\end{itemize}
\end{minipage}
}

The equations of motion are integrated using the velocity Verlet algorithm:
\begin{align*}
    \rvec(t + \Delta t) &= \rvec(t) + \vvec(t)\Delta t + \frac{\Fvec(t)}{2m}\Delta t^2 \\
    \vvec(t + \Delta t) &= \vvec(t) + \frac{\Fvec(t + \Delta t) + \Fvec(t)}{2m}\Delta t.
\end{align*}
This algorithm has a \hl{glocal/accumulated} error of $\mathcal{O}(\Delta t^2)$\todo{either \cite{thijssen1999computational} sec. 8.4.1-8.4.3 or \cite{frenkel2001understanding} sec. 4.3.3}.

\subsection{Derivation of Verlet algorithm}
\todo{why do we use velocity Verlet}
The Verlet algorithm\cite{verlet1967computer} is a simple method for integrating second order differential equations of the form 
\begin{align*}
    \dod[2]{\rvec(t)}{t} = \Fvec\big[\rvec(t), t\big] = \Fvec(t).
\end{align*}
We first let
\begin{align*}
    \dod{\rvec(t)}{t} &= \vvec(t),
\end{align*}
and
\begin{align*}
    \dod{\vvec(t)}{t} &= \avec(t) = \frac{\Fvec(t)}{m}.
\end{align*}

We then do a Taylor expansion of $\rvec(t \pm \Delta t)$ around time $t$
\begin{align}
    \rvec(t + \Delta t) &= \rvec(t) + \vvec(t)\Delta t + \avec(t)\frac{\Delta t^2}{2} + \dod[3]{\rvec(0)}{t}\frac{\Delta t^3}{6} + \mathcal{O}(\Delta t^4), \label{eq:verlet_plus}\\
    \rvec(t - \Delta t) &= \rvec(t) - \vvec(t)\Delta t + \avec(t)\frac{\Delta t^2}{2} - \dod[3]{\rvec(0)}{t}\frac{\Delta t^3}{6} + \mathcal{O}(\Delta t^4).\label{eq:verlet_minus}
\end{align}
By summing these two equations we get
\begin{align*}
    \rvec(t + \Delta t) + \rvec(t - \Delta t) = 2\rvec(t) + \avec(t)\Delta t^2 + \mathcal{O}(\Delta t^4),
\end{align*}
which by rearranging can be written as
\begin{align*}
    \rvec(t + \Delta t) \approx 2\rvec(t) - \rvec(t - \Delta t) + \avec(t)\Delta t^2,
\end{align*}
Which is the equation used to update the positions in the regular Verlet algorithm. We see that the estimate of the new position contains an truncation error for one timestep $\Delta t$ of the order $\mathcal{O}(\Delta t^4)$.

The Verlet algorithm does not use the velocity to compute the new position, but we can find an estimate of the velocity by taking the difference between \cref{eq:verlet_plus,eq:verlet_minus}
\begin{align*}
    \rvec(t + \Delta t) - \rvec(t - \Delta t) = 2\vvec(t)\Delta t + \mathcal{O}(\Delta t^3),
\end{align*}
which by rearranging can be written as
\begin{align*}
    \vvec(t) = \frac{\rvec(t + \Delta t) - \rvec(t - \Delta t)}{2\Delta t} + \mathcal{O}(\Delta t^2).
\end{align*}
We see that this estimate of the velocity has a truncation error of the order $\mathcal{O}(\Delta t^2)$, compared to the error in the position $\mathcal{O}(\Delta t^4)$.
\todo{Something about lower precision}

A modification of the Verlet algorithm usually called the velocity Verlet algorithm\cite{swope1982computer} \todo{something about why velocity Verlet is good} can be derived in a similar way. We have the same Taylor expansion of $\rvec(t+\Delta t)$ around $t$ as before
\begin{align}
    \rvec(t + \Delta t) &= \rvec(t) + \vvec(t)\Delta t + \avec(t)\frac{\Delta t^2}{2} + \mathcal{O}(\Delta t^3), %\label{eq:position_taylor}
\end{align}
and now we also expand $\vvec(t + \Delta t)$ around $t$
\begin{align}
    \vvec(t + \Delta t) 
%     &= \vvec(t) + \dod{\vvec(t)}{t}\Delta t + \dod[2]{\vvec(t)}{t}\frac{\Delta t^2}{t} + \mathcal{O}(\Delta t^3) \\
    &= \vvec(t) + \avec(t)\Delta t + \dod[2]{\vvec(t)}{t}\frac{\Delta t^2}{2} + \mathcal{O}(\Delta t^3). \label{eq:velocity_taylor}
\end{align}
We now need an expression for $\od[2]{\vvec(t)}{t}$, which can be found by a Taylor expansion of $\od{\vvec(t+\Delta t)}{t}$
\begin{align*}
    \dod{\vvec(t+\Delta t)}{t} = \dod{\vvec(t)}{t} + \dod[2]{\vvec(t)}{t}\Delta t + \mathcal{O}(\Delta t^2),
\end{align*}
which by rearranging and multiplying with $\frac{\Delta t}{2}$ gives
\begin{align*}
    \dod[2]{\vvec}{t}\frac{\Delta t^2}{2} 
    &= \left( \dod{\vvec(t+\Delta t)}{t} - \dod{\vvec(t)}{t}\right)\frac{\Delta t}{2} + \mathcal{O}(\Delta t^3) \\
    &= \big[\avec(t + \Delta t) - \avec(t)\big] \frac{\Delta t}{2} + \mathcal{O}(\Delta t^3).
\end{align*}
Inserting this into \cref{eq:velocity_taylor} we get
\begin{align}
    \vvec(t + \Delta t) 
    &= \vvec(t) + \avec(t)\Delta t + \big[\avec(t + \Delta t) - \avec(t)\big] \frac{\Delta t}{2} + \mathcal{O}(\Delta t^3) \nonumber\\
    &= \vvec(t) + \big[\avec(t) + \avec(t + \Delta t)\big] \frac{\Delta t}{2} + \mathcal{O}(\Delta t^3). %\label{eq:verlet_velocity_taylor_insertion}
\end{align}
So the total velocity Verlet algorithm with truncation of the higher-order terms is
\begin{align}
    \rvec(t + \Delta t) &= \rvec(t) + \vvec(t)\Delta t + \avec(t)\frac{\Delta t^2}{2}, %\label{eq:velocity_verlet_position}\\
    \vvec(t + \Delta t) &= \vvec(t) + \big[\avec(t) + \avec(t + \Delta t)\big] \frac{\Delta t}{2}, %\label{eq:velocity_verlet_velocity}
\end{align}
with the truncation error for one timestep $\Delta t$ being of the order $\mathcal{O}(\Delta t^3)$ for both the position and the velocity. 

The algorithm is usually rewritten in the following way, to optimize the implementation on a computer. We see that the new velocities can be written as
\begin{align}
    \vvec(t+\Delta t) = \tilde\vvec(t + \tfrac{1}{2}\Delta t) + \avec(t+\Delta t)\frac{\Delta t}{2}, %\label{eq:verlet_velocity_with_halfstep}
\end{align}
where
\begin{align}
    \tilde\vvec(t + \tfrac{1}{2}\Delta t) = \vvec(t) + \avec(t)\frac{\Delta t}{2}. %\label{eq:verlet_halfstep}
\end{align}
We see that \cref{eq:verlet_halfstep} can be used in updating the positions, so we rewrite \cref{eq:velocity_verlet_position} to
\begin{align}
    \rvec(t + \Delta t) &= \rvec(t) + \tilde\vvec(t+\tfrac{1}{2}\Delta t)\Delta t. %\label{eq:velocity_verlet_positions_halfstep}
\end{align}
Which leads us to the usual way of implementing the algorithm\cite{allen1989computer}:
\begin{itemize}
    \item Calculate the velocities at $t+\tfrac{1}{2}\Delta t$ using \cref{eq:verlet_halfstep} \hl{(repeated here)}
    \begin{align*}
        \tilde\vvec(t + \tfrac{1}{2}\Delta t) = \vvec(t) + \frac{\Fvec(t)}{m}\frac{\Delta t}{2}.
    \end{align*}
    \item Calculate the new positions at $t + \Delta t$ using \cref{eq:velocity_verlet_positions_halfstep} \hl{(repeated here)}
    \begin{align*}
        \rvec(t + \Delta t) &= \rvec(t) + \tilde\vvec(t+\tfrac{1}{2}\Delta t)\Delta t.
    \end{align*}
    \item Calculate the new forces $\Fvec(t+\Delta t)$.
    \item Calculate the new velocities at $t+\Delta t$ using \cref{eq:verlet_velocity_with_halfstep} \hl{(repeated here)}
    \begin{align*}
        \vvec(t+\Delta t) = \vvec(t + \tfrac{1}{2}\Delta t) + \frac{\Fvec(t + \Delta t)}{m}\frac{\Delta t}{2}.
    \end{align*}
\end{itemize}
This implementation minimizes the memory needs, as we only need to store one copy of $\rvec$, $\vvec$ and $\Fvec$ at all times, compared to implementing \cref{eq:velocity_verlet_position,eq:velocity_verlet_velocity} which needs to store the values of both $\Fvec(t)$ and $\Fvec(t+\Delta)$ to calculate the new velocities. 

Pseudocode: \todo{do we want this?}
\begin{verbatim}
    v += F*dt/(2*m);
    r += v*dt;
    F = calculate_forces(r, v);
    v += F*dt/(2*m);
\end{verbatim}



% The velocity Verlet algorithm can be implemented as follows\cite{allen1989computer}
% \begin{samepage}
% \begin{itemize}
%     \item Calculate the new positions at $t + \Delta t$ using \cref{eq:velocity_verlet_position}.
%     \item Calculate the velocities at $t + \frac{1}{2}\Delta t$ using
%     \begin{align*}
%         \vvec(t + \tfrac{1}{2}\Delta t) = \vvec(t) + \frac{\Fvec(t)}{m}\frac{\Delta t}{2}.
%     \end{align*}
%     \item Calculate the new forces $\Fvec(t + \Delta t)$.
%     \item Finally, calculate the new velocities at $t + \Delta t$ using
%     \begin{align*}
%         \vvec(t + \Delta t) = \vvec(t + \tfrac{1}{2}\Delta t) + \frac{\Fvec(t + \Delta t)}{m}\frac{\Delta t}{2}
%     \end{align*}
% \end{itemize}
% \end{samepage}
% But 
% This implementation uses $9N$ units of memory, 

\subsection{Global error in the Verlet algorithm}
Sources:
\begin{itemize}
    \item \url{http://math.stackexchange.com/questions/668707/verlet-method-global-error}
    \item \url{http://www.saylor.org/site/wp-content/uploads/2011/06/MA221-6.1.pdf} (same as Wikipedia)
\end{itemize}

\todo{I don't know if the calculations below are correct} 
The global error in position can be derived from the local error for one timestep $\Delta t$. We see from \cref{eq:position_taylor} that
\begin{align*}
    \text{error}\big(\rvec(t_0+\Delta t)\big) = \mathcal{O}(\Delta t^3),
\end{align*}
and from \cref{eq:verlet_velocity_taylor_insertion}
\begin{align*}
    \text{error}\big(\vvec(t_0+\Delta t)\big) = \mathcal{O}(\Delta t^3).
\end{align*}
The error for two timesteps is
\begin{align*}
    \rvec(t_0+2\Delta t) 
    &= \rvec(t_0+\Delta t) + \vvec(t_0+\Delta t)\Delta t + \avec(t_0+\Delta t)\frac{\Delta t^2}{2} + \mathcal{O}(\Delta t^3), \\
\end{align*}
which gives \todo{Error in $\avec(t_0 + \Delta t) = \mathcal{O}(\Delta t^3)$ ??}
\begin{align*}
    &\text{error}\big(\rvec(t_0+2\Delta t)\big) \\
    &~= \text{error}\big(\rvec(t_0+\Delta t)\big) + \text{error}\big(\vvec(t_0+\Delta t)\big)\Delta t + \text{error}\big(\avec(t_0+\Delta t)\big)\frac{\Delta t^2}{2} + \mathcal{O}(\Delta t^3)\\
    &~= \mathcal{O}(\Delta t^3) + \mathcal{O}(\Delta t^3)\Delta t + \mathcal{O}(\Delta t^3)\frac{\Delta t^2}{2} + \mathcal{O}(\Delta t^3)\\
    &~= 2\mathcal{O}(\Delta t^3).
\end{align*}
Similarly we find
\begin{align*}
    &\text{error}\big(\rvec(t_0+3\Delta t)\big) = 3\mathcal{O}(\Delta t^3) \\
    &\text{error}\big(\rvec(t_0+4\Delta t)\big) = 4\mathcal{O}(\Delta t^3) \\
    &\text{error}\big(\rvec(t_0+5\Delta t)\big) = 5\mathcal{O}(\Delta t^3),
\end{align*}
\begin{align*}
    \text{error}\big(\rvec(t_0 + n\Delta t)\big) = n\mathcal{O}(\Delta t^3) = \mathcal{O}(\Delta t^2)
\end{align*}
