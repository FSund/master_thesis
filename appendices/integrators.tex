\chapter{Verlet integrators\label{appendix:verlet_integrators}}
The Verlet algorithm\cite{verlet1967computer} is a simple method for numerically integrating second order differential equations of the form 
\begin{align*}
    \dod[2]{\rvec(t)}{t} = \Fvec\big[\rvec(t), t\big] = \Fvec(t).
\end{align*}
The algorithm has several equivalent forms, and the form originally used by Verlet is
\begin{align*}
    \rvec(t + \Delta t) \approx 2\rvec(t) - \rvec(t - \Delta t) + \avec(t)\Delta t^2,
\end{align*}
where $\Delta t$ is the timestep, and $\avec(t)$ is the velocity at time $t$. An equivalent formulation, usually called the velocity Verlet algorithm, has the form
\begin{align*}
    \rvec(t + \Delta t) &= \rvec(t) + \vvec(t)\Delta t + \frac{\Fvec(t)}{2m}\Delta t^2 \\
    \vvec(t + \Delta t) &= \vvec(t) + \frac{\Fvec(t + \Delta t) + \Fvec(t)}{2m}\Delta t.
\end{align*}
The velocity Verlet algorithm is the most used form of the algorithm, and it has a accumulated error of $\mathcal{O}(\Delta t^2)$, as we show in \cref{subsec:verlet_error_liouville}. %\hl{why?}, and \hl{something about errors, references to below}.

We will now first derive the regular Verlet and the velocity Verlet algorithms using Taylor expansions, and then using the Liouville formulation of classical mechanics.
% \todo{why use two methods? what do we learn from them?} \todo{mention what errors we find?}
