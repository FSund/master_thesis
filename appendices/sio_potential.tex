\section{SiO2 potential (move to appendix?)}
\orangebox{
    \begin{itemize}
        \item Which of the parameters given below are constants?
        \item Does the potential care about number of neighbors (for example via adjustment of some parameters) ?
    \end{itemize}
}





\section{Potential}


The interatomic potential\cite{vashishta1990interaction} \todo{not exactly same as vashishta, see nanobubble supplements} we use for for both silica and water consists of two-body and three-body terms, and has the form
\begin{align*}
    E_\text{tot} = \sum_{i<j} V_{ij}^{(2)}(r_{ij}) + \sum_{i<j<k}V_{ijk}^{(3)}(\rvec_{ij},\rvec_{ij}),
\end{align*}
for
\begin{align*}
    1\leq \{i,j,k\} \leq N.
\end{align*}


$V_{ij}^{(2)}$ is the two-body term, which consists of four terms that take into account steric repulsion, charge-charge (Coulomb), charge-dipole, and dipole-dipole (van der Waals) interactions, and has the form
\begin{align*}
    V_{ij}^{(2)} (r) = 
    \underbrace{
        \frac{H_{ij}}{r^{\eta_{ij}}}
    }_{\text{steric repulsion}}
    +~ 
    \underbrace{
        \frac{Z_iZ_j}{r}e^{-r/r_{1\text{s}}}
    }_{\text{Coulomb}}
    ~-~
    \underbrace{
        \frac{D_{ij}}{2r^4}e^{-r/r_{4\text{s}}}
    }_{\text{charge-dipole}}
    ~- 
    \underbrace{
        \frac{w_{ij}}{r^6}
    }_{\text{van der Waals}}
    ,
\end{align*}
where $r$ is the distance between two atoms $i$ and $j$, $H_{ij}$ and $\eta_{ij}$ are the strengths of the steric repulsion, $Z_i$ is the charge associated with atom $i$, $D_{ij}$ \hl{controls the charge-dipole interaction}, $w_{ij}$ \hl{controls the dipole-dipole interaction}, and $r_{1\text{s}}$ and $r_{4\text{s}}$ are the screening lengths for the Coulumb and charge-dipole interactions respectively. 

$V_{ijk}^{(3)}$ is the three-body term, which take into account bending and stretching of covalent bonds, and has the form\todo{split into $f(\rvec_{ij}, \rvec_{ik})$ and $p(\theta_{ijk}, \theta_0)$ as in vashista 1990?}
\begin{align*}
    V^{(3)}_{jik}(\rvec_{ij}, \rvec_{ik}) = 
    B_{ijk} 
    \underbrace{ % use vphantom to get proper vertical alignment
        \vphantom{\frac{\big)^2}{\big)^2}}\exp\left( \frac{\xi}{r_{ij} - r_0} + \frac{\xi}{r_{ij} - r_0} \right)
    }_{\text{bond-stretching}}
    \underbrace{
        \frac{\left(\cos\theta_{ijk} - \cos\theta_0\right)^2}{1 + C_{ijk}\left(\cos\theta_{ijk} - \cos\theta_0\right)^2}
    }_{\text{bond-bending}}
    ,
\end{align*}
for
\begin{align*}
    \{r_{ij}, r_{ik}\} \leq r_0,
\end{align*}
where $\rvec_{ij} = \rvec_i - \rvec_j$, $r_{ij} = |\rvec_{ij}|$, $r_0$ is the cutoff distance for the three-body interaction, $\theta_{ijk}$ is the angle between $\rvec_{ij}$ and $\rvec_{ik}$, $B_{ijk}$ is the strength of the three-body interaction, and $\theta_0$ is a parameter that controls the angle at which the three-body term vanishes.
\todo[inline]{$\xi$ and $C_{ijk}$ ???}
\todo[inline]{cuts off interaction at $r_0$ with no discontinuities in the derivatives with respect to $r$ (vashista 1990)}
