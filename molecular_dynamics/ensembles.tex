\chapter{Ensembles\label{chap:ensembles}}
% \orangebox{
% \begin{itemize}
%     \item canonical = constant $N$, $V$ and $T$
%     \item microcanonical = constant $N$, $V$ and $E$
% \end{itemize}
% }
\todoao{Clean up ensembles intro}%
\todo{see \cite[section 3.7]{griebel2007numerical} for good intro to thermostats/ensembles}%
\todoco{small discussion of how heat bath really works? see thijssen p. 203(216)}%
%
% \section{Ensemble/thermostats}
We have now developed a molecular dynamics program that simulates simple systems like Argon, with constant number of particles $N$, constant volume $V$ and constant energy $E$, which means that the system can not exchange particles or energy with the environment. This forms a statistical ensemble called the microcanonical ensemble or the $NVE$-ensemble, which has some implications for what we can simulate and measure. Other ensembles that might be of interest is the canonical ensemble (constant $N$, $V$ and temperature $T$) and the grand canonical ensemble (constant $N$, pressure $P$, and temperature $T$).

% \hl{To study the canonical ensemble we can use a thermostat, which is a method for controlling the temperature of the system.} \todo{transition to parts below}

The most often studied ensemble other than the microcanonical is the canonical, with constant temperature instead of energy. To study this we need a way of controlling the temperature of the system, which is usually done by using a \emph{thermostat}. A thermostat simulates the system being in contact with a heat bath with temperature $T_\text{bath}$. The temperature is defined via the kinetic energy of the system, so we know we have to somehow control and modify the velocities of the atoms in the system to control the temperature.
% 
% \orangebox{
%     \begin{itemize}
%         \item Thermostats == change ensemble
%         \item Measure pressure, Frenkel eq. (3.4.1) p. 52.
%         \item Measure temperature, Frenkel eq. (4.1.1) and (4.1.2) p. 64
%     \end{itemize}
% }