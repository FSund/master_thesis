\chapter{SiO/water potential and program}
To study an advanced system like silica and water\hl{(, which consists of hydrogen (H), oxygen (O) and silicon (Si) atoms,)} we need a more advanced potential than the Lennard-Jones potential. We know water likes to have a H-O-H angle of around 105, and similarly we know that silica usually appears in tetrahedral structures. Structures like usually don't appear if we use a simple diatomic potential like Lennard-Jones, so we need to use a higher order potential. We would also like to be able to simulate chemical reactions we know happen, like transfer of oxygen atoms from silica to water.

The potential implemented in the program we use in this thesis has been developed at the University of Southern California (USC) by P. Vashishta, Rajiv K. Kalia, José P. Rino and Ingvar Ebbsj\"o\cite{vashishta1990interaction}, see also\cite{shekhar2013nanobubble,shekhar2013nanobubble_supplements}. The potential contains two-body and three-body terms, and allows oxygen atoms to be transferred from silica to water. See \cref{sec:sio2_potential} for more details on the potential.

The potential is implemented in a FORTRAN 77 program which has been parallellized an highly optimized \hl{source?}. \hl{The program is unfortunately closed-source}.

The program implements a Nos\'e-Hoover chain to sample the microcanonical \hl{(or the canonical?)} ensemble, using reversible multiple time-scale integrator derived using the Trotter factorization of the Liouville propagator\cite{tuckerman1992reversible}. The program also has a reversible integrator for NVT (canonical) and NPT (isothermal–isobaric) ensembles\cite{martyna1996explicit}. An article where the program has been used to study silica and water can be seen in\cite{shekhar2013nanobubble}. See also the supplements to the article\cite{shekhar2013nanobubble_supplements}.

\orangebox{
    \begin{itemize}
        \item Variable time step integrator (hydrogen vibrations/oscillations much faster than Si-O movement/vibration)
        \item Advanced thermostat
        \item SiO2-oxygen vs. H2O-oxygen
    \end{itemize}
}

% \begin{cppcode}
% c  Reversible extended (NVT or NPT) molecular dynamics (MD) for 
% c  multicomponent systems.
% c
% c  Multiple time-scale NPT dynamics with the XI scheme.
% c
% c  Constant shear-rate simulation by increasing H(3,2,0).  (Atoms 
% c  within distance YEDGE from the Y edges are frozen.)
% c
% c  ABOUT THE PROGRAM
% c
% c  1. Reversible integrator for NVT & NPT ensembles [G.J.Martyna, M.E.
% c     Tuckerman, D.J.Tobias, & M.L.Klein, Mol.Phys. 87, 1117 (1996)].
% c  2. Canonical MD by Nose-Hoover chain [Martyna et al, above].
% c  3. Hydrostatic variable-shape MD [Martyna et al., above;
% c     M.Parrinello & A.Rahman, J.Appl.Phys. 52, 7182 (1981)].
% c  4. Reversible multiple time-scale (MTS) scheme [M.Tuckerman, B.J. 
% c     Berne, & G.J.Martyna, J.Chem.Phys. 97, 1990 (1992)]; linked-cell
% c     & neighbor lists used for long- & short-range force calculations.
% c  5. MPI parallel program using spatial-decomposition [D.C.Rapaport,
% c     Comput.Phys.Commun. 62, 217 (1991)] & 6-step message passing.
% c  6. Interatomic potential [P.Vashishta et al., Phys.Rev. B 41,12197
% c     (1990)] is modified to a screened-Coulomb form & is truncated;
% c     3-body potential is modified to a fractional form; multicomponent
% c     extention is done by averaging.
% c  7. Atomic unit is used throughout the program.
% \end{cppcode}

\section{Potential\label{sec:sio2_potential}}
\orangebox{
    \begin{itemize}
        \item Which of the parameters given below are constants?
        \item Does the potential care about number of neighbors (for example via adjustment of some parameters) ?
    \end{itemize}
}

The interatomic potential\cite{vashishta1990interaction} \todo{not exactly same as vashishta, see nanobubble supplements} we use for for both silica and water consists of two-body and three-body terms, and has the form
\begin{align*}
    E_\text{tot} = \sum_{i<j} V_{ij}^{(2)}(r_{ij}) + \sum_{i<j<k}V_{ijk}^{(3)}(\rvec_{ij},\rvec_{ij}),
\end{align*}
for
\begin{align*}
    1\leq \{i,j,k\} \leq N.
\end{align*}


$V_{ij}^{(2)}$ is the two-body term, which consists of four terms that take into account steric repulsion, charge-charge (Coulomb), charge-dipole, and dipole-dipole (van der Waals) interactions, and has the form
\begin{align*}
    V_{ij}^{(2)} (r) = 
    \underbrace{
        \frac{H_{ij}}{r^{\eta_{ij}}}
    }_{\text{steric repulsion}}
    +~ 
    \underbrace{
        \frac{Z_iZ_j}{r}e^{-r/r_{1\text{s}}}
    }_{\text{Coulomb}}
    ~-~
    \underbrace{
        \frac{D_{ij}}{2r^4}e^{-r/r_{4\text{s}}}
    }_{\text{charge-dipole}}
    ~- 
    \underbrace{
        \frac{w_{ij}}{r^6}
    }_{\text{van der Waals}}
    ,
\end{align*}
where $r$ is the distance between two atoms $i$ and $j$, $H_{ij}$ and $\eta_{ij}$ are the strengths of the steric repulsion, $Z_i$ is the charge associated with atom $i$, $D_{ij}$ \hl{controls the charge-dipole interaction}, $w_{ij}$ \hl{controls the dipole-dipole interaction}, and $r_{1\text{s}}$ and $r_{4\text{s}}$ are the screening lengths for the Coulumb and charge-dipole interactions respectively. 

$V_{ijk}^{(3)}$ is the three-body term, which take into account bending and stretching of covalent bonds, and has the form\todo{split into $f(\rvec_{ij}, \rvec_{ik})$ and $p(\theta_{ijk}, \theta_0)$ as in vashista 1990?}
\begin{align*}
    V^{(3)}_{jik}(\rvec_{ij}, \rvec_{ik}) = 
    B_{ijk} 
    \underbrace{ % use vphantom to get proper vertical alignment
        \vphantom{\frac{\big)^2}{\big)^2}}\exp\left( \frac{\xi}{r_{ij} - r_0} + \frac{\xi}{r_{ij} - r_0} \right)
    }_{\text{bond-stretching}}
    \underbrace{
        \frac{\left(\cos\theta_{ijk} - \cos\theta_0\right)^2}{1 + C_{ijk}\left(\cos\theta_{ijk} - \cos\theta_0\right)^2}
    }_{\text{bond-bending}}
    ,
\end{align*}
for
\begin{align*}
    \{r_{ij}, r_{ik}\} \leq r_0,
\end{align*}
where $\rvec_{ij} = \rvec_i - \rvec_j$, $r_{ij} = |\rvec_{ij}|$, $r_0$ is the cutoff distance for the three-body interaction, $\theta_{ijk}$ is the angle between $\rvec_{ij}$ and $\rvec_{ik}$, $B_{ijk}$ is the strength of the three-body interaction, and $\theta_0$ is a parameter that controls the angle at which the three-body term vanishes.
\todo[inline]{$\xi$ and $C_{ijk}$ ???}
\todo[inline]{cuts off interaction at $r_0$ with no discontinuities in the derivatives with respect to $r$ (vashista 1990)}

