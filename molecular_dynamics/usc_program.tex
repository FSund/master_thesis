\chapter{Molecular dynamics program used for simulations}
% \todoa{Finish writing about sio2/water program}%
To study an advanced system like silica and water we need a more advanced potential than the Lennard-Jones potential. We know water likes to have a certain H-O-H angle, and that water molecules have strong hydrogen bonds between them that make water have some unique properties. Similarly we know that silica usually appears in tetrahedral structures, with certain Si-O-Si angles more stable than others. To simulate the forces that make structures like that appear we need to use something more advanced than a diatomic potential Lennard-Jones, so we use a higher order potential. We would also like to be able to simulate chemical reactions like transfer of oxygen atoms from silica to water, which requires further modifications of the simulations.

The potential implemented in the program we use in this thesis has been developed at the University of Southern California (USC) by P. Vashishta, Rajiv K. Kalia, José P. Rino and Ingvar Ebbsj\"o\cite{vashishta1990interaction} (see also\cite{shekhar2013nanobubble,shekhar2013nanobubble_supplements}). The potential contains two-body and three-body terms, and allows oxygen atoms to be transferred between silica and water. See \cref{sec:sio2_potential} for more details on the potential.

The potential is implemented in a \mono{FORTRAN 77} program, which has been parallellized and highly optimized for running on high-performance computing clusters like Abel. The program is unfortunately closed-source, and we are not allowed to distribute this openly.

The program implements both the fast and simple Berendsen thermostat and the more sophisticated Nos\'e-Hoover chains thermostat to simulate the $NPT$-ensemble, and a thermostat for the NPT ensemble. It also implements so-called variable time-step integrators, which saves CPU-time when we have silica and water in the same system. See \cref{sec:sio2_integrator} for more information about this.

An article where the program has been used to study silica and water can be seen in\cite{shekhar2013nanobubble}, where they do simulations of systems with 1 billion atoms. See also the supplements to the article\cite{shekhar2013nanobubble_supplements}.

% \todob{Timestep set according to vibrational frequency of water?}

% \orangebox{
%     \begin{itemize}
%         \item Variable time step integrator (hydrogen vibrations/oscillations much faster than Si-O movement/vibration)
%         \item Advanced thermostat
%         \item SiO2-oxygen vs. H2O-oxygen
%     \end{itemize}
% }

% \begin{cppcode}
% c  Reversible extended (NVT or NPT) molecular dynamics (MD) for 
% c  multicomponent systems.
% c
% c  Multiple time-scale NPT dynamics with the XI scheme.
% c
% c  Constant shear-rate simulation by increasing H(3,2,0).  (Atoms 
% c  within distance YEDGE from the Y edges are frozen.)
% c
% c  ABOUT THE PROGRAM
% c
% c  1. Reversible integrator for NVT & NPT ensembles [G.J.Martyna, M.E.
% c     Tuckerman, D.J.Tobias, & M.L.Klein, Mol.Phys. 87, 1117 (1996)].
% c  2. Canonical MD by Nose-Hoover chain [Martyna et al, above].
% c  3. Hydrostatic variable-shape MD [Martyna et al., above;
% c     M.Parrinello & A.Rahman, J.Appl.Phys. 52, 7182 (1981)].
% c  4. Reversible multiple time-scale (MTS) scheme [M.Tuckerman, B.J. 
% c     Berne, & G.J.Martyna, J.Chem.Phys. 97, 1990 (1992)]; linked-cell
% c     & neighbor lists used for long- & short-range force calculations.
% c  5. MPI parallel program using spatial-decomposition [D.C.Rapaport,
% c     Comput.Phys.Commun. 62, 217 (1991)] & 6-step message passing.
% c  6. Interatomic potential [P.Vashishta et al., Phys.Rev. B 41,12197
% c     (1990)] is modified to a screened-Coulomb form & is truncated;
% c     3-body potential is modified to a fractional form; multicomponent
% c     extention is done by averaging.
% c  7. Atomic unit is used throughout the program.
% \end{cppcode}