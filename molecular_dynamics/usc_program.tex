\chapter{Molecular dynamics program used for simulations}
% \todoa{Finish writing about sio2/water program}%
To study an advanced system like silica and water we need a more advanced potential than the Lennard-Jones potential. We know water likes to have a certain H-O-H angle, and that water molecules have strong hydrogen bonds between them that make water have some unique properties. Similarly we know that silica usually appears in tetrahedral structures, with certain Si-O-Si angles more stable than others. To simulate the forces that make structures like that appear we need to use something more advanced than a diatomic potential Lennard-Jones, so we use a higher order potential. We would also like to be able to simulate chemical reactions like transfer of oxygen atoms from silica to water, which requires further modifications of the simulations.

The potential implemented in the program we use in this thesis has been developed at the University of Southern California (USC) by P. Vashishta, Rajiv K. Kalia, José P. Rino and Ingvar Ebbsj\"o\cite{vashishta1990interaction} (see also\cite{shekhar2013nanobubble,shekhar2013nanobubble_supplements}). The potential contains two-body and three-body terms, and allows oxygen atoms to be transferred between silica and water. See \cref{sec:sio2_potential} for more details on the potential.

The potential is implemented in a \mono{FORTRAN 77} program, which has been parallellized and highly optimized for running on high-performance computing clusters like Abel. The program is unfortunately closed-source, and we are not allowed to distribute this openly.

The program implements both the fast and simple Berendsen thermostat and the more sophisticated Nos\'e-Hoover chains thermostat to simulate the $NPT$-ensemble, and a thermostat for the NPT ensemble. It also implements so-called variable time-step integrators, which saves CPU-time when we have silica and water in the same system. See \cref{sec:sio2_integrator} for more information about this.

An article where the program has been used to study silica and water can be seen in\cite{shekhar2013nanobubble}, where they do simulations of systems with 1 billion atoms. See also the supplements to the article\cite{shekhar2013nanobubble_supplements}.

% \todob{Timestep set according to vibrational frequency of water?}

% \orangebox{
%     \begin{itemize}
%         \item Variable time step integrator (hydrogen vibrations/oscillations much faster than Si-O movement/vibration)
%         \item Advanced thermostat
%         \item SiO2-oxygen vs. H2O-oxygen
%     \end{itemize}
% }

% \begin{cppcode}
% c  Reversible extended (NVT or NPT) molecular dynamics (MD) for 
% c  multicomponent systems.
% c
% c  Multiple time-scale NPT dynamics with the XI scheme.
% c
% c  Constant shear-rate simulation by increasing H(3,2,0).  (Atoms 
% c  within distance YEDGE from the Y edges are frozen.)
% c
% c  ABOUT THE PROGRAM
% c
% c  1. Reversible integrator for NVT & NPT ensembles [G.J.Martyna, M.E.
% c     Tuckerman, D.J.Tobias, & M.L.Klein, Mol.Phys. 87, 1117 (1996)].
% c  2. Canonical MD by Nose-Hoover chain [Martyna et al, above].
% c  3. Hydrostatic variable-shape MD [Martyna et al., above;
% c     M.Parrinello & A.Rahman, J.Appl.Phys. 52, 7182 (1981)].
% c  4. Reversible multiple time-scale (MTS) scheme [M.Tuckerman, B.J. 
% c     Berne, & G.J.Martyna, J.Chem.Phys. 97, 1990 (1992)]; linked-cell
% c     & neighbor lists used for long- & short-range force calculations.
% c  5. MPI parallel program using spatial-decomposition [D.C.Rapaport,
% c     Comput.Phys.Commun. 62, 217 (1991)] & 6-step message passing.
% c  6. Interatomic potential [P.Vashishta et al., Phys.Rev. B 41,12197
% c     (1990)] is modified to a screened-Coulomb form & is truncated;
% c     3-body potential is modified to a fractional form; multicomponent
% c     extention is done by averaging.
% c  7. Atomic unit is used throughout the program.
% \end{cppcode}

\section{Potential\label{sec:sio2_potential}}
% \orangebox{
%     \begin{itemize}
%         \item Which of the parameters given below are constants?
%         \item Does the potential care about number of neighbors (for example via adjustment of some parameters) ?
%     \end{itemize}
% }
%
The interatomic potential\cite{vashishta1990interaction} % \todoco{not exactly same as vashishta, see nanobubble supplements} 
we use for for both silica and water consists of a two-body and a three-body part, and has the form
\begin{align*}
    E_\text{tot} = \sum_{i<j} V_{ij}^{(2)}(r_{ij}) + \sum_{i<j<k}V_{ijk}^{(3)}(\rvec_{ij},\rvec_{ij}),
\end{align*}
for
\begin{align*}
    1\leq \{i,j,k\} \leq N.
\end{align*}

$V_{ij}^{(2)}$ is the two-body term, which consists of four terms that take into account steric repulsion, charge-charge (Coulomb), charge-dipole, and dipole-dipole (van der Waals) interactions. The two-body term only depends on the interatomic distance between atom $i$ and $j$, $|\rvec_{ij}| = r_{ij} = r$, and it has the form%
\begin{align*}
    V_{ij}^{(2)} (r) = 
    \underbrace{
        \frac{H_{ij}}{r^{\eta_{ij}}}
    }_{\text{steric repulsion}}
    +~ 
    \underbrace{
        \frac{Z_iZ_j}{r}e^{-r/r_{1\text{s}}}
    }_{\text{Coulomb}}
    ~-~
    \underbrace{
        \frac{D_{ij}}{2r^4}e^{-r/r_{4\text{s}}}
    }_{\text{charge-dipole}}
    ~- 
    \underbrace{
        \frac{w_{ij}}{r^6}
    }_{\text{van der Waals}}
    ,
\end{align*}%
% where $r$ is the distance between two atoms $i$ and $j$, $H_{ij}$ and 
where the parameters are%
\begin{itemize}[%
    leftmargin=*,%
    label={}%,
%     font=\itshape%
]
    \item \textit{Steric repulsion}
    \begin{itemize}[label=$\bullet$]
        \item $H_{ij}$ controls the strength of the steric repulsion
        \item $\eta_{ij}$ is the strength/exponent of the steric repulsion
    \end{itemize}
    \item \textit{Charge-charge (Coulomb) interaction}
    \begin{itemize}[label=$\bullet$]
        \item $Z_i$ is the charge associated with atom $i$
        \item $r_{1\text{s}}$ is the screening length for the interaction
    \end{itemize}
    \item \textit{Charge-dipole interaction}
    \begin{itemize}[label=$\bullet$]
        \item $D_{ij}$ controls the strength of the interaction
        \item $r_{4\text{s}}$ is the screening length for the interaction
    \end{itemize}
    \item \textit{Dipole-dipole (van der Waals) interaction}
    \begin{itemize}[label=$\bullet$]
        \item $w_{ij}$ controls the strength of the interaction
        \item $r_{4\text{s}}$ is the screening length for the interaction
    \end{itemize}
    \vspace{10pt}
\end{itemize}
% $\eta_{ij}$ are the strengths of the steric repulsion, $Z_i$ is the charge associated with atom $i$, $D_{ij}$ \hl{controls the charge-dipole interaction}, $w_{ij}$ \hl{controls the dipole-dipole interaction}, and $r_{1\text{s}}$ and $r_{4\text{s}}$ are the screening lengths for the Coulumb and charge-dipole interactions respectively. 

$V_{ijk}^{(3)}$ is the three-body term, which take into account bending and stretching of covalent bonds. This term depends on the distances between atom $i$, $j$ and $k$, and also the angle $\theta_{ijk}$ between the atoms. The term has the form %\todoco{split into $f(\rvec_{ij}, \rvec_{ik})$ and $p(\theta_{ijk}, \theta_0)$ as in vashista 1990?}
\begin{align*}
    V^{(3)}_{jik}(\rvec_{ij}, \rvec_{ik}) = 
    B_{ijk} 
    \underbrace{ % use vphantom to get proper vertical alignment
        \vphantom{\frac{\big)^2}{\big)^2}}\exp\left( \frac{\xi}{r_{ij} - r_0} + \frac{\xi}{r_{ij} - r_0} \right)
    }_{\text{bond-stretching}}
    \underbrace{
        \frac{\left(\cos\theta_{ijk} - \cos\theta_0\right)^2}{1 + C_{ijk}\left(\cos\theta_{ijk} - \cos\theta_0\right)^2}
    }_{\text{bond-bending}}
    ,
\end{align*}
for
\begin{align*}
    \{r_{ij}, r_{ik}\} \leq r_0.
\end{align*}
The parameters of the three-body term are as follows
\begin{itemize}[leftmargin=*,label={}]
    \item  % This needs to be here so the document compiles, but I don't think it adds any space
    \begin{itemize}[label=$\bullet$]
        \item $B_{ijk}$ controls the strength of the three-body interaction
    \end{itemize}
    \item \textit{Bond-stretching}
    \begin{itemize}[label=$\bullet$]
        \item $\xi$ controls the strength of the bond-stretching
        \item $r_0$ is the cutoff distance for the three-body interaction
    \end{itemize}
    \item \textit{Bond-bending}
    \begin{itemize}[label=$\bullet$]
        \item $C_{ijk}$ controls the strength of the bond-bending
        \item $\theta_{ijk}$ is the angle between $\rvec_{ij}$ and $\rvec_{ik}$
        \item $\theta_0$ is the angle at which the three-body term vanishes
    \end{itemize}
    \vspace{10pt}
\end{itemize}

The parameters used in our simulations were chosen by first determing good parameters for pure water and silica independently, and then interpolating between these parameters to allow transfer of oxygen atoms between silica and water. See for example \cite{vashishta1990interaction} for more details on this. The actual details of how oxygen is transferred between silica and water is based on the number of silicon and hydrogen neighbors an oxygen atom has, but the details of this method is outside the scope of this thesis.
% \todoa{More about how parameters were decided, see Mathilde's master, and [25] Wang et al in it}

The potential is further optimized by linearizing it, meaning that the potential is calculated for example for different distances $r_{ij}$ and angles $\theta_{ijk}$ when the program starts, and then the forces can be looked up directly in a table without having to evaluate the potential, when we run the simulations.

% where $\rvec_{ij} = \rvec_i - \rvec_j$, $r_{ij} = |\rvec_{ij}|$, $r_0$ is the cutoff distance for the three-body interaction, $\theta_{ijk}$ is the angle between $\rvec_{ij}$ and $\rvec_{ik}$, $B_{ijk}$ is the strength of the three-body interaction, and $\theta_0$ is a parameter that controls the angle at which the three-body term vanishes.
% \todo[inline]{$\xi$ and $C_{ijk}$ ???}
% \todod{cuts off interaction at $r_0$ with no discontinuities in the derivatives with respect to $r$ (vashista 1990)}

\section{Integrator\label{sec:sio2_integrator}}
% \todoa{Write about integrator in USC program here (already have something about it above), or remove section}
% \todob{Plot of energy conservation here?}
%
The program implements the Nos\'e-Hoover thermostat described in \cref{sec:nose_hoover_chain} to sample the microcanonical ensemble, using a reversible multiple time-scale integrator derived using the Trotter factorization of the Liouville propagator\cite{tuckerman1992reversible}, in a similar way to the way we derive the velocity Verlet algorithm in \cref{sec:liou}. The program also has a reversible integrator for NVT (canonical) and NPT (isothermal–isobaric) ensembles\cite{martyna1996explicit} derived in a similar way. 

See \cref{fig:energy_conservation_plot} for a plot of the energy over 100 000 timesteps. We see that the energy is very well conserved, with a relative increase in energy of just 2e-6 over 100 000 timesteps.
%
\begin{figure}[!htb]%
    \centering%
    \includesvg[width=0.6\textwidth, svgpath = ./images/energy_plots/]{system02_total_energy_thermalized_percentage}%
    \caption{%
        Plot of the relative energy change in a molecular dynamics simulation of water in nanoporous silica, with a total of approximately 400 000 atoms, 111k SiO$_2$ units and 19k H$_2$O-units. Simulations were done in the $NVE$-ensemble, and we have plotted 100 000 timesteps of 0.050 picoseconds.%
        \label{fig:energy_conservation_plot}%
    }%
\end{figure}%

The multiple time-scale integrator utilizes the fact that the vibrational frequencies in water is much higher than the frequencies of silica, which means that we can use larger timesteps to integrate the motions of the silica molecules than the water molecules. When using the program we use timesteps that are calculated from these vibrational frequencies, to make sure we have small enough timesteps. We used a main timestep of approximately 0.050 picoseconds in all simulations. \todoao{water vs silica timestep???} %\todobo{more about how timestep was chosen?}

% %
% \begin{figure}[htpb]%
%     \centering%
%     \includesvg[width=.7\textwidth, svgpath = ./images/energy_plots/]{total_energy01}%
%     \caption{Plot of total energy as function of timesteps in a thermalized nanoporous system filled with water.}%
% %     \label{fig:distance_to_atom_r20}%
% \end{figure}%