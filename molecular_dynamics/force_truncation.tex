\section{Optimization via force truncation\label{sec:cell_lists}}
% \todoa{Move optimizations to main md part?}
% \section{Verlet- and cell-lists\label{sec:cell_lists}}
% \todoa{What name should cell-lists have?}
% \todobo{We call it cell-lists, but the example program use ``box''}
If we try to simulate systems with a lot of atoms using the program we now have developed, we see that the number of evaluations of the potential quickly grow with the number of atoms, scaling as $\mathcal{O}(N^2)$. To optimize the program we can limit the number of evaluations by realizing that the Lennard-Jones potential decays as $r^{-12}$, meaning that the force between most atoms will be neglible (see \cref{fig:lennard-jones_potential} for a plot of the potential). 

Using the parameters for Argon, we find from \cref{eq:lennard-jones_potential} that the equilibrium distance of the potential is $r_{\text{eq}} =  2^{1/6}\sigma \approx 3.8 \text{~\AA}$. We also find that the value of the potential has decreased to $21\%$ of the equilibrium value at a distance $r_{ij} =  5.5$ \AA, and to $0.5\%$ of the equilibrium value at a distance $r_{ij} =  3\sigma \approx 10.2\text{~\AA}$. From this we decide to truncate the potential at a cutoff distance $r_\text{cutoff} = 3\sigma$.
%
\todoao{Add constant term to force, to avoid jumps in potential? THis has some implications and changes}%

The naive implementation of this cutoff length is to do a test inside the force calculation (for example in \mono{calculateForces} in \cref{list:calculate_forces}) and see if the distance between atom $i$ and $j$ is greater than the cutoff distance, $r_{ij} > r_\text{cutoff}$. This approach this still requires the calculation of a lot of interatomic distances, so what we do instead is to implement so-called \emph{cell-lists}. 

\subsection{Cell lists}
To truncate the force using cell-lists we divide the system into (3d) cells (or boxes) of size 
\begin{align*}
    l &= L/n,
\end{align*}
where $n$ is the number of cells in a direction. We can calculate the number of cells from the cutoff length $r_\text{cutoff}$ using the floor function $\lfloor x \rfloor$ as follows
\begin{align*}
    n &= \lfloor  L/r_\text{cutoff} \rfloor.
\end{align*}
Using the floor function guarantees that $l \geq r_\text{cutoff}$. 

The truncation is now done by only calculating the force between atom $i$ and all atoms in the cell it belongs to, and between it all atoms in the 26 neighboring cells. Since $l\geq r_\text{cutoff}$, this means that we include all atoms within a distance of at least $r_\text{cutoff}$ in the force calculations. See \cref{fig:cell_lists} for a 2-dimensional illustration of this.
\begin{figure}[htpb]%
    \centering%
    \includesvg[width=0.6\textwidth, svgpath=./images/lennard-jones/]{cell_list02}%
%     \includesvg[width=0.7\textwidth, svgpath=./images/lennard-jones/]{lennard-jones}%
%     \caption{%
%         An illustration of cell lists in 2 dimensions. We divide the system into cells of size $r \geq r_\text{cutoff}$, and calculate the force between atom $i$ and all atoms in the cell of that atom, and between that atom and all atoms in the 8 neighbor cells (26 neighbor cells in 3 dimensions). %
%         \label{fig:cell_lists}%
%     }%
    \caption{%
        An illustration of cell lists in 2 dimensions. We truncate the potential at $r_\text{cutoff}$ by only calculating the force between atom $i$ and all atoms in the cell of that atom, and between that atom and all atoms in the 8 neighbor cells (26 neighbor cells in 3 dimensions). %
        \label{fig:cell_lists}%
    }%
\end{figure}%

One issue that arises when using the cell lists is how to utilize Newton's third law. We see that after calculating the force between all atoms in cell $(i,j,k)$ and all atoms in the 26 neighboring cells, while adding the calculated forces to the other atoms using Newton's third law, we must not include that cell (cell $(i,j,k)$) in any other force calculations this timestep, or else we will get double contributions from atoms in that cell. One simple way of solving this is to keep a list of all cells we have calculated the forces on (and from) so far, and then check against that list every time we are calculating the force from a neighbor cell. But if we loop through the cells in the same order every time, we see that it would perhaps be more efficent to make a list over neighbor cells for each cell, so that we can just loop through a list of neighbors. The time spent on this in the simulations is nevertheless neglible compared to the evaluations of the potential.

An implementation of the calculation of forces using cell lists can be seen in \cref{list:cutoff_forcecalculation,list:sort_atoms_into_cells,list:calculateForceFromNeighborCells}. In these examples we do not use Newton's third law for optimization, to shorten the code and make the example simpler.
%
\begin{listing}[!htb]%
\begin{cppcode*}{gobble=4}
    void calculateForces(System &system) {
        ivec3 nCells;
        vector<Atom*> cells = 
            sortAtomsIntoCells(system, cutoffLength, nCells);
        
        // Loop over all cells
        for (int i = 0; i < nCells[0]; i ++)
        for (int j = 0; j < nCells[1]; j ++)
        for (int k = 0; k < nCells[2]; k ++)
        {{{
            for (Atom *atom : cells[i][j][k]) {
                atom->force() += 
                    calculateForceFromNeighborCells(
                        cells, nCells, atom, i, j, k
                    );
            }
        }}}
    }
\end{cppcode*}
\caption{%
    An example of an implementation of the force calculation \mono{calculateForces} from \cref{list:simple_md_program}, using the Lennard-Jones potential with a cutoff length for the force, and cell lists. Notice that we do not use Newton's third law, to simplify the example. %
    %Examples of how to calculate the forces using cell lists and Newton's third law are described \cref{sec:cell_lists}.%
    \label{list:cutoff_forcecalculation}%
}%
\end{listing}%
%
\begin{listing}[!htb]%
\begin{cppcode*}{gobble=4}
    vector<Atom*> sortAtomsIntoCells(
        System &system, double cutoffLength, ivec3 &nCells) {
        
        nCells = floor(system.size() / cutoffLength);
        ivec3 boxSize = system.size() / vec3(nCells);
        
        vector<Atom*> cells;
        for (Atom *atom : system.atoms()) {
            ivec3 index = floor(atom->position() / boxSize);
            cells[index[0]][index[1]][index[2]].push_back(atom);
        }
        return cells;
    }
\end{cppcode*}
\caption{%
    An example of an implementation of \mono{sortAtomsIntoCells} from \cref{list:cutoff_forcecalculation}. This listing shows how to sort atoms into cells for the cell list optimization described in \cref{sec:cell_lists}.%
    \label{list:sort_atoms_into_cells}%
}%
\end{listing}%
%
\begin{listing}[!htb]%
\begin{cppcode*}{gobble=4}
    vec3 calculateForceFromNeighborCells(
        vector<Atom*> &cells, ivec3 &nCells, Atom *atom1, 
        int i1, int j1, int k1) {
        
        vec3 force = zeros<vec3>();
        // Loop over 27 neighbor cells (including self)
        for (int di = -1; di <= 1; di++)
        for (int dj = -1; dj <= 1; dj++)
        for (int dk = -1; dk <= 1; dk++)
        {{{
            // Periodic boundary conditions
            int i2 = (i1 + di + nCells[0]) % nCells[0];
            int j2 = (j1 + dj + nCells[1]) % nCells[1];
            int k2 = (k1 + dk + nCells[2]) % nCells[2];
            
            // Loop over atoms in neighbor cell
            for (Atom *atom2 : cells[i2][j2][k2]) {
                if (atom1 == atom2) continue; // Skip i == j
                force() += calculateForceBetweenTwoAtoms(atom1, atom2);
            }
        }}}
        return force;
    }
\end{cppcode*}
\caption{%
    An example of an implementation of \mono{calculateForceFromNeighborCells} from \cref{list:cutoff_forcecalculation}. This listing shows how to calculate the force on an atom (\mono{atom1}), from the atoms in the cell it belongs to (\mono{cells[i1][j1][k1]}), and from the atoms in all 26 neighbor cells.%
    \label{list:calculateForceFromNeighborCells}%
}%
\end{listing}%

When implementing cell-lists we see that we need to have accurate lists of the atoms in each cell. But if we realize that the atoms rarely move very far in one timestep, we see that we can get away with only updating the cell-lists every other timestep. How often we need update the lists depends on the temperature (the average velocity of the atoms) and the timestep.

% \todo[inline]{scaling of cell lists compared to regular, and Verlet lists?}
% \todo[inline]{illustration of neighbor box vs. cutoff length?}
% \todoc{Update lists every other timestep, $r+dr$, }
