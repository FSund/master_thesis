\section{Nos\'e-Hoover thermostat and Nos\'e-Hoover chains}
The Nos\'e-Hoover thermostat is an advanced thermostat that generates a correct canonical ensemble and give very accurate dynamics\cite[section 6.1]{frenkel2001understanding}. The effect of applying Nosé-Hoover type thermostats it that we get an additional \emph{friction}-term to the forces on the atoms, instead of directly changing the velocities as with the Andersen and Berendsen thermostats.

\todoa{A bit more general about Nosé-Hoover thermostats, developmend history stuff}
\todob{We have derived the equations of motion we get when using the Nosé-Hoover thermostat in \cref{appendix:nose_hoover}, and repeat some of the results here.}

% A further development of an even more a These type of thermostats are in widespread use because of these properties, although they are a lot harder to implement than the Berendsen and Andersen thermostat. A derivation of the Nosé-Hoover thermostat can be seen in \cref{appendix:nose_hoover}, and some of the results are repeated here.

\subsection{Nosé-Hoover thermostat}
The simplest form of the Nosé-Hoover type thermostats is the Nosé-Hoover thermostat. This introduces a thermodynamic friction coefficient $\xi$ in the equations of motion, and we get the following equations of motion (the time derivative noted by a dot)
\begin{align}
    \dot \rvec &= \vvec \label{eq:nose_hoover_velocity}\\
    \Fvec &= -\nabla U(\rvec) - \xi m \vvec, \label{eq:nose_hoover_force}
\end{align}
where the friction coefficient generally depends on time, $\xi = \xi(t)$, and the choice of $\xi(t)$ determines the characteristics of the thermostat. See \cref{appendix:nose_hoover} for a derivation of this.

If we try to integrate these equations of motion using our integrator of choice, the velocity Verlet algorithm, a problem with using this thermostat will become apparent. The velocity Verlet integrator has the followin form (from \cref{eq:velocity_verlet_position,eq:velocity_verlet_velocity})
\begin{align*}
    \rvec(t + \Delta t) &= \rvec(t) + \vvec(t)\Delta t + \avec(t)\frac{\Delta t^2}{2}\\
    \vvec(t + \Delta t) &= \vvec(t) + \big[\avec(t) + \avec(t + \Delta t)\big] \frac{\Delta t}{2}.
\end{align*}
If we insert \cref{eq:nose_hoover_force} into these equations (using $\Fvec = m\avec$) we get
\begin{align*}
    \rvec(t + \Delta t) &= \rvec(t) + \vvec(t)\Delta t + \left[ -\frac{\nabla U(t)}{m} - \xi(t)\vec v(t) \right] \frac{\Delta t^2}{2}\\
    \vvec(t + \Delta t) 
        &= \vvec(t) + \Bigg[ 
%             \left(
                -\frac{\nabla U(t)}{m} - \xi(t)\vec v(t) \\
                &\phantom{= \vvec(t) + \Big[} -\frac{\nabla U(t+\Delta t)}{m} - \xi(t+\Delta t)\vec v(t+\Delta t)
%             \right)
        \Bigg] 
        \frac{\Delta t}{2},
\end{align*}
where $U(\rvec(t)) = U(t)$. The problem is now apparent: to calcuate $\vvec(t+\Delta t)$ we need to already know $\vvec(t+\Delta)$, which turns the integrator into a \emph{implicit} integrator. For this reason the Nosé-Hoover thermostat can be implemented using a predictor-corrector scheme, or solved iteratively\cite[Appendix E.2]{frenkel2001understanding}. This has the disadvantage that the solution is no longer time reversible. A solution to this has been proposed by Martyna et al. in \cite{martyna1996explicit}, where they develop a set of explicit reversible integrators using the Louiville approach for this type of extended systems, similar to the way we derive the velocity Verlet algorithm in \cref{appendix:liouville_verlet}.

\subsubsection{Choice of $\xi$}
In the so-called Nosé-Hoover thermostat the choice of $\xi$ is
\begin{align}
    \xi             &= \frac{sp_s}{Q},\label{eq:nose_hoover_xi}
\end{align}
and the time-evolution of $\xi$ is described by
\begin{align}
    \dot\xi         &= \left( \sum_{i=1}^N \frac{p_i^2}{m_i} - 3Nk_BT \right) / Q.\label{eq:nose_hoover_dot_xi}
\end{align}
Here $s$ is a degree of freedom introduced to the Lagrangian and Hamiltonian of the system, $p_s$ is the momentum associated with $s$, and $Q$ the ``mass'' associated with $s$. $Q$ controls the coupling to the heat bath, and has to be chosen with care. A large value for $Q$ leads to a weak coupling.

An important result derived by Hoover in \cite{hoover1986constant} is that this choice of $\xi$ (\cref{eq:nose_hoover_xi,eq:nose_hoover_dot_xi}) is the \emph{only} choice that can lead to a canonical distribution.

\subsection{Nosé-Hoover chains\label{sec:nose_hoover_chain}}
It can be shown that the Nosé-Hoover thermostat (\cref{eq:nose_hoover_xi,eq:nose_hoover_dot_xi}) only generates a correct canonical distribution for molecular systems in which there are no external forces, and the center of mass remains fixed \cite[Appendix B.2.1]{frenkel2001understanding}\cite{hoover1985canonical}. The last condition can be obeyed if we initialize the system with a net zero center-of-mass velocity, which we usually do in our simulations to avoid drift. If we want to simulate systems with an external force, for example to introduce flow, or a system where the center off mass is not fixed, we can use what is called Nosé-Hoover \emph{chains}, as proposed by Martyna et al.\cite{martyna1992nose}. 

Nosé-Hoover chains is a scheme where we use a Nosé-Hoover thermostat which is coupled to another thermostat, or a whole chain of thermostats. In the Nosé-Hoover chains thermostat\cite{martyna1992nose} we get the following equations of motion
\begin{align*}
    \dot\rvec &= \vvec \\
    \Fvec &= -\nabla U(\rvec) - \frac{p_{\xi,1}}{Q_1}m\vvec,
\end{align*}
which we see have a similar form to the equations we get when using a regular Nosé-Hoover thermostat (\cref{eq:nose_hoover_velocity,eq:nose_hoover_force}), with a friction term $\vvec p_{\xi,1}/Q_1$ added to the force on the atoms. The equations of motion for the thermostats, with $M$ coupled thermostats, are
\begin{align*}
    \dot \xi_k &= \frac{p_{\xi,k}}{Q_k} \qquad\text{for } k = 1,\dots,M \\
    \dot{p_{\xi,1}} &= \left( \sum_i \frac{p_i^2}{m_i} - 3Nk_BT \right) - \frac{p_{\xi,2}}{Q_2} p_{\xi,1} \\
    \dot{p_{\xi,k}} &= \left[ \frac{p_{\xi_k-1}^2}{Q_{k-1}} - k_BT \right] - \frac{p_{\xi,k+1}}{Q_{k+1}}p_{\xi,k} \\
    \dot{p_{\xi,M}} &= \left[ \frac{p_{\xi_M-1}^2}{Q_{M-1}} - k_BT \right].
\end{align*}
As with the Nosé-Hoover thermostat, the regular velocity Verlet integrator can not be used with this thermostat, but a derivation of an integrator for these equations of motion can be seen in \cite[Appendix E.2.1]{frenkel2001understanding} and \cite{martyna1996explicit}, where they again use the Liouville approach similar to the one we used to derive the velocity Verlet integrator in \cref{appendix:liouville_verlet}.

The Nosé-Hoover chains thermostat will generate a canonical distribution even in a system with external forces, and center of mass that is not fixed\cite[Appendix B.2.2]{frenkel2001understanding}, and will for example work well in a simulation where we introduce flow by some external force.


% Since the force depends on both the positions $\rvec$ and velocities $\vvec$, and our integrator of choice, the velocity Verlet algorithm, needs the forces in the next timestep to calculate the velocities in the next timestep, we turn the velocity Verlet integrator into a  (from \cref{eq:velocity_verlet_position,eq:velocity_verlet_velocity})
% \begin{align*}
%     \rvec(t + \Delta t) &= \rvec(t) + \vvec(t)\Delta t + \avec(t)\frac{\Delta t^2}{2}\\
%     \vvec(t + \Delta t) &= \vvec(t) + \big[\avec(t) + \avec(t + \Delta t)\big] \frac{\Delta t}{2}.
% \end{align*}
% Using the Nosé-Hoover equations of motion in \cref{eq:nose_xi,eq:nose_position,eq:nose_momentum,eq:nose_dotxi} we get
% \begin{align*}
%     \rvec(t + \Delta t) &= \rvec(t) + \vvec(t)\Delta t + \left[ \frac{\vec F(t)}{m} - \xi(t)\vec v(t) \right] \frac{\Delta t^2}{2}\\
%     \vvec(t + \Delta t) 
%         &= \vvec(t) + \Bigg[ 
%             \left(\frac{\vec F(t+\Delta t)}{m} - \xi(t+\Delta t)\vec v(t+\Delta t)\right) \\
%             &\phantom{= \vvec(t) + \Big[} + \left(\frac{\vec F(t)}{m} - \xi(t)\vec v(t)\right)
%         \Bigg] 
%         \frac{\Delta t}{2}.
% \end{align*}
% 
% 
% \begin{align}
%     \xi             &= \frac{sp_s}{Q} \label{eq:nose_xi}\\
%     \dot{\vec r}_i  &= \frac{\vec p_i}{m} \label{eq:nose_position}\\
%     \dot{\vec p}_i  &= -\dpd{U(\rvec)}{r_i} - \xi \vec p_i \label{eq:nose_momentum}\\
%     \dot\xi         &= \left( \sum_{i=1}^N \frac{p_i^2}{m_i} - 3Nk_BT \right) / Q, \label{eq:nose_dotxi}%\\
% %     \frac{\dot s}{s} &= \dod{\ln s}{t} = \xi
% \end{align}
% % Note that the last equation is redundant, since eq. 1-4 form a closed set
% where $s$ and $p_s$
% 
% A problem with the Nosé-Hoover thermostat is now apparent. Since the velocity appears on both sides of \cref{eq:nose_momentum}, we can't implement it directly in the velocity Verlet integrator. To see this we consider a standard microcanonical simulation \hl{(constant $N$, $V$, and $E$}. There the velocity Verlet algorithm is of the form
% (from \cref{eq:velocity_verlet_position,eq:velocity_verlet_velocity})
% \begin{align*}
%     \rvec(t + \Delta t) &= \rvec(t) + \vvec(t)\Delta t + \avec(t)\frac{\Delta t^2}{2}\\
%     \vvec(t + \Delta t) &= \vvec(t) + \big[\avec(t) + \avec(t + \Delta t)\big] \frac{\Delta t}{2}.
% \end{align*}
% Using the Nosé-Hoover equations of motion in \cref{eq:nose_xi,eq:nose_position,eq:nose_momentum,eq:nose_dotxi} we get
% \begin{align*}
%     \rvec(t + \Delta t) &= \rvec(t) + \vvec(t)\Delta t + \left[ \frac{\vec F(t)}{m} - \xi(t)\vec v(t) \right] \frac{\Delta t^2}{2}\\
%     \vvec(t + \Delta t) 
%         &= \vvec(t) + \Bigg[ 
%             \left(\frac{\vec F(t+\Delta t)}{m} - \xi(t+\Delta t)\vec v(t+\Delta t)\right) \\
%             &\phantom{= \vvec(t) + \Big[} + \left(\frac{\vec F(t)}{m} - \xi(t)\vec v(t)\right)
%         \Bigg] 
%         \frac{\Delta t}{2}.
% \end{align*}
% We see that calculating $\vec r(t+\Delta t)$ goes well, but to calculate $\vec v(t+\Delta t)$ we see that we need to know $\vec v(t+\Delta t)$ itself \hl{($\vec v(t+\Delta t)$ appears on both sides)}, turning the velocity Verlet integrator into an \emph{implicit} integrator. For this reason the Nosé-Hoover thermostat is usually implemented using a \hl{predictor-corrector scheme, or solved iteratively(ref [138] Frenkel, p. 535/555)}. This has the disadvantage that the solution is no longer time reversible. Martyna \emph{et al.}\cite{martyna1996explicit} has developed a set of explicit reversible integrators using the \hl{Louiville approach} for this type of extended systems.\todoa{this paragraph is copied from Frenkel p. 535/555..!}
% 
% It can be shown that the Nosé-Hoover scheme only generates a correct canonical distribution for molecular systems in which \hl{there is only one conserved quantity or if} there are no external forces and the center of mass remains fixed \cite{frenkel2001understanding}. The last condition can be obeyed if we initialize the system with a net zero center-of-mass velocit, which we usually do in our simulations, \hl{to avoid drift/flow}.