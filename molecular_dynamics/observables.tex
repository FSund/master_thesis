\chapter{Observables? Measurements?}
% To measure an observable quantity in a simulation we must be able to express it as a function of the positions, velocities and forces on the particles in the system. 
% 
% According to the equipartition principle, the average total kinetic energy $\langle E_k \rangle$ is
% \begin{align*}
%     \left\langle E_k \right\rangle = \left\langle \frac{1}{2}mv^2 \right\rangle = \frac{3}{2}Nk_B T,
% \end{align*}
% from which we can derive the temperature of the system. Here $m$ is the mass of a particle, $v$ is the speed of a particle, and $T$ is the corresponding temperature of the system.
% 
% An often used method for measuring the pressure $P$ is derived from the virial equation for the pressure, which gives
% \begin{align*}
%     P = \rho k_B T + \frac{1}{3V}\sum_i \sum_{j>i} \bvec F(\rvec_{ij}) \cdot \rvec_{ij},
% \end{align*}
% where $V$ is the volume, $\rho$ is the atom density, $\bvec F(\bvec r)$ is the force between two atoms separated by $\bvec r$, and $\rvec_{ij} = \rvec_j - \rvec_i$ is the vector between atom $i$ and atom $j$. \hl{1) this equation depends on the ensemble, and is only valid for micro-canonical ensemble -- project 1 FYS4460} \hl{2) this expression is derived for a system at constant $N$, $V$, and $T$, see Frenkel p. 84(104) sec. 4.4}.
%
%
%
\begin{listing}[!htb]%
\begin{cppcode*}{gobble=4}
    void sample(System &system)
    {
    }
\end{cppcode*}
\caption{%
    \hl{FINISH THIS LISTING}. Implementation of the function \texttt{sample} from \cref{list:simple_md_program}.%
    \label{list:sampling}%
}%
\end{listing}%
\todo{finish this listing}


\section{Temperature}
\todo[inline]{derivation of temperature? See Anderhaf}
According to the equipartition principle the average total kinetic energy \hl{per atom}, for a system consisting of $N$ particles with three degrees of freedom each, can be related to the temperature of the system via
\begin{align*}
    \Braket{E_k} = \frac{3}{2}Nk_\text{B}T,
\end{align*}
where $T$ is the temperature of the system. We calculate the average kinetic energy using
\begin{align*}
    \Braket{E_k} = \frac{1}{2N}\sum_{i=1}^N m_i v_i^2,
\end{align*}
where $m_i$ and $v_i = |\vec v_i|$ is respectively the mass and speed of atom $i$. 

See \cref{list:temperatureSample} for an example of how to calculate the temperature in a molecular dynamics simulation.
%
\begin{listing}[!htb]%
\begin{cppcode*}{gobble=4}
    double temperatureSample(System &system)
    {
        double kineticEnergy = 0.0;
        for (Atom *atom : system.atoms())
        {
            kineticEnergy += atom->velocity().lenghtSquared();
        }

        kineticEnergy *= 0.5;
        double temperature = 2.0*kineticEnergy/(3.0*system.nAtoms()
                                                *boltzmannConstant);  // SI units
        return temperature;
    }
\end{cppcode*}
\caption{%
    An example of how to calculate the temperature in a molecular dynamics simulation. Example implementation of \texttt{temperatureSample} from \cref{list:sampling}.%
    \label{list:temperatureSample}%
}%
\end{listing}%

\todo[inline]{Something about temperature in system with flow? (not that relevant for me though...)}

\section{Pressure\label{subsec:pressure}}
\todo[inline]{derivation of pressure? See Anderhaf}
To measure the pressure when using potentials with pairwise additive interactions, like the case is for our example program with the Lennard-Jones potential, we can use a method derived from the virial equation for the pressure\cite[Section~4.4]{frenkel2001understanding}. In a volume $V$ with particle density $\rho = N/V$, the average pressure is
% For pairwise additive interactions we can write\cite[Section~4.4]{frenkel2001understanding}
\begin{align}
    P = pk_\text{B}T + \frac{1}{dV} \Braket{\sum_{i<j} \bvec F (\bvec r_{ij}) \cdot \bvec r_{ij}},
    \label{eq:measure_pressure}
\end{align}
where $\vec F(\rvec_{ij})$ is the force between particle $i$ and $j$, and $\rvec_{ij}$ is the distance between the particles. \hl{Note that this expression for the pressure has been derived for a system at constant $N$, $V$ and $T$, whereas our simulations are performed at a constant $N$, $V$, and $E$}

We see that we need the force from each atom $j$ on atom $i$, $\vec F(\rvec_{ij})$ to calculate the pressure, so for efficiency we should calculate the contribution to the pressure, $\vec F(\rvecij)\cdot\rvecij$, while doing the force calculations. The contribution to the pressure should then be stored so we can calculate the average in \cref{eq:measure_pressure} later.

An example of how to calculate the pressure in a molecular dynamics simulation can be seen in \cref{list:pressureSample}.
%
\begin{listing}[!htb]%
\begin{cppcode*}{gobble=4}
    double pressureSample(System &system, double temperature)
    {
        double pressure;
        for (Atom *atom : system.atoms())
        {
            pressure += atom->pressure();
        }
        pressure /= (3.0*system.volume());
        double density = system.nAtoms()/system.volume(); // Assume homogeneous
        pressure += density*boltzmannConstant*temperature; // SI units
        return pressure;
    }
\end{cppcode*}
\caption{%
    An example of how to calculate the pressure in a molecular dynamics simulation. Example implementation of \texttt{pressureSample} from \cref{list:sampling}. Note that this function needs the temperature of the system as input, and assumes that the system is homogeneous, so we can estimate the density using $\rho = N/V$. We assume that the contribution to the pressure from each atom $\sum_{i<j}\vec F(\rvec_{ij})\cdot\rvec_{ij}$ (stored as \texttt{atom->pressure()}) has been calculated previously. This is usually calculated while calculating the forces between the atoms, since we need $\vec F(\rvec_{ij})$. See \cref{subsec:pressure} for more information.%
    \label{list:pressureSample}%
}%
\end{listing}%

\section{Diffusion}
\todo[inline]{derivation of diffusion? See \cite[Section~4.4.1]{frenkel2001understanding}}
We can measure the diffusion constant $D$ by measuring the square displacement $r_i^2(t)$ of each atom as a function of time, and averaging over all atoms. We measure the mean square displacement as
\begin{align*}
    \Braket{r^2(t)} = \frac{1}{N} \sum_{i=1}^N \left( \rvec_i(t) - \rvec_i(t=0) \right)^2,
\end{align*}
where $\rvec_i(t=0)$ is the initial position of atom $i$. From theoretical considerations of the diffusion process we can relate the diffusion constant to the mean square displacement through\cite[Section~4.4.1]{frenkel2001understanding}
\begin{align*}
    &\Braket{r^2(t)} = 2dDt &\text{when }t\rightarrow \infty,
\end{align*}
where $d$ is the \hl{spatial} dimensionality. This means that we can find the diffusion constant in a molecular dynamics simulation by measuring the mean square displacement for many timesteps, and plotting $\Braket{r^2(t)}/(2dt)$ as a function of time. The diffusion constant will then be the value this expression approaches when simulating for enough timesteps.

An example of how to measure the mean square displacement in a molecular dynamics simulation can be seen in \cref{list:diffusionSample}.
% Fick's law relates flux $\bvec j$ of diffusing species to the concentration gradient $\bvec \nabla c$ of the species
% \begin{align*}
%     \bvec j = -D\bvec \nabla c,
% \end{align*}
% where $D$ is the proportionality constant called the diffusion \hl{coefficient/constant}.
% 
% Conservation of \hl{species/atoms?}
% \begin{align*}
%     \dpd{c(r,t)}{t} + \div \cdot \vec j(r,t) = 0,
% \end{align*}
% which gives
% \begin{align}
%     \dpd{c(r,t)}{t} - D\nabla^2 c(r,t) = 0.
%     \label{eq:diffusion_diff}
% \end{align}
% % We can solve this equation with the boundary condition
% % \begin{align*}
% %     c(r,0) = \delta (r),
% % \end{align*}
% % where $\delta(r)$ is the Dirac delta function, which gives
% % \begin{align*}
% %     c(r,t) = \frac{1}{(4\pi Dt)^{d/2}} \exp\left(-\frac{r^2}{4Dt}\right).
% % \end{align*}
% By multiplying \cref{eq:diffusion_diff} by $r^2$ and integrating over all space we get
% \begin{align*}
%     \dpd{}{t}\int \drvec~ r^2 c(r,t) = D\int \drvec~ r^2 \nabla^2 c(r,t).
% \end{align*}
% We recognize the integral on the left-hand side as the the mean square displacement %\hl{time dependence of the second moment of $c(r,t)$???}
% \todo{what???}
% \begin{align*}
%     \int \drvec~ c(r,t) r^2 \equiv \Braket{r^2(t)},
% \end{align*}
% where we have imposed
% \begin{align*}
%     \int \drvec~ c(r,t) = 1.
% \end{align*}
% Applying partial integration to the right-hand side we obtain
% \begin{align*}
%     \dpd{\Braket{r^2(t)}}{t} 
%     &= D\int \drvec~ r^2 \nabla^2 c(r,t) \\
%     &= D\int \drvec~ \div \cdot \left(r^2\nabla c(r,t)\right) - D\int\drvec~ \div r^2 \cdot \nabla c(r,t) \\
%     &= D\int \dif \vec S~ \left(r^2\div c(r,t)\right) - 2D\int\drvec~ \rvec \cdot\div c(r,t) \\
%     &= 0 - 2D\int\drvec~ (\div \cdot \rvec c(r,t)) + 2D\int\drvec~ (\div \cdot \rvec) c(r,t) \\
%     &= 0 + 2dD\int \drvec~ c(r,t) \\
%     &= 2dD
% \end{align*}
% 
% 
% \begin{align*}
%     \dpd{\Braket{r^2(t)}}{t} = 6D
% \end{align*}
% $\Rightarrow$ Plot mean square distance as function of time
% \begin{align*}
%     \Braket{\delta r(t)^2} = \frac{1}{N} \sum_{i=1}^N \delta \rvec_i(t)^2
% \end{align*}
% and find $6D$ as slope of plot \hl{(after a while?)}
%
\begin{listing}[!htb]%
\begin{cppcode*}{gobble=4}
    double diffusionSample(System &system)
    {
        double rSquared = 0.0;
        for (Atom *atom : system.atoms())
        {
            drVec = atom->positiom() - atom->initialPosition() 
                    + atom->getBoundaryCrossings()*system.size();
            rSquared += drVec.lengthSquared();
        }
        rSquared /= system.nAtoms();
        return rSquared;
    }
\end{cppcode*}
\caption{%
    An example of how to calculate the mean square displacement in a molecular dynamics simulation. Example implementation of \texttt{diffusionSample} from \cref{list:sampling}. We store the inital positions of the atoms as \texttt{atom->initialPosition()}, and when using periodic boundary conditions we count the number of times we have to translate the atom one system-size in each direction, so while the position of the atom will always be inside the system, the \hl{\emph{real}} position of the atom can be calculated by adding \texttt{atom->getBoundaryCrossings()*system.size()} to $\rvec$.%
    \label{list:diffusionSample}%
}%
\end{listing}%
