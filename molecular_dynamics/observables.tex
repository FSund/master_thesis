% \chapter{Observables? Measurements? Pysical quantities?}
\section{Observables}
% \todoa{Move observables part to md chapter, since it's so short after moving diffusion?, or move diffusion and density back to observables chapter?}
% \todoa{Intro to observables? Some stat mech stuff?}
% We now have an efficient molecular dynamics code, that can simulate simple atoms like Argon.
%
% To measure an observable quantity in a simulation we must be able to express it as a function of the positions, velocities and forces on the particles in the system. 
% 
% According to the equipartition principle, the average total kinetic energy $\langle E_k \rangle$ is
% \begin{align*}
%     \left\langle E_k \right\rangle = \left\langle \frac{1}{2}mv^2 \right\rangle = \frac{3}{2}Nk_B T,
% \end{align*}
% from which we can derive the temperature of the system. Here $m$ is the mass of a particle, $v$ is the speed of a particle, and $T$ is the corresponding temperature of the system.
% 
% An often used method for measuring the pressure $P$ is derived from the virial equation for the pressure, which gives
% \begin{align*}
%     P = \rho k_B T + \frac{1}{3V}\sum_i \sum_{j>i} \bvec F(\rvec_{ij}) \cdot \rvec_{ij},
% \end{align*}
% where $V$ is the volume, $\rho$ is the atom density, $\bvec F(\bvec r)$ is the force between two atoms separated by $\bvec r$, and $\rvec_{ij} = \rvec_j - \rvec_i$ is the vector between atom $i$ and atom $j$. \hl{1) this equation depends on the ensemble, and is only valid for micro-canonical ensemble -- project 1 FYS4460} \hl{2) this expression is derived for a system at constant $N$, $V$, and $T$, see Frenkel p. 84(104) sec. 4.4}.
%
%
%
%
\todoa{Write introduction to observables. Statistical mechanics?}
We are now ready to start doing some simple measurements on the systems we simulate. One of the most basic things we measure is the temperature and pressure in a system
\begin{listing}[!htb]%
\begin{cppcode*}{gobble=4}
    void sample(System &system) {
        double temperature = temperatureSample(system);
        double pressure = pressureSample(system, temperature);
        double rSquared = diffusionSample(system);        
    }
\end{cppcode*}
\caption{%
    Implementation of the function \mono{sample} from \cref{list:simple_md_program}. See %
    \cref{list:temperatureSample}, \cref{list:pressureSample}, and \cref{list:diffusionSample} %
%     \crefrange{list:temperatureSample}{list:diffusionSample} %
    for example implementation of the functions used.%
    \label{list:sampling}%
}%
\end{listing}%

\subsection{Temperature}
\todob{derivation of temperature?}
According to the equipartition principle the average total kinetic energy \hl{per atom}, for a system consisting of $N$ particles with three degrees of freedom each, can be related to the temperature of the system via
\begin{align*}
    \Braket{E_k} = \frac{3}{2}Nk_\text{B}T,
\end{align*}
where $T$ is the temperature of the system. We calculate the average kinetic energy of a system using
\begin{align*}
    \Braket{E_k} = \frac{1}{N}\sum_{i=1}^N \frac{1}{2} m_i v_i^2,
\end{align*}
where $m_i$ and $v_i = |\vec v_i|$ is respectively the mass and speed of atom $i$. From this we find the temperature of the system as
\begin{align*}
    T = \frac{2}{3} \frac{\Braket{E_k}}{Nk_B} = \frac{1}{3N^2k_B} \sum_{i=1}^N m_i v_i^2.
\end{align*}

See \cref{list:temperatureSample} for an example of how to calculate the temperature in a molecular dynamics \hl{simulation/program?} similar to the one we have outlined in \cref{chap:simple_md_program}.
%
\begin{listing}[!htb]%
\begin{cppcode*}{gobble=4}
    double temperatureSample(System &system) {
        double kineticEnergy = 0.0;
        for (Atom *atom : system.atoms()) {
            kineticEnergy += atom->velocity().lenghtSquared();
        }

        kineticEnergy *= 0.5;
        double temperature = 2.0*kineticEnergy/(3.0*system.nAtoms()
                                                *boltzmannConstant);  // SI units
        return temperature;
    }
\end{cppcode*}
\caption{%
    An example of how to calculate the temperature in a molecular dynamics simulation. Example implementation of \mono{temperatureSample} from \cref{list:sampling}.%
    \label{list:temperatureSample}%
}%
\end{listing}%

\todod{Something about temperature in system with flow? (not that relevant for me though...)}

\subsection{Pressure\label{subsec:pressure}}
\todoa{Mention that it's hard to measure pressure since we don't know the program, and thus can't easily calculate forces?}
\todob{derivation of pressure? See Anderhaf}
\todoc{we haven't measured pressure, so maybe remove this section?}
To measure the pressure when using potentials with pairwise additive interactions, like the case is for our example program with the Lennard-Jones potential, we can use a method derived from the virial equation for the pressure\cite[Section~4.4]{frenkel2001understanding}. In a volume $V$ with particle density $\rho = N/V$, the average pressure is
% For pairwise additive interactions we can write\cite[Section~4.4]{frenkel2001understanding}
\begin{align}
    P = pk_\text{B}T + \frac{1}{dV} \Braket{\sum_{i<j} \bvec F (\bvec r_{ij}) \cdot \bvec r_{ij}},
    \label{eq:measure_pressure}
\end{align}
where $\vec F(\rvec_{ij})$ is the force between particle $i$ and $j$, and $\rvec_{ij}$ is the distance between the particles. \hl{Note that this expression for the pressure has been derived for a system at constant $N$, $V$ and $T$, whereas our simulations are performed at a constant $N$, $V$, and $E$}

We see that we need the force from each atom $j$ on atom $i$, $\vec F(\rvec_{ij})$ to calculate the pressure, so for efficiency we should calculate the contribution to the pressure, $\vec F(\rvecij)\cdot\rvecij$, while doing the force calculations. The contribution to the pressure should then be stored so we can calculate the average in \cref{eq:measure_pressure} later.

An example of how to calculate the pressure in a molecular dynamics \hl{simulation/program} as outlined in \cref{chap:simple_md_program} can be seen in \cref{list:pressureSample}.
%
\begin{listing}[!htb]%
\begin{cppcode*}{gobble=4}
    double pressureSample(System &system, double temperature) {
        double pressure;
        for (Atom *atom : system.atoms()) {
            pressure += atom->pressure();
        }
        pressure /= (3.0*system.volume());
        double density = system.nAtoms()/system.volume(); // Assume homogeneous
        pressure += density*boltzmannConstant*temperature; // SI units
        return pressure;
    }
\end{cppcode*}
\caption{%
    An example of how to calculate the pressure in a molecular dynamics simulation. Example implementation of \mono{pressureSample} from \cref{list:sampling}. Note that this function needs the temperature of the system as input, and assumes that the system is homogeneous, so we can estimate the density using $\rho = N/V$. We assume that the contribution to the pressure from each atom $\sum_{i<j}\vec F(\rvec_{ij})\cdot\rvec_{ij}$ (stored as \mono{atom->pressure()}) has been calculated previously. This is usually calculated while calculating the forces between the atoms, since we need $\vec F(\rvec_{ij})$. See \cref{subsec:pressure} for more information.%
    \label{list:pressureSample}%
}%
\end{listing}%
