\chapter{Statistical mechanics}
% \hl{The idea behind Molecular Dynamics simulations is that we can study the average behaviour of a many-body system by computing the natural time evolution of the system numerically and averaging the quantity of interest over a sufficiently long time.} -- Frenkel p. 15
% 
% Before we start work on to do simulations using molecular dynamics, we need some basics on \emph{why} we can use MD to study atomic systems like silica and water. We first use the Born-Oppenheimer approximation to justify that the Hamiltonian of the system can be expressed as a function of the positions and velocities of the atoms, having averaged out the rapid motion of the individual subatomic particles (like electrons and protons). We then make the approximation that a classical description of the system is adequate, which we 
% 
% % To do this we need some \hl{something} from the field of statistical mechanics.
% 
% Using statistical mechanics stated in quantum mechanical terms, and using
% 
% The basic assumption behind statistical mechanics, from which much of statistical mechanics follows, is that a system with fixed number of particles $N$, volume $V$ and energy $E$ is equally likely to be found in any of its eigenstates. Using this assumption, and stating statistical mechanics in quantum mechanical terms, it can be shown that the probability for finding a system at temperature $T$ in quantum state $i$ is
% \begin{align}
% %     P_i = 
%     \frac{\exp\left(-\beta E_i\right)}{\sum_j \exp\left( -\beta E_j\right)},
%     \label{eq:boltzmann}
% \end{align}
% where $\beta = 1/k_B T$, $k_B$ is the Boltzmann constant, $E_i$ is the energy of a system in state $i$, and the sum in the divisor goes over all states at temperature $T$. This equation is the well-known Boltzmann distribution for a system at temperature $T$. From \cref{eq:boltzmann} it can be shown that the thermal average of an observable $A$ can be computed as
% \begin{align}
%     \langle A \rangle =
%     \frac{
%         \sum_i \exp \left( -\beta E_i \right) \Braket{i|A|i}
%     }{
%         \exp \left( -\beta E_i \right)
%     },
%     \label{eq:quantum_thermal_average}
% \end{align}
% where $\Braket{i|A|i}$ is the expectation value of the operator $A$ in quantum state $i$. The problem is that to compute a thermal average like this, we have to solve the \Schr~ equation for an arbitray many-body system, which is, in general, an untractable problem. Even if we could, the number of quantum states that contribute to the average in \cref{eq:quantum_thermal_average} would be so astronomically large ($\mathcal{O}(10^{10^{25}})$) that a numerical evaluation of all expectation values would be impossible.
% 
% Fortunately \cref{eq:quantum_thermal_average} can be simplified in the classical limit, where we consider \hl{actions} on a scale much larger than $\hbar$, meaning that we can ignore\todo{somethings}. In the classical limit we get
% \begin{align*}
%     \left\langle A\right\rangle = \frac
%     {
%         \bigint \dif \bvec r^N \dif \bvec p^N \exp 
%         \left\{
%             -\beta \Ham \left( \bvec r^N, \bvec p^N \right)
%         \right\} 
%         A\left(\bvec r^N, \bvec p^N\right)
%     }
%     {
%         \bigint \dif \bvec r^N \dif \bvec p^N \exp 
%         \left\{
%             -\beta \Ham\left( \bvec r^N, \bvec p^N \right)
%         \right\}
%     },
% \end{align*}
% where $\Ham$ is the Hamilton operator
% \begin{align*}
%     \Ham(\bvec r^N, \bvec p^N) = 
%         U(\rvec^N) + \sum_i^N \frac{p_i^2}{2m_i},
% \end{align*}
% $\bvec r^N$ is the position of all the atoms, $\bvec p^N$ is the corresponding momenta, $m_i$ is the mass of atom $i$, and $U$ is the potential energy of the system.
% 
% So far we have seen that the 
% 
% average value of an observable $A$ can be computed as
% \begin{align*}
%     \langle A \rangle =
%     \frac{
%         \sum_i \exp \left( -\beta E_i \right) \Braket{i|A|i}
%     }{
%         \exp \left( -\beta E_i \right)
%     },
% \end{align*}
% where $i$ denotes the quantum state of a system, the sum $\sum_i$ goes over all  $\beta = 1/k_B T$, $k_B$ is the Boltzmann constant, $T$ is the temperature, $E_i$ is the energy of a system 
% 
% 
% time average of a physical quantity
% \begin{align*}
%     \bar A = \lim_{T\rightarrow \infty} \frac{1}{T} \int_0^T A(t) \dif t
% \end{align*}
% then using classical mechanics, classical limit, NVE ensemble
% denoting the state of the system as $X$, and the sum over all states with a fixed energy $E$ as $\sum_{\{X|E\}}$
% \begin{align*}
%     \langle A \rangle = 
%     \frac{
% %         \displaystyle
%         \sum_{\{X|E\}} A(X)
%     }{
% %         \displaystyle
%         \sum_{\{X|E\}}
%     }
%     =
%     \frac{
% %         \displaystyle
%         \sum_X A(X) \deltaop \left[ \Ham(X) - E \right]
%     }{
% %         \displaystyle
%         \sum_X \deltaop \left[ \Ham(X) - E \right]
%     }
%     = \bar A
% \end{align*}
% in the sums on the right had side the delta-function $\deltaop[\Ham(X)-E]$ restricts the sum to states with energy $E$
% 
% 
% in the classical limit we can write the \hl{thermal???} average of an observable $A$ as\cite{frenkel2001understanding}\todo{eq. (2.2.6), p. 13 + eq. (3.1.2) p. 23}
% % \begin{align*}
% %     \left\langle A\right\rangle = \frac
% %     {
% %         \biggerint \dif \bvec p^N \dif \rvec^N \exp \left\{ -\beta
% %             \left[ 
% %                 \sum_i \frac{p_i^2}{2m_i} + U(\rvec^N)
% %             \right]
% %         \right\} 
% %         A(\bvec p^N, \bvec q^N)
% %     }
% %     {
% %         \biggerint \dif \bvec p^N \dif \rvec^N \exp \left\{ -\beta
% %             \left[ 
% %                 \sum_i \frac{p_i^2}{2m_i} + U(\rvec^N)
% %             \right]
% %         \right\}
% %     }
% % \end{align*}
% \begin{align*}
%     \left\langle A\right\rangle = \frac
%     {
%         \bigint \dif \bvec r^N \dif \bvec p^N \exp 
%         \left\{
%             -\beta \Ham \left( \bvec r^N, \bvec r^N \right)
%         \right\} 
%         A\left(\bvec r^N, \bvec p^N\right)
%     }
%     {
%         \bigint \dif \bvec r^N \dif \bvec p^N \exp 
%         \left\{
%             -\beta \Ham\left( \bvec r^N, \bvec r^N \right)
%         \right\}
%     },
% \end{align*}
% where $\beta = 1/k_B T$, $T$ is the temperature, $k_B$ is the Boltzmann constant, $\bvec r^N$ is the positions of all $N$ atoms, and $\bvec p^N$ the corresponding momenta. The function $\Ham(\bvec r^N, \bvec p^N)$ is the Hamiltonian of the system, and can be expressed as
% \begin{align*}
%     \Ham(\bvec r^N, \bvec p^N) = 
%         U(\rvec^N) + \sum_i^N \frac{p_i^2}{2m_i}
% \end{align*}
% 
% From this we can show that we can measure for example the average pressure and temperature of a system from the time evolution of those quantities
% \begin{align*}
%     \left\langle \frac{1}{2}mv^2 \right\rangle = \frac{1}{2}k_B T
% \end{align*}
% 
% \todo[inline]{The alternative is Monte Carlo simulations?}



In molecular dynamics we study systems of many interacting atoms and molecules by assuming that they behave classically, and solve Newton's equations of motion using an appropriate integration scheme to evolve the system in time. By assuming that the atoms behave classically we mean that we model the atoms as point particles, and characterize them using their position, $\bvec r$, velocity, $\bvec v$ and the force acting on them, $\bvec f$. The interactions between the atoms, that stem from the quantum mechanical \hl{something}, are described using potentials. It is into these potentials we bake the physical insight of the systems we want to simulate, which we often do by finding potentials using studies and simulations of the underlying quantum mechanical nature of the interactions between the atoms in the system.
