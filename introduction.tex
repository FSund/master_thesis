\chapter*{Introduction to the thesis}%
\addcontentsline{toc}{chapter}{Introduction to the thesis}%

\section*{Background and motivation}
\addcontentsline{toc}{section}{Background and motivation}%

The behaviour of water in nanoporous silica is important for many geological, biological, physical and biomedical processes. We are able to produce nanoporous silica with tuneable pore sizes, and the pore surfaces can be functionalized, which can lead to a wide range of industrial applications, many of which we have yet to discover, or the process behind is not fully understood. Examples of such applications are filtration, catalysis, phase separation, medical delivery systems, desalination of sea water, and CO$_2$ capture and storage.

Experiments have shown that the behaviour of fluids confined in nanoscale pores is different to that of bulk fluids. This is caused not only by the confinement to a small volume, but also by the high ratio of surface area to pore volume. There is evidence that water near surfaces can reach super-cooled states without forming ice, and that the density and dynamics of water near surfaces is different than that of bulk water. The detailed mechanisms for such behaviours are still uncertain.

While there has been performed significant number of high-quality experimental studies of the properties of water in nanoscale pores in the recent years, simulational studies are still limited, and many of the simulational studies that have been performed have been for small systems with limited dynamics. This makes this a good area for application of large-scale modelling and simulation.

We choose to model the nanoporous system at an atomic scale, both because we think that such a scale is appropriate for predicting the effects at nano-scales, and because similar studies have been performed with good results. We choose to use molecular dynamics to to the modelling, but an alternative method is direct simulation Monte Carlo, which have shown equally promising results applied on similar problems.

The goal of this master thesis is to do a numerical study of water trapped in nanoscale pores in solid silica, using a well-constrained interatomic potential that accurately predict the behaviour of both silica, water, and the interactions between the two. The potential we use has been developed by researchers at the University of Southern California, and the models have been well tested and calibrated to experimental and other numerical studies.

To study water trapped in nanoporous silica we need to develop reproducable methods for intializing realistic nanoporous silica, and methods for filling the nanoscale pores with water. We want to study the structure, behaviour and transport properties of water in silica, so we also need to find some appropriate way of quantifying and analyzing the dynamic and structural properties of water near the surface.

% Silica is hydrophilic -- expect changes near surface -- measure as function of distance to matrix
% 
% \todoa{Why study this? See master }
% % \todoa{Write about hydrophilic silica, distance to matrix effects, surface interaction}

 

% We know that silica is hydrophilic, so we expect the behaviour of water to change as we get closer to the silica matrix. We want to develop methods for accurately studying these surface effects.

% \todoa{MORE}

\section*{Structure of the thesis}
\addcontentsline{toc}{section}{Structure of the thesis}%
We start with a description of the numerical methods we will use to study silica and water. We then describe the methods we use for creating and initializing the systems we want to simulate and do measurements on. We then prepare, simulate, and do measurements on the systems. Finally we analyze the results, compare the measurements against each other, and draw some conclusions.
% \todoa{More?}
% The results are analyzed, and conclusions are drawn.\todoa{FIX THIS}

% We need to find an appropriate method for studying such a system
% 
% We want to study the effects of trapping water in nanoscale pores in solid silica. We want to look at the effects on the structure and transport properties of water.


% We want to study water in nanoporous silica
% Structure, transport properties
% How to make nanoporous?

% \begin{itemize}
%     \item Water confined in nanoporous silica
%     \item Characterization of porous silica
%     \item 
% \end{itemize}

% \begin{center}
% \begin{table}
%     \begin{tabular}{ | l | l | l | p{5cm} |}
%     \hline
%     Day & Min Temp & Max Temp & Summary \\ \hline
%     Monday & 11C & 22C & A clear day with lots of sunshine.  
%     However, the strong breeze will bring down the temperatures. \\ \hline
%     Tuesday & 9C & 19C & Cloudy with rain, across many northern regions. Clear spells
%     across most of Scotland and Northern Ireland,
%     but rain reaching the far northwest. \\ \hline
%     Wednesday & 10C & 21C & Rain will still linger for the morning.
%     Conditions will improve by early afternoon and continue
%     throughout the evening. \\
%     \hline
%     \end{tabular}
% \caption{A table for the list of tables, so it won't become envious of the other lists.}
% \end{table}
% \end{center}
