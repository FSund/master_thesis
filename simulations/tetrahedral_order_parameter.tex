\section{Tetrahedral order parameter}
% \todoa{Do we measure TOP for molecules or atoms?}%
% \todoa{Write about how we measure P(Q)?, since we measure probability/relative occurrence and not TOP as function of something}%
The tetrahedral order parameter\cite{errington2001relationship} is effectively a measure of how tetrahedral a set of four points are. The tetrahedral order parameter $Q$ for a point $k$ is calculated as follows%
\begin{align}
    Q_k = 1 - \frac{3}{8}\sum_i^3\sum_{j=i+1}^4 \left[ \cos \theta_{ikj} + \frac{1}{3} \right]^2,\label{eq:top_definition}
\end{align}%
%
% \begin{minipage}[c][4cm][c]{\textwidth}%
%     \begin{minipage}[c]{0.7\textwidth}%
%         \centering%
%         $Q_k = 1 - \frac{3}{8} \displaystyle\sum_i^3 \displaystyle\sum_{j=i+1}^4 \left[ \cos \theta_{ikj} + \frac{1}{3} \right]^2$%
%     \end{minipage}%
% %
%     \begin{minipage}[c]{0.25\textwidth}%
%         \centering
%         \includesvg[pretex=\normalsize, width=\textwidth, svgpath=./images/tetrahedral_order_parameter/]{tetrahedra02}%
%         \captionof{figure}{\label{hei}}
%     \end{minipage}%
%     
%     \begin{figure}[htpb]%
%         \includesvg[pretex=\normalsize, width=\textwidth, svgpath=./images/tetrahedral_order_parameter/]{tetrahedra02}%
%         \caption{a}%
%     \end{figure}%
%     
%     \caption{%
%         Illustration of the angles and molecules involved in the calculation of the tetrahedral order parameter. The \hl{blue} dots are molecules, in our case usually water molecules. We have the center molecule $k$, and \hl{its} four nearest neighbors. $\theta_{ikj}$ is the angle between molecule $i$, $k$ and $j$, as indicated by the \hl{orange} arc. \hl{FINISH CAPTION}. %
%         \label{fig:top_tetrahedra}%
%     }%
% \end{figure}%
% \end{minipage}%
% \captionof{figure}{alfa}
%
where $\theta_{ikj}$ is the angle %
%between two vectors from the main point, $k$, to point $i$ and point $j$, respectively, 
$i-k-j$ (with $k$ in the vertex of the angle) %
and the two sums go over the 6 possible angles $\theta$, between the main point and its four nearest neighbors. See \cref{fig:top_tetrahedra} for an illustration of the angles and points involved in the calculation. If we have $Q = 1$ the four points are arranged in a perfect tetrahedron, with $\theta_{ijk} = 2\arctan(2\sqrt(2)) \approx 109.47$ for all 6 possible angles between the four points, and as the points move away from this arrangement $Q$ decreases following \cref{eq:top_definition} (negative $Q$-values are possible).
%
\begin{figure}[htpb]%
    \centering%
    \includesvg[pretex=\normalsize, width=0.3\textwidth, svgpath=./images/tetrahedral_order_parameter/]{tetrahedra02}%
    \caption{%
        Illustration of the angles and points involved in the calculation of the tetrahedral order parameter. The blue dots are points, in our case usually water molecules. We have the center point $k$, and its four nearest neighbors. $\theta_{ikj}$ is the angle between point $i$, $k$ and $j$, as indicated by the orange line.%
        \label{fig:top_tetrahedra}%
    }%
\end{figure}%

We use the tetrahedral order parameter for investigating the coordination of water molecules relative to other water molecules. When we lower the temperature in water below freezing the average $Q$ will increase, since the water molecules in ice has very high coordination. Liquid water also has high coordination caused by the hydrogen bonds, but much lower coordination than ice, so the distribution of $Q$-values are more spread out, with the mean lower than the mean for ice.

When we measure the tetrahedral order parameter we measure $Q$ for all water molecules, defining the position of the water molecules as the position of the oxygen atom in each molecule, and plot the relative occurrence (or probability density), $P(Q)$, to investigate the distribution of $Q$-values. Since $Q$ is not dependent on several timesteps (like for example diffusion), averaging over several timesteps is trivial. Since silica is hydrophilic we expect $Q$ to be different for water molecules near the silica surface, so we will measure $P(Q)$ as function of distance to the silica matrix.

% \todob{Write about time and atom average?}