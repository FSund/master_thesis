% \FloatBarrier
\section{Manhattan distance to nearest atom\label{sec:generation_matrix}}
Since the program from \cref{sec:distance_to_atom} takes a long time to run to get decent maps with high resolution, we decided to also develop a similar program that creates a 3d map of the space, but this time creating a map with the Manhattan distance from each point on the grid to the nearest atom. To ease calculation we this time used a method inspired by the voxelation technique from \cref{sec:voxelation}. This program first divides the system into $n_x\times n_y\times n_z$ voxels, and make a 3d matrix of the same size for storing the Manhattan distance to the nearest atom in each point. We first give all voxels with one or more atoms in them the distance 0. We then label the rest of the voxels using an iterative method, increasing the number by one for each iteration. In each iteration we find the voxels that have a neighbor voxel \hl{labelled} with the previous label (\mono{label-1}), using 4-neighbor connectivity, and give them the current label. When all voxels are labelled, they should have a label corresponding to the \hl{Manhattan distance} to the nearest atom.

Although the Manhattan distance is not as useful as the regular Euclidean distance we calculate using the \hl{``distance to atom''}-program, the benefit is that making a 3d map of space using the \hl{``generation matrix''} method uses about 3\% of the time that ``distance to atom'' uses for the same system and same resolution. 

\todoa{Something about the usefulness of this measure?}

\orangebox{A system of 347176 atoms (\mono{rough\_fracture03}), water and SiO2, $256^3$ voxels, $~5$ seconds for ``generation matrix'', 2m27seconds for ``distance to atom'' (on one cpu)}
\todoc{Voxel counter code example?}