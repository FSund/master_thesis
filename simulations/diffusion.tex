\section{Diffusion}
\todob{derivation of diffusion? See \cite[Section~4.4.1]{frenkel2001understanding}}
When we talk about diffusion in \hl{the context of} this thesis we mean the process of \emph{self-diffusion}, which is different from \hl{``normal''} diffusion which is the net movement of a substance in the presence of a gradient, which can be for example a concentration gradient, a temperature gradient, or a pressure gradient. With \emph{self-diffusion} we mean the \hl{random?} movement in a substance that has no gradients.

Diffusion can be characterized by a constant $D$, which is related to the displacement of each atom relative to a initial position. We can measure this constant by measuring the mean square displacement $r_i^2(t)$ of each atom as a function of time, and average over all atoms. The mean square displacement is measured as
\begin{align*}
    \Braket{r^2(t)} = \frac{1}{N} \sum_{i=1}^N \left( \rvec_i(t) - \rvec_i(t=0) \right)^2,
\end{align*}
where $\rvec_i(t=0)$ is the initial position of atom $i$. From theoretical considerations of the diffusion process we can relate the diffusion constant to the mean square displacement through\cite[Section~4.4.1]{frenkel2001understanding}
\begin{align}
    \lim_{t\rightarrow \infty}\dpd{}{t}\Braket{r^2(t)} = 2dD\label{eq:diffusion_derivative}
\end{align}
where $d$ is the \hl{spatial} \hl{dimensionality/dimension?}. This means that we can find the diffusion constant in a molecular dynamics simulation by measuring the mean square displacement for many timesteps, and find the slope of this data as the diffusion constant in the limit $t\rightarrow \infty$. \hl{We are limited in that we ca not actually simulate infinite number of timesteps, but have to find a reasonable number of timesteps to measure over.} An example of how to sample the mean square displacement $\Braket{r^2(t)}$ in a simulation can be seen in \cref{list:diffusionSample}.
%
% This means that we can find the diffusion constant in a molecular dynamics simulation by measuring the mean square displacement for many timesteps, and plotting $\Braket{r^2(t)}/(2dt)$ as a function of time. The diffusion constant will then be the value this expression approaches when simulating for enough timesteps.
%
%An example of how to measure the mean square displacement in a molecular dynamics \hl{simulation/program} as outlined in \cref{chap:simple_md_program} can be seen in \cref{list:diffusionSample}.
% Fick's law relates flux $\bvec j$ of diffusing species to the concentration gradient $\bvec \nabla c$ of the species
% \begin{align*}
%     \bvec j = -D\bvec \nabla c,
% \end{align*}
% where $D$ is the proportionality constant called the diffusion \hl{coefficient/constant}.
% 
% Conservation of \hl{species/atoms?}
% \begin{align*}
%     \dpd{c(r,t)}{t} + \div \cdot \vec j(r,t) = 0,
% \end{align*}
% which gives
% \begin{align}
%     \dpd{c(r,t)}{t} - D\nabla^2 c(r,t) = 0.
%     \label{eq:diffusion_diff}
% \end{align}
% % We can solve this equation with the boundary condition
% % \begin{align*}
% %     c(r,0) = \delta (r),
% % \end{align*}
% % where $\delta(r)$ is the Dirac delta function, which gives
% % \begin{align*}
% %     c(r,t) = \frac{1}{(4\pi Dt)^{d/2}} \exp\left(-\frac{r^2}{4Dt}\right).
% % \end{align*}
% By multiplying \cref{eq:diffusion_diff} by $r^2$ and integrating over all space we get
% \begin{align*}
%     \dpd{}{t}\int \drvec~ r^2 c(r,t) = D\int \drvec~ r^2 \nabla^2 c(r,t).
% \end{align*}
% We recognize the integral on the left-hand side as the the mean square displacement %\hl{time dependence of the second moment of $c(r,t)$???}
% \todo{what???}
% \begin{align*}
%     \int \drvec~ c(r,t) r^2 \equiv \Braket{r^2(t)},
% \end{align*}
% where we have imposed
% \begin{align*}
%     \int \drvec~ c(r,t) = 1.
% \end{align*}
% Applying partial integration to the right-hand side we obtain
% \begin{align*}
%     \dpd{\Braket{r^2(t)}}{t} 
%     &= D\int \drvec~ r^2 \nabla^2 c(r,t) \\
%     &= D\int \drvec~ \div \cdot \left(r^2\nabla c(r,t)\right) - D\int\drvec~ \div r^2 \cdot \nabla c(r,t) \\
%     &= D\int \dif \vec S~ \left(r^2\div c(r,t)\right) - 2D\int\drvec~ \rvec \cdot\div c(r,t) \\
%     &= 0 - 2D\int\drvec~ (\div \cdot \rvec c(r,t)) + 2D\int\drvec~ (\div \cdot \rvec) c(r,t) \\
%     &= 0 + 2dD\int \drvec~ c(r,t) \\
%     &= 2dD
% \end{align*}
% 
% 
% \begin{align*}
%     \dpd{\Braket{r^2(t)}}{t} = 6D
% \end{align*}
% $\Rightarrow$ Plot mean square distance as function of time
% \begin{align*}
%     \Braket{\delta r(t)^2} = \frac{1}{N} \sum_{i=1}^N \delta \rvec_i(t)^2
% \end{align*}
% and find $6D$ as slope of plot \hl{(after a while?)}
%
\begin{listing}[!htb]%
\begin{cppcode*}{gobble=4}
    double diffusionSample(System &system) {
        double rSquared = 0.0;
        for (Atom *atom : system.atoms()) {
            drVec = atom->positiom() - atom->initialPosition() 
                    + atom->getBoundaryCrossings()*system.size();
            rSquared += drVec.lengthSquared();
        }
        rSquared /= system.nAtoms();
        return rSquared;
    }
\end{cppcode*}
\caption{%
    An example of how to calculate the mean square displacement in a molecular dynamics simulation. Example implementation of \mono{diffusionSample} from \cref{list:sampling}. We store the inital positions of the atoms as \mono{atom->initialPosition()}, and when using periodic boundary conditions we count the number of times we have to translate the atom one system-size in each direction, so while the position of the atom will always be inside the system, the \hl{\emph{real}} position of the atom can be calculated by adding \mono{atom->getBoundaryCrossings()*system.size()} to $\rvec$.%
    \label{list:diffusionSample}%
}%
\end{listing}%

% To improve our statistics when measuring the diffusion constant as function of distance to the surface of the pore we can use different time origins to get different samples. %
To measure $D$ we see from \cref{eq:diffusion_derivative} that we have to let $t\rightarrow \infty$, but in practice we usually see that the gradient of $\Braket{r^2(t)}$ usually have stabilzed near its final value after ${\sim} 5\text{ k}$ timesteps of 0.050 picoseconds. We can use this to get more samples for our measurements, by using different time origos. This technique involves using different initial positions for the atoms, from different timesteps in the simulation, and then finding $\Braket{r^2(t)}$ for $t_i\leq t\leq t_{i+n}$, where we $n$ is the number of timesteps we want to use for each time origo. We can in theory use overlapping intervals for $t$, but we chose to use adjacent, non-overlapping intervals. 

See \cref{fig:find_diff_const_example} for an example of how we find the diffusion constant as the gradient of $\Braket{r^2(t)}$, using different time origos.
%
% \todoao{Finish diffusion time origo stuff}
% \todoa{mention using average distance to matrix, procedure from \cref{sec:measuring_distance_to_matrix}}
% \todob{Plot of $r^2$ using standard settings ($dt$ = 20.67, 100 timesteps between states) to show approx how many origos we need to get stable gradient}
%
%
\begin{figure}[!htb]%
    \centering% 
    {
        \newcommand{\la}{\langle}%
        \newcommand{\ra}{\rangle}%
        \includesvg[width=0.8\textwidth, svgpath = ./images/diffusion/]{diffusion_constant_example02}%
    }
    \caption{%
        Illustration of how we find the diffusion constant, using two different time origos, with 40 timesteps of 0.050 picoseconds for each time origo. We find the diffusion constant as the gradient of $\langle r^2(t)\rangle/6$ for the 25 last timesteps for each origo. We then use the average of the gradients for each time origo as our approximation of $D$.%
        \label{fig:find_diff_const_example}%
    }%
\end{figure}%
% r_start = 0.0
% r_step = 0.25
% n_steps = 40
% folder_name_iterator_start = 0
% folder_name_step = 1
% folder_name_n_steps = 200
% 
% steps_per_origin = 40
%
%
% Diff const (gradient)
% 0.0295056927145
% 0.0495650040286
% 0.117231421271
% 0.0246869083899
% 0.0785722907501
% 0.156826102758
