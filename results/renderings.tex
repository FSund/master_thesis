\section{Visualizations}
We have made some renderings of the systems we have studied, as can be seen in \crefrange{fig:renderings_rough_fracture01_abel}{fig:renderings_flat_fractures02}\todoao{Check that this range is correct before handing in}. All renderings and visualizations were made using the program Ovito\cite{stukowski2010ovito}, using the built-in open-source ``Tachyon'' rendering engine.

In the renderings in this section we have colored the silicon atoms yellow, the oxygen atoms blue, and the hydrogen atoms white. The silicon atoms have been given a radius of 1 \AA, the oxygen atoms 0.6 \AA, and the hydrogen atoms 0.3 \AA.

% ---- rough_fracture01_abel ---- %
%
\begin{figure}[!p]%
%     \centering%
    \setlength{\myfigwidth}{0.55\textwidth}%
    \setlength{\myhfillwidth}{5mm}%
    %     \setlength{\mycaptionwidth}{0.3\textwidth}%
    %
    % Use makebox to center figures below that are wider than \textwidth
    % Use \centering inside subfigure to center image over caption
    %
    \makebox[\textwidth][c]{%
        \begin{subfigure}[t]{\myfigwidth}%
            \centering%
            \includegraphics[width=\textwidth]{images/systems/trimmed-rough_fracture01_abel_13}%
            \caption{The whole system.}%
        \end{subfigure}%
        \hspace{\myhfillwidth}%
        \begin{subfigure}[t]{\myfigwidth}%
            \centering%
            \includegraphics[width=\textwidth]{images/systems/trimmed-rough_fracture01_abel_15}%
            \caption{The whole system, with the size of the silicon and silica-oxygen atoms reduced to 0.1 \AA.}%
        \end{subfigure}%
    }%
        \vspace{10pt}\\%
    \makebox[\textwidth][c]{%
        \begin{subfigure}[t]{\myfigwidth}%
            \centering%
            \includegraphics[width=\textwidth]{images/systems/trimmed-rough_fracture01_abel_16}%
            \caption{20 \AA\ thick slice.}%
        \end{subfigure}%
        \hspace{\myhfillwidth}%
        \begin{subfigure}[t]{\myfigwidth}%
            \centering%
            \includegraphics[width=\textwidth]{images/systems/trimmed-rough_fracture01_abel_22}%
            \caption{The pore volume.}%
        \end{subfigure}%
    }%
    \vspace{10pt}\\%
    \caption{%
        ``Rough fracture \#1'', a randomly generated fracture with varying width. Generated from two random surfaces.%
        \label{fig:renderings_rough_fracture01_abel}%
    }%
\end{figure}%


% ---- rough_fracture03 ---- %
%
\begin{figure}[!p]%
    \centering%
    \setlength{\myfigwidth}{0.55\textwidth}%
    \setlength{\myhfillwidth}{5mm}%
    %     \setlength{\mycaptionwidth}{0.3\textwidth}%
    %
    % Use makebox to center figures below that are wider than \textwidth
    % Use \centering inside subfigure to center image over caption
    %
    \makebox[\textwidth][c]{%
        \begin{subfigure}[t]{\myfigwidth}%
            \centering%
            \includegraphics[width=\textwidth]{images/systems/trimmed-rough_fracture03_06}%
            \caption{The whole system.}%
        \end{subfigure}%
        \hspace{\myhfillwidth}%
        \begin{subfigure}[t]{\myfigwidth}%
            \centering%
            \includegraphics[width=\textwidth]{images/systems/trimmed-rough_fracture03_05}%
            \caption{The whole system, with the size of the silicon and silica-oxygen atoms reduced to 0.1 \AA.}%
        \end{subfigure}%
    }%
    \vspace{10pt}\\%
    \makebox[\textwidth][c]{%
        \begin{subfigure}[t]{\myfigwidth}%
            \centering%
            \includegraphics[width=\textwidth]{images/systems/trimmed-rough_fracture03_07}%
            \caption{20 \AA\ thick slice.}%
        \end{subfigure}%
        \hspace{\myhfillwidth}%
        \begin{subfigure}[t]{\myfigwidth}%
            \centering%
            \includegraphics[width=\textwidth]{images/systems/trimmed-rough_fracture03_11}%
            \caption{The pore volume.}%
        \end{subfigure}%
    }%
    \vspace{10pt}\\%
    \caption{%
        ``Rough fracture \#2'', a randomly generated fracture with varying width. Generated from two random surfaces.%
        \label{fig:renderings_rough_fracture03}%
    }%
\end{figure}%


% ---- rough_fracture04_same_distance ---- %
%
\begin{figure}[!p]%
    \centering%
    \setlength{\myfigwidth}{0.55\textwidth}%
    \setlength{\myhfillwidth}{5mm}%
    %     \setlength{\mycaptionwidth}{0.3\textwidth}%
    %
    % Use makebox to center figures below that are wider than \textwidth
    % Use \centering inside subfigure to center image over caption
    %
    \makebox[\textwidth][c]{%
        \begin{subfigure}[t]{\myfigwidth}%
            \centering%
            \includegraphics[width=\textwidth]{images/systems/trimmed-rough_fracture04_06}%
            \caption{The whole system.}%
        \end{subfigure}%
        \hspace{\myhfillwidth}%
        \begin{subfigure}[t]{\myfigwidth}%
            \centering%
            \includegraphics[width=\textwidth]{images/systems/trimmed-rough_fracture04_07}%
            \caption{The whole system, with the size of the silicon and silica-oxygen atoms reduced to 0.1 \AA.}%
        \end{subfigure}%
    }%
    \vspace{10pt}\\%
    \makebox[\textwidth][c]{%
        \begin{subfigure}[t]{\myfigwidth}%
            \centering%
            \includegraphics[width=\textwidth]{images/systems/trimmed-rough_fracture04_05_20ang}%
            \caption{20 \AA\ thick slice.}%
        \end{subfigure}%
        \hspace{\myhfillwidth}%
        \begin{subfigure}[t]{\myfigwidth}%
            \centering%
            \includegraphics[width=\textwidth]{images/systems/trimmed-rough_fracture04_11}%
            \caption{The pore volume.}%
        \end{subfigure}%
    }%
    \vspace{10pt}\\%
    \caption{%
        ``Rough fracture \#3'', a randomly generated fracture generated from one surface repeated for the top and bottom half, with 14.4 \AA\ between the surfaces, giving approximately uniform width of the pore.%
        \label{fig:renderings_rough_fracture04_same_distance}%
    }%
\end{figure}%


% ---- rough_fracture05 ---- %
\begin{figure}[!p]%
    \centering%
    \setlength{\myfigwidth}{0.55\textwidth}%
    \setlength{\myhfillwidth}{5mm}%
    %     \setlength{\mycaptionwidth}{0.3\textwidth}%
    %
    % Use makebox to center figures below that are wider than \textwidth
    % Use \centering inside subfigure to center image over caption
    %
    \makebox[\textwidth][c]{%
        \begin{subfigure}[t]{\myfigwidth}%
            \centering%
            \includegraphics[width=\textwidth]{images/systems/trimmed-rough_fracture05_05}%
            \caption{The whole system.}%
        \end{subfigure}%
        \hspace{\myhfillwidth}%
        \begin{subfigure}[t]{\myfigwidth}%
            \centering%
            \includegraphics[width=\textwidth]{images/systems/trimmed-rough_fracture05_04}%
            \caption{The whole system, with the size of the silicon and silica-oxygen atoms reduced to 0.1 \AA.}%
        \end{subfigure}%
    }%
    \vspace{10pt}\\%
    \makebox[\textwidth][c]{%
        \begin{subfigure}[t]{\myfigwidth}%
            \centering%
            \includegraphics[width=\textwidth]{images/systems/trimmed-rough_fracture05_02_20ang}%
            \caption{20 \AA\ thick slice.}%
        \end{subfigure}%
        \hspace{\myhfillwidth}%
        \begin{subfigure}[t]{\myfigwidth}%
            \centering%
            \includegraphics[width=\textwidth]{images/systems/trimmed-rough_fracture05_09}%
            \caption{The pore volume.}%
        \end{subfigure}%
    }%
    \vspace{10pt}\\%
    \caption{%
        ``Rough fracture \#4'', a randomly generated fracture generated from one surface repeated for the top and bottom half, with 28.8 \AA\ between the surfaces, giving approximately uniform width of the pore. \hl{Caption} %
        \label{fig:renderings_rough_fracture05}%
    }%
\end{figure}%


% ---- Reference systems ---- %
%
\begin{figure}[!p]%
%     \centering%
    \setlength{\myfigwidth}{0.55\textwidth}%
    \setlength{\myhfillwidth}{5mm}%
    %     \setlength{\mycaptionwidth}{0.3\textwidth}%
    %
    % Use makebox to center figures below that are wider than \textwidth
    % Use \centering inside subfigure to center image over caption
    %
    \makebox[\textwidth][c]{%
        \begin{subfigure}[t]{\myfigwidth}%
            \centering%
            \includegraphics[width=\textwidth]{images/systems/trimmed-flat_square_fracture02_03}%
            \caption{``Reference \#1''.}%
            \label{fig:renderings_flat_square_fracture02}%
        \end{subfigure}%
        \hspace{\myhfillwidth}%
        \begin{subfigure}[t]{\myfigwidth}%
            \centering%
            \includegraphics[width=\textwidth]{images/systems/trimmed-flat_square_fracture03_04}%
            \caption{``Reference \#2''.}%
            \label{fig:renderings_flat_square_fracture03}%
        \end{subfigure}%
    }%
    \vspace{10pt}\\%
    \caption{%
        Reference systems \#1 and \#2, 86 \AA\ wide flat pores.%
        \label{fig:renderings_flat_fractures01}%
    }%
\end{figure}%


% ---- Reference systems ---- %
%
\begin{figure}[!p]%
%     \centering%
    \setlength{\myfigwidth}{0.55\textwidth}%
    \setlength{\myhfillwidth}{5mm}%
    %     \setlength{\mycaptionwidth}{0.3\textwidth}%
    %
    % Use makebox to center figures below that are wider than \textwidth
    % Use \centering inside subfigure to center image over caption
    %
    \makebox[\textwidth][c]{%
        \begin{subfigure}[t]{\myfigwidth}%
            \centering%
            \includegraphics[width=\textwidth]{images/systems/trimmed-flat_fracture02_03}%
            \caption{``Reference \#3''.}%
            \label{fig:renderings_flat_fracture02}%
        \end{subfigure}%
        \hspace{\myhfillwidth}%
        \begin{subfigure}[t]{\myfigwidth}%
            \centering%
            \includegraphics[width=\textwidth]{images/systems/trimmed-flat_fracture03_03}%
            \caption{``Reference \#4''.}%
            \label{fig:renderings_flat_fracture03}%
        \end{subfigure}%
    }%
    \vspace{10pt}\\%
    \caption{%
        Reference systems \#3 and \#4, respective a 14.4 and a 28.8 \AA\ wide flat pore.
        \label{fig:renderings_flat_fractures02}%
    }%
\end{figure}%