\section{Conclusions}
\todoa{Check that everything in this section is mentioned in results?}
To investigate the structure of water trapped in pores in nanoporous silica we have measured the density and tetrahedral order parameter of water, as function of the distance to the silica matrix. We found that the structure and density of water was drastically changed as we got within 6-8 \AA\ of the silica matrix.

When measuring the water density we saw that it seemed to decrease by around 10\% of the bulk density when we got to around 5 \AA\ from the silica matrix, before increasing to at higher than the the bulk density at 3 \AA\ from the matrix. The minimum point around 5 \AA\ seemed roughly constant for all pore geometries and bulk densities, but the falloff for the density, around 7 \AA\ seemed to move closer to the silica matrix as we increased the bulk density.

When we measured the tetrahedral order parameter for water molecules, a number between 0 and 1 that tells us something about how ``tetrahedral'' a set of four points is, we saw that the water molecules had the expected structure in bulk conditions (more than 10 \AA\ from a silica matrix), but that as we got to 5.5 \AA\ and closer to the silica matrix this parameter was disturbed by the silica matrix, and we saw large changes in the intermolecular structure.

This change in the structure and density of the water molecules can be caused by the hydrophilic nature of silica, and the interactions between the silica matrix and the water molecules. What we think happen is that the water molecules close to the silica matrix are attracted to the matrix, and they arrange themselves in a structure similar to the silica structure. Since the water molecules has strong hydrogen bonds between them, this change in structure will appear for several layers of water molecules into the water, beyond the reach of the water-silica interaction.

To study the transport properties of water in nanoporous silica we measured the self-diffusion constant of the water, both in bulk-like water far from the silica surface, and as function of the distance to the silica surface. We found that the diffusion was approximately equal to the bulk at up to around 7 \AA\ from the silica matrix, where it started decreasing. The diffusion constant decreased approximately linearly from 5 to 3 \AA\ from the silica matrix, at which point all diffusion stopped. This behaviour can again be caused by the surface interaction between water and silica, where the silica attract the water molecules, and hinders the movement and diffusion. This attraction causes the water molecules to arrange in a certain way near the silica surface, which again affects the next layers of water molecules via the strong hydrogen bonds between water molecules, and we see a reduction in diffusion several \Ang from the silica matrix.

When measuring diffusion we also noticed that the self diffusion decreased with increasing water density. This is as expected, since the increased density will make it harder for water molecules to diffuse.

Perhaps the most interesting thing we saw during our simlulations was that the geometries of the fractures and pores had little to no effect on the water, as water in both completely flat pores and in very rough and random fractures seemed to exhibit almost the same characteristics, both in transport properties and structure. The system that deviated the most from this was the system with a narrow 14.4 \AA\ rough fracture, which had about the same water density as the other rough fracture systems, but showed a somewhat reduced diffusion at all distances from the matrix, compared to the other rough fractures. This system also showed some differences in the tetrahedral order parameter.


% When measuring the density water in the different systems as function of distance to the silica matrix we have seen that the density is approximately constant up to 7-8 \AA\ from the silica matrix. At this point there is a small peak in the density, before it decreases by close to 10\%, a minimum which is usually found near 5 \AA\ from the silica matrix. From here the density rises as we reduce the distance to the matrix to 3 \AA, usually rising to higher densities than the inital density at 8 \AA. This behaviour is seen in all systems, independent of the pore geometry and overall water density in the system.
% 
% One trend that appeared when we increased the overall density, in otherwise identical systems, was that distance where the peak of the density appeared seemed to decrease. The distance at where the minimum was reached also seemed to appear closer to the matrix when we increased the overall density, although this trend was much less apparent.


% - small differences between flat and rough, and between the two rough types, but clear trends in water transport properties and structure (diff, TOP) near surface
% - density 8-10 ang, diffusion and TOP 5-6 ang

% important with density control in future experiments - new methods (use tetrahedra?)

% \todoa{Write about normalized densities, why we normalize, vacuum problems..}

% diffusion:
% 28.8 narrow fracture very similar to 86 reference with normal density -- we should study narrower systems to see anything interesting
% random rough fractures very similar to 86 reference with normal density -- can try more systems like that, but using narrower pore geometry?

\section{Future}
When doing measurements on the structure and transport properties of water we were surprised to see that the simulations we did showed little differences, even though the structure and characteristics of the different systems we simulated were very different. We saw some interesting differences when simulating systems with pores as narrow as 14.4 \AA, and a more extensive and thorough study of pores of this size, and smaller, could turn out to be fruitful. 

One big problem if one wants to study narrow pores is, as we discovered, controlling the density of water in the pores. The method we developed and used for filling pores and fractures with water works good in large fractures and far from the silica surface, but in narrow fractures with a lot of silica surface the voxelation technique breaks down, and we are unable to insert the number of water molecules needed to get the density we want. To do further studies of water in nanoporous silica one should thus try to improve this method for filling pores with water, or develop a completely new method for filling narrow pores with water.

One of the hurdles in studying any surface, or a fractured or porous system, is defining \emph{where} the surface actually \emph{is}. We \hl{circumvented} this problem by defining the distance to the surface as the distance to the nearest silicon atom, since we only needed the distance to the surface. This works well in a lot of cases, but partly breaks down when we go to distances similar to or smaller than the average distances between the atoms the surface consist of (closer than about 3 \AA\ in our case). A different solution is to define the surface as a set of mathematical planes, and the most practical thing to do here is probably to define the surface as sets of \emph{triangles}. Triangulation of surfaces and volumes is a method studied a lot both by mathematicians and by computer graphics developers, so there are a lot of efficient methods for doing calculations on these kinds of data. \hl{Although a method based on triangles seem good at first glance, one must ask if it is even possible to define the surface of something as a plane at an atomic scale. Because at those scales a surface actually consist of atoms and molecules, and not simple planes.}

An improvement that could improve on our results would be to develop a method that generates fractures with actual uniform width across the whole system, since the ``uniform width'' fractures we simulated only really had uniform distance between the two surfaces in the $z$-direction, and thus the actual width of the pore varied throughout the system, although it was somewhat limited by the distance between the two surfaces.

Lastly we would have liked to measure the diffusion normal to and parallell to the silica surfaces in our systems. But this would require us to locate and define the surface, which we have seen is a complicated problem.

\todoa{Normal and parallell diffusion! effect of binning on diffusion}

% Measure in narrower fracture -- hard to initialize -- new inject methods
% Better method for creating uniform width fractures? Since our method only creates equal $\Delta z$