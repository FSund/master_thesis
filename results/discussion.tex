\section{Discussion}
After doing a thorough study of the measurements done in the simulations, we will now try to draw some conclusions from the different results we have found. We will first discuss the individual results from each physical quantity we have measured, before we try to put these results into perspective.
% \hl{We will now compare the different results to each other, to see if we can draw any conclusions from them.}

% \todoa{Surface interactions}

The results of the measurements of the water density in the different simulated systems shows that some of our systems has much higher density than the other systems, especially refence system \#2-4. When filling the pores in these systems with water we used a higher input density, so this was according to our intention, and shows that the method we use for filling pores with water works as intended for those systems. But we also used much higher input densities in rough fracture system \#3 and \#4 than \#1 and \#2, and we did not measure higher water density in those systems. This can be explained by looking at the geometry of the fractures in system \#3 and \#4, which are very narrow fractures, respectively 14.4 and 28.8 \AA\ wide. The method we use for filling the pores with water first divides the system into cuboid voxels (with the size of the voxels calculated from the wanted density), marks all voxels with an atom in them as occupied, and then puts one water atom in each unoccupied voxel. This voxelation approach has trouble when we use it on rough surfaces, as it will mark a lot of the voxels near the surface as occupied, even though we in reality could fit a water molecule in the voxel. In a narrow pore, with a lot of surface area compared to the pore volume, a lot of the voxels near the surface will be marked as occupied, and the result is that we get a lower water density then we wanted.
% When we fill the pores with water we divide the system into voxels, and mark all voxels with atoms in them as occupied, an unavailable for putting water molecules in. What happens when we do this in very narrow fractures is that a lot of the voxels near the surface of the silica matrix will be unavailable for water molecules, and since most of the volume in the narrow fractures is close to the silica matrix, a lot of the volume end up without water molecules in it. The result of this is that we get a much lower water density than expected.\todobo{Improve this paragraph about voxelation/water filling}

When we look at the water density as function of distance to the silica matrix we see the same behaviour in all systems, independently of the structure of the pores and fractures. The density peaks at around 7-8 \AA\ from the silica matrix, and drop by almost 10\% at 5 \AA. From this it is obvious that the the hydrophilic attraction of the silica matrix, and the interactions between the water molecules and the silica matrix has some effect on the water molecules.

We also saw a trend in this peak in density, which appeared around 7-8 \AA\ from the silica matrix, but which seemed to move closer to the silica matrix as we increased the bulk density. From this it seems like the increased water pressure makes the water molecules retain their bulk-like properties closer to the surface.
% The cause of this may be because the increased water pressure makes the water retain is b
% We suspect that the cause of this is simply the increased water pressure compressing the water structure near the silica surface, pusing the peak closer to the matrix.

% When comparing our estimates and measurements of the water density with the wanted density we used when filling the pores with water, we saw that the method we used for filling the pores and fractures with water struggled with filling the narrowest and tightest pores. This is caused by the voxelation method we use to find room for water molecules, which uses rectangular voxels that are marked as occupied or unoccupied based on the neighboring atoms. The use of rectangular voxels means that we often get voxels near the surface that marked as occupied, and in a narrow pore with the width of the pore comparable to the size of the voxels, we risk ending up with all voxels marked as occupied, even though there in reality is room for a water molecule between the walls of the pore.\todoao{explain this better?}

In the measurements of the diffusion we saw clear changes in the transport properties as we got close to the silica matrix. We saw the same behaviour in all simulations, independent of the pore geometry. The diffusion constant was stable up to around 7 \AA\ from the silica matrix, where it made a dip near 5.5 \AA, before reaching a local peak at 5 \AA, and then going linearly to zero from 5 to 3 \AA. We again see clear results of the interactions between the water molecules and the silica matrix. We know that silica is hydrophilic, so when the diffusion goes to zero the cause can be that the water molecules are attracted to the silica surface, which slows down the self diffusion. We will also have an effect of water molecules colliding with the silicon, oxygen and hydrogen atoms in the silica structure, which further slows down diffusion. 

When comparing reference systems \#1 and \#2 which have the exact same pore geometry and dimensions, we see a clearly reduced diffusion in one of the systems. As the only difference between those two systems is the density, we see that this increase in density lowers self diffusion. This is only natural, since the water molecules will collide more often with other water atoms, which slows down diffusion.

Although the diffusion all systems had very similar behaviour, rough fracture system \#3, which is a 14.4 \AA\ narrow rough fracture, displayed a somewhat reduced diffusion constant at all distances from the matrix, compared to the other rough fracture systems. We compare this to to rough fracture system \#4, which has a similar fracture, but twice as wide (28.8 \AA). This system shows no reduction in the diffusion, but almost perfectly matches the diffusion in a 86 \AA\ wide flat pore (reference system \#1). We also compare it to reference system \#3, which has a 14.4 \AA\ flat pore. This system has very similar diffusion to rough fracture system \#3, although this system also has a somewhat higher water density, which may cause reduced diffusion. We conclude with that the reduction in diffusion in rough fracture system \#3 is caused by the width of the fracture itself, since the water density in this system is similar to the density in the other rough fracture systems. \todoao{improve this paragrap?}

% We saw a clearly lowered diffusion in a system that, except for an increased density, was equivalent to another system

When measuring the tetrahedral order parameter we saw similar behaviour in all systems, independent of pore geometry and structure. We observed clear changes in the structure of the water molecules as we go close to the silica matrix. At 5.5 \AA\ and further away from the matrix the water had close to bulk properties, but as we moved closer than this one of the observed peaks in the distribution was reduced, by almost a factor 2 by the time we get to 3.5 \AA. This is a clear indication that the silica matrix is affecting the internal structure in the water, and the arrangement of the water molecules. This can again be caused by the hydrophilic silica, which can make the water molecules arrange themselves in a structure similar to the nearby silica structure. Since water has strong hydrogen bonds between water molecules, this effect and arrangement can spread in through the water, several layers from the silica matrix, beyond the reach of the force from the silica itself.

Although the tetrahedral order parameter was very similar for all systems independent of pore geometry and other characteristics, we see that the reference systems are more similar to each other than to the rough fracture systems, and that the rough fracture systems are more similar to each other than to the reference systems. The only real deviation we see is in rough fracture system \#3, which is a 14.4 \AA\ narrow fracture. For this system we see that one of the peaks in the distribution of tetrahedral order paramters seems to rise a bit faster when we increase the distance to the silica matrix, compared to the other rough fracture systems.

When we compare the results from the diffusion and the tetrahedral order parameter we see that the silica surface seems to affect the water molecules a lot. The surface effects and interactions between silica and water molecules changes the internal ordering of water molecules in the water, which again slows down diffusion, and reduces the water density, as this new structure have a lower density than the structure of water in bulk conditions. We see that all systems behave very similarly, with the only real deviation being rough fracture system \#3, which show show somewhat different transport properties and structural properties.

\todoa{Check that everything in this section is mentioned in results?}

% In total we see that the structure and geometry of the pores have little to no effect on the structure and water

% When we measured the density of water in our simulations we noticed some trends in the behaviour of the density as function of the distance to the silica matrix, that seemed to be independent of the characteristics of the pores and fractures in the systems, and independent of the overall density in the systems

% diff: higher density lowers diff

% TOP: silica structure in water?