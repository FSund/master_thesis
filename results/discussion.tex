\chapter{Discussion}
The results of the measurements of the water density showed that some of our systems had much higher density than the other systems, as was our intention as we used a higher input density when we filled the fractures and pores with water in those systems. We see the same behaviour in all systems, where the density seems to peak at around 7-8 \AA\ from the silica matrix, and drop by almost 10\% at 5 \AA.

When comparing our estimates and measurements of the water density with the wanted density we used when filling the pores with water, we saw that the method we used for filling the pores and fractures with water struggled with filling the narrowest and tightest pores. This is caused by the voxelation method we use to find room for water molecules, which uses rectangular voxels that are marked as occupied or unoccupied based on the neighboring atoms. The use of rectangular voxels means that we often get voxels near the surface that marked as occupied, and in a narrow pore with the width of the pore comparable to the size of the voxels, we risk ending up with all voxels marked as occupied, even though there in reality is room for a water molecule between the walls of the pore.\todoao{explain this better?}

We noticed a trend in the peak of the density, which seemed to move closer to the silica matrix as we increased the bulk density. We suspect that the cause of this is simply the increased water pressure compressing the water structure near the silica surface, pusing the peak closer to the matrix.

In the measurements of the diffusion we saw clear changes in the transport properties as we got close to the silica matrix, with the diffusion constant going to zero at around 3 \AA from the silica matrix. We know that silica is hydrophilic, so the cause of this can be that the water molecules are attracted to the silica surface, which slows down the self diffusion. We will also have an effect of water molecules colliding with the silicon, oxygen and hydrogen atoms in the silica structure, which further slows down diffusion.

When measuring the tetrahedral order parameter


When we measured the density of water in our simulations we noticed some trends in the behaviour of the density as function of the distance to the silica matrix, that seemed to be independent of the characteristics of the pores and fractures in the systems, and independent of the overall density in the systems

diff: higher density lowers diff

TOP: silica structure in water?