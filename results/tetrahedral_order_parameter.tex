\section{Tetrahedral order parameter}
\todoa{Bin width 0.5 \AA\ and $\Delta Q = 0.01$ for all results! ($\Delta Q$ isn't shown anywhere)}
% \todoa{Make two separate TOP figures, one for regular rough, and one for rough with same surfac x2 ?}
% \todoa{Update bulk TOP figure with results from long analysis run}
% %
% \begin{figure}[htpb]%
%     \centering%
%     \includesvg[width=1.0\textwidth, svgpath = ./images/tetrahedral_order_parameter/]{fancyfig04}%
%     \caption{}%
% %     \label{fig:distance_to_atom_r20}%
% \end{figure}%
%
% \begin{figure}[htpb]%
%     \centering%
%     \includesvg[%
% %         width=\textwidth,% choose which fits best, width or height (regular includegraphics can use both and add "keepaspectratio", but includesvg can't)
%         height=\textheight,%
%         svgpath = ./images/tetrahedral_order_parameter_new/]{figure01}%
%     \caption{\hl{Caption, add legends!}}%
% %     \label{fig:distance_to_atom_r20}%
% \end{figure}%
%

We have measured the thetrahedral order parameter for our four reference systems, and the four random fracture systems, and plotted the \hl{relative occurrence/probability} $P(Q)$ as function of $Q$, for water molecules with different distances from the silica matrix. We used used 150 bins for the $Q$-values, with $\Delta Q = 0.01$, and bins of $\Delta r = 0.5$ for the different distances to the matrix, for all measurements presented here.

For comparison we have also estimated $P(Q)$ in bulk water, by measuring the tetrahedral order parameter for all water molecules further from the silica matrix than 10 \AA, in the two reference systems with 86 \AA\ wide flat pores. The results for bulk water can be seen in \cref{fig:top_bulk}. We see that we get \hl{two peaks (``bimodal''?)}, one just above $Q = 0.5$ and one right below $Q = 0.75$. \todoao{Compare with some reference, for example Mathilde's ref [26]?}
%
\begin{figure}[htpb]%
    \centering% 
    \includesvg[%
        width=0.5\textwidth,
        svgpath = ./images/tetrahedral_order_parameter_new/%
    ]{figure01_bulk}%
    \caption{%
        Plot of tetrahedral order parameter for bulk water (water molecules further than 10 \AA\ from the silica matrix), in the two reference systems with 86 \AA\ wide flat pores (reference system \#1 and \#2, see \cref{fig:renderings_flat_square_fracture02,fig:renderings_flat_square_fracture03}). The average $Q$ is indicated by a vertical line. %
    }%
    \label{fig:top_bulk}%
\end{figure}%

In \cref{fig:top_normal_and_reference,fig:top_narrow01_and_reference,fig:top_narrow02_and_reference} we have plotted the results for different combinations of random fractures and flat pores. We have plotted this for a total of %
% 10 %
8 %
different distances from the silica matrix, for $Q$-values ranging from -0.5 to 1.0. 

We see that all systems have pretty similar behaviour for different $Q$-values, but we also see that $P(Q)$ changes a lot when we go from 3.0 \AA\ from the silica matrix to 5 \AA. At 5.5 \AA we are already pretty close to the bulk behaviour seen in \cref{fig:top_bulk}, with a small gradual change from 5.5 \AA\ to 6.5 \AA\ (and further to 10 \AA, not plotted here), where we are close to bulk behaviour. In general we see that near the silica matrix water seems to have a very different internal coordination compared to in bulk water, with the average $Q$-value going from $Q_\text{mean} \approx 0.2$ at $r = 3.0$ \AA\ to $Q_\text{mean} \approx 0.55$ at $r = 4.5$ \AA, in all systems, which is close to the bulk value of $Q_\text{mean, bulk} \approx 0.63$. We also see that the distribution has a different shape at $r = 4.5$ \AA\ than the bulk distribution. We also see that $P(Q)$ is much more spread out close to the silica matrix compared to further away, and that we only have a single peak, while bulk water has \hl{two peaks (``bimodal''?)} and is less spread out.\todobo{come up with something more accurate than ``spread out''}
\todob{Table of mean Q? Stddv?}

%
\begin{figure}[!p]%
    \centering%
    \includesvg[%
        width=\textwidth,% choose which fits best, width or height (regular includegraphics can use both and add "keepaspectratio", but includesvg can't)
%         height=\textheight,%
        svgpath = ./images/tetrahedral_order_parameter_new/%
    ]{figure02_regular_and_wide}%
    {%
        \captionsetup{width=\textwidth}%
        \caption{%
            Plot of $P(Q)$ for the two regular rough fractures (the solid lines), and a reference system with a 86 \AA\ wide flat pore (the dashed line) (see \cref{fig:renderings_rough_fracture01_abel,fig:renderings_rough_fracture03,fig:renderings_flat_square_fracture03}), for different distances to the silica matrix (as indicated by the text in top left corner of each plot). The average $Q$ is indicated by a vertical line. %
            \label{fig:top_normal_and_reference}%
        }%
    }%
\end{figure}%

In figure \cref{fig:top_normal_and_reference} we have plots of $P(Q)$ for rough systems \#1 and \#2, which are the two random rough fractures, and reference system \#1, which is a 86 \AA\ wide flat pore with a bulk water density of $\sim 1038$ kg/m$^3$. We see that the two random fracture systems have very similar behaviour, as expected, since they were prepared very similarly. We see that the average $Q$-value of the fracture systems is generally lower than in the reference system, and that the fracture systems seem to ``lag behind'' the reference system when we increase the distance to the silica matrix, with the fractured systems catching up the reference system near 6.5 \AA, as all three plots get close to a bulk-like shape.

%
\begin{figure}[!p]%
    \centering% 
    \includesvg[%
        width=\textwidth,% choose which fits best, width or height (regular includegraphics can use both and add "keepaspectratio", but includesvg can't)
%         height=\textheight,%
        svgpath = ./images/tetrahedral_order_parameter_new/%
    ]{figure02_flat_and_wide01}%
    {%
        \captionsetup{width=\textwidth}%
        \caption{%
            Plot of $P(Q)$ for a rough fracture with uniform width of 14.4 \AA\ (the blue solid line), a reference system with a 14.4 \AA\ wide flat pore (the dashed line), and a flat reference system with a 86 \AA\ wide flat pore (the red solid line) (see \cref{fig:renderings_rough_fracture04_same_distance,fig:renderings_flat_fracture03,fig:renderings_flat_square_fracture03}), for different distances to the silica matrix (as indicated by the text in top left corner of each plot). The average $Q$ is indicated by a vertical line. %
            \label{fig:top_narrow01_and_reference}%
        }%
    }%
\end{figure}%

In \cref{fig:top_narrow01_and_reference} we have plots of $P(Q)$ for rough system \#3, which is the fracture with uniform width of 14.4 \AA, and reference systems \#1 and \#3, which are the 86 and 14.4 \AA\ wide flat pores. %
We see that that the reference system with a wide flat pore seems to lie between the two systems with narrow pores for $r\in[3.5,5.5]$, and that the narrow fracture seems to lag behind the two reference systems as we increase $r$ from 3.5 to 6.0 \AA, with the behaviour in all systems being close to bulk at $6.0$ \AA. We see that the average $Q$ for the narrow fracture seems to lie lower than the wide flat pore reference system, and the narrow flat pore reference system seems to lie higher.
% We see the same trend in $P(Q)$ as before, with the average $Q$ being lower for the fracture and the narrow flat pore for $r < 5.5$ \AA, and the shape of $P(Q)$ getting close to bulk as $r\rightarrow 6.0$ \AA. We see that $P(Q)$ for the narrow flat pore seems to lie between the narrow fracture and the reference system, and that both the narrow fracture and narrow pore seems to lag behind the reference system as we increase the distance to the silic matrix.

%
\begin{figure}[!p]%
    \centering% 
    \includesvg[%
        width=\textwidth,% choose which fits best, width or height (regular includegraphics can use both and add "keepaspectratio", but includesvg can't)
%         height=\textheight,%
        svgpath = ./images/tetrahedral_order_parameter_new/%
    ]{figure02_flat_and_wide02}%
    {%
        \captionsetup{width=\textwidth}%
        \caption{%
            Plot of $P(Q)$ for a rough fracture with uniform width of 28.8 \AA\ (the blue solid line), a reference system with a 28.8 \AA\ wide flat pore (the dashed line), and a reference system with a 86 \AA\ wide flat pore (the red solid line) (see \cref{fig:renderings_rough_fracture05,fig:renderings_flat_fracture03,fig:renderings_flat_square_fracture03}), for different distances to the silica matrix (as indicated by the text in top left corner of each plot). The average $Q$ is indicated by a vertical line. %
            \label{fig:top_narrow02_and_reference}%
        }%
    }%
\end{figure}%

In \cref{fig:top_narrow01_and_reference} we have plots of $P(Q)$ for rough system \#4, which is the fracture with uniform width of 28.8 \AA, and reference systems \#1 and \#4, which are the 86 and 28.8 \AA\ wide flat pores. We see the same behaviour as for the 14.4 \AA\ fracture, albeit with the 28.8 \AA\ reference system being a bit closer to the reference system with a wide fracture.

In both \cref{fig:top_narrow01_and_reference,fig:top_narrow02_and_reference} it seems like the systems with a narrow flat pore (the red solid lines) seems to deviate the most from the wide reference system.

%
\begin{figure}[htpb]%
    \centering% 
    \includesvg[%
        width=\textwidth,% choose which fits best, width or height (regular includegraphics can use both and add "keepaspectratio", but includesvg can't)
%         height=\textheight,%
        svgpath = ./images/tetrahedral_order_parameter_new/%
    ]{figure02_all_rough}%
    {%
%         \captionsetup{width=\textwidth}%
        \caption{%
            \hl{Caption} %
            \label{fig:top_all_rough_and_one_reference}%
        }%
    }%
\end{figure}%

In \cref{fig:top_all_rough_and_one_reference} we have plotted $P(Q)$ for all rough fracture systems and reference \#1, which is the 86 \AA\ wide flat fracture. We see that the average $Q$ for all fracture systems seems to lie below the reference system, and they all seem to lag beind the reference system as we increase $r$ from 3.5 to 5.5 \AA.