\section{Tetrahedral order parameter}
% \todoa{Make two separate TOP figures, one for regular rough, and one for rough with same surfac x2 ?}
\todoa{Update bulk TOP figure with results from long analysis run}
% %
% \begin{figure}[htpb]%
%     \centering%
%     \includesvg[width=1.0\textwidth, svgpath = ./images/tetrahedral_order_parameter/]{fancyfig04}%
%     \caption{}%
% %     \label{fig:distance_to_atom_r20}%
% \end{figure}%
%
% \begin{figure}[htpb]%
%     \centering%
%     \includesvg[%
% %         width=\textwidth,% choose which fits best, width or height (regular includegraphics can use both and add "keepaspectratio", but includesvg can't)
%         height=\textheight,%
%         svgpath = ./images/tetrahedral_order_parameter_new/]{figure01}%
%     \caption{\hl{Caption, add legends!}}%
% %     \label{fig:distance_to_atom_r20}%
% \end{figure}%
%
\begin{figure}[htpb]%
    \centering% 
    \includesvg[%
        width=0.7\textwidth,
        svgpath = ./images/tetrahedral_order_parameter_new/%
    ]{figure01_bulk}%
    \caption{%
        Plot of tetrahedral order parameter for bulk water (water molecules further than 10 \AA\ from any silicon atoms), in the systems in \cref{fig:renderings_flat_square_fracture02,fig:renderings_flat_square_fracture03}.%
    }%
%     \label{fig:distance_to_atom_r20}%
\end{figure}%
%
\begin{figure}[!p]%
    \centering%
    \includesvg[%
        width=\textwidth,% choose which fits best, width or height (regular includegraphics can use both and add "keepaspectratio", but includesvg can't)
%         height=\textheight,%
        svgpath = ./images/tetrahedral_order_parameter_new/%
    ]{figure02_regular_and_wide}%
    {
        \captionsetup{width=\textwidth} 
        \caption{%
            Plot of tetrahedral order parameter for two regular rough fractures, and a flat reference system with a $\sim 86 \AA$ wide fracture. See \cref{fig:renderings_rough_fracture01_abel,fig:renderings_rough_fracture03,fig:renderings_flat_square_fracture03}.%
        }%
    }
%     \label{fig:distance_to_atom_r20}%
\end{figure}%
%
\begin{figure}[!p]%
    \centering% 
    \includesvg[%
        width=\textwidth,% choose which fits best, width or height (regular includegraphics can use both and add "keepaspectratio", but includesvg can't)
%         height=\textheight,%
        svgpath = ./images/tetrahedral_order_parameter_new/%
    ]{figure02_flat_and_wide01}%
    {
        \captionsetup{width=\textwidth} 
        \caption{%
            Plot of tetrahedral order parameter for a rough fracture with uniform width of $\sim 14.4\AA$, a flat reference system with a $14.4\AA$ wide fracture, and a flat reference system with a $\sim 86 \AA$ wide fracture. See \cref{fig:renderings_rough_fracture04_same_distance,fig:renderings_flat_fracture03,fig:renderings_flat_square_fracture03}.%
        }%
    }
%     \label{fig:distance_to_atom_r20}%
\end{figure}%
%
\begin{figure}[!p]%
    \centering% 
    \includesvg[%
        width=\textwidth,% choose which fits best, width or height (regular includegraphics can use both and add "keepaspectratio", but includesvg can't)
%         height=\textheight,%
        svgpath = ./images/tetrahedral_order_parameter_new/%
    ]{figure02_flat_and_wide02}%
    {
        \captionsetup{width=\textwidth} 
        \caption{%
            Plot of tetrahedral order parameter for a rough fracture with uniform width of $\sim 28.8\AA$, a flat reference system with a $28.8\AA$ wide fracture, and a flat reference system with a $\sim 86 \AA$ wide fracture. See \cref{fig:renderings_rough_fracture05,fig:renderings_flat_fracture03,fig:renderings_flat_square_fracture03}.%
        }%
    }
%     \label{fig:distance_to_atom_r20}%
\end{figure}%