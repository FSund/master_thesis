\section{Tetrahedral order parameter}
\todoa{Bin width 0.5 \AA\ and $\Delta Q = 0.01$ for all results! ($\Delta r$ not that important, since it's shown in the plots, but $\Delta Q$ isn't shown)}
% \todoa{Make two separate TOP figures, one for regular rough, and one for rough with same surfac x2 ?}
% \todoa{Update bulk TOP figure with results from long analysis run}
% %
% \begin{figure}[htpb]%
%     \centering%
%     \includesvg[width=1.0\textwidth, svgpath = ./images/tetrahedral_order_parameter/]{fancyfig04}%
%     \caption{}%
% %     \label{fig:distance_to_atom_r20}%
% \end{figure}%
%
% \begin{figure}[htpb]%
%     \centering%
%     \includesvg[%
% %         width=\textwidth,% choose which fits best, width or height (regular includegraphics can use both and add "keepaspectratio", but includesvg can't)
%         height=\textheight,%
%         svgpath = ./images/tetrahedral_order_parameter_new/]{figure01}%
%     \caption{\hl{Caption, add legends!}}%
% %     \label{fig:distance_to_atom_r20}%
% \end{figure}%
%

We have measured the thetrahedral order parameter for our four reference systems, and the four random fracture systems, and plotted the \hl{relative occurrence/probability} $P(Q)$ as function of $Q$, for sets of water molecules with different distances from the silica matrix. For comparison we also measured $P(Q)$ in bulk water in the two reference systems with 86 \AA\ wide flat pores, by measuring the tetrahedral order parameter for all water molecules further from the silica matrix than 10 \AA.
%
\begin{figure}[htpb]%
    \centering% 
    \includesvg[%
        width=0.6\textwidth,
        svgpath = ./images/tetrahedral_order_parameter_new/%
    ]{figure01_bulk}%
    \caption{%
        Plot of tetrahedral order parameter for bulk water (water molecules further than 10 \AA\ from any silicon atoms), in the two reference systems with 86 \AA\ wide flat pores (see \cref{fig:renderings_flat_square_fracture02,fig:renderings_flat_square_fracture03} for illustrations).%
    }%
%     \label{fig:distance_to_atom_r20}%
\end{figure}%
%
\begin{figure}[!p]%
    \centering%
    \includesvg[%
        width=\textwidth,% choose which fits best, width or height (regular includegraphics can use both and add "keepaspectratio", but includesvg can't)
%         height=\textheight,%
        svgpath = ./images/tetrahedral_order_parameter_new/%
    ]{figure02_regular_and_wide}%
    {
        \captionsetup{width=\textwidth} 
        \caption{%
            Plot of tetrahedral order parameter for two regular rough fractures, and a reference system with a 86 \AA\ wide flat por (see \cref{fig:renderings_rough_fracture01_abel,fig:renderings_rough_fracture03,fig:renderings_flat_square_fracture03} for illustrations).%
        }%
    }
%     \label{fig:distance_to_atom_r20}%
\end{figure}%
%
\begin{figure}[!p]%
    \centering% 
    \includesvg[%
        width=\textwidth,% choose which fits best, width or height (regular includegraphics can use both and add "keepaspectratio", but includesvg can't)
%         height=\textheight,%
        svgpath = ./images/tetrahedral_order_parameter_new/%
    ]{figure02_flat_and_wide01}%
    {
        \captionsetup{width=\textwidth} 
        \caption{%
            Plot of tetrahedral order parameter for a rough fracture with uniform width of 14.4 \AA\, a reference system with a 14.4 \AA\ wide flat pore, and a flat reference system with a 86 \AA\ wide flat pore (see \cref{fig:renderings_rough_fracture04_same_distance,fig:renderings_flat_fracture03,fig:renderings_flat_square_fracture03} for illustrations).%
        }%
    }
%     \label{fig:distance_to_atom_r20}%
\end{figure}%
%
\begin{figure}[!p]%
    \centering% 
    \includesvg[%
        width=\textwidth,% choose which fits best, width or height (regular includegraphics can use both and add "keepaspectratio", but includesvg can't)
%         height=\textheight,%
        svgpath = ./images/tetrahedral_order_parameter_new/%
    ]{figure02_flat_and_wide02}%
    {
        \captionsetup{width=\textwidth} 
        \caption{%
            Plot of tetrahedral order parameter for a rough fracture with uniform width of 28.8 \AA\, a reference system with a 28.8 \AA\ wide flat pore, and a reference system with a 86 \AA\ wide flat pore (see \cref{fig:renderings_rough_fracture05,fig:renderings_flat_fracture03,fig:renderings_flat_square_fracture03} for illustrations).%
        }%
    }
%     \label{fig:distance_to_atom_r20}%
\end{figure}%