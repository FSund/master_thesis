\section{Tetrahedral order parameter}
% \todoa{Bin width 0.5 \AA\ and $\Delta Q = 0.01$ for all results! ($\Delta Q$ isn't shown anywhere)}
% \todoa{Make two separate TOP figures, one for regular rough, and one for rough with same surfac x2 ?}
% \todoa{Update bulk TOP figure with results from long analysis run}
% %
% \begin{figure}[htpb]%
%     \centering%
%     \includesvg[width=1.0\textwidth, svgpath = ./images/tetrahedral_order_parameter/]{fancyfig04}%
%     \caption{}%
% %     \label{fig:distance_to_atom_r20}%
% \end{figure}%
%
% \begin{figure}[htpb]%
%     \centering%
%     \includesvg[%
% %         width=\textwidth,% choose which fits best, width or height (regular includegraphics can use both and add "keepaspectratio", but includesvg can't)
%         height=\textheight,%
%         svgpath = ./images/tetrahedral_order_parameter_new/]{figure01}%
%     \caption{\hl{Caption, add legends!}}%
% %     \label{fig:distance_to_atom_r20}%
% \end{figure}%
%

We have measured the thetrahedral order parameter for our four reference systems, and the four random fracture systems, and plotted the relative occurrence (or probability) $P(Q)$, for water molecules with different distances from the silica matrix. We used used 150 bins for the $Q$-values, with $\Delta Q = 0.01$, and bins of $\Delta r = 0.5$ for the different distances to the matrix, for all measurements presented here. The results are plotted in \crefrange{fig:first_top_figure}{fig:last_top_figure}. We have also measured the bulk order parameter, which is plotted in \cref{fig:top_bulk}.

We have estimated $P(Q)$ in bulk water by measuring the tetrahedral order parameter for all water molecules further from the silica matrix than 10 \AA, in the two reference systems with 86 \AA\ wide flat pores. The results for bulk water can be seen in \cref{fig:top_bulk}. We see that we get two peaks in the distribution, one just above $Q = 0.5$ and one right below $Q = 0.75$. These two peaks fit well with the results of Kumar et al. in \cite{kumar2009tetrahedral} for water at 300 K.
%
\begin{figure}[!htb]%
    \centering% 
    \includesvg[%
        width=0.5\textwidth,
        svgpath = ./images/tetrahedral_order_parameter_new/%
    ]{figure01_bulk}%
    \caption{%
        Plot of tetrahedral order parameter for bulk water (water molecules further than 10 \AA\ from the silica matrix), in the two reference systems with 86 \AA\ wide flat pores (reference system \#1 and \#2). The average $Q$ is indicated by a vertical line. %
    }%
    \label{fig:top_bulk}%
\end{figure}%

In \cref{fig:top_all_reference} we have plots of $P(Q)$ for all four reference systems, and in \cref{fig:top_all_rough} we have plots for all four random rough fracture systems. The overall trend we see is that the different geometries and pore widths doesn't affect $P(Q)$ a lot, but we see a clear change in the structure of water as we get closer than 5.5 \AA\ to the silica matrix. From 6.5 to 5.5 \AA\ $P(Q)$ is very similar to for bulk, but as we go closer than this the right peak get gradually lower, and $P(Q)$ is more spread out. This indicates a clear change in the structure of water molecules when we get close to the silica matrix, a change that seems almost independent of the structure of the surface of the matrix.

%
\begin{figure}[!p]%
    \centering% 
    \includesvg[%
        width=\textwidth,% choose which fits best, width or height (regular includegraphics can use both and add "keepaspectratio", but includesvg can't)
%         height=\textheight,%
        svgpath = ./images/tetrahedral_order_parameter_new/%
    ]{figure02_all_references}%
    \captionsetup{width=\textwidth}%
    \caption{%
        Plots of $P(Q)$ for all reference systems, for different distances to the silica matrix $r$. The solid lines are 86 \AA\ wide flat pores, and the dashed lines respectively a 14.4 \AA\ wide flat pore (\#3, red) and a 28.8 \AA\ wide flat pore (\#4, teal).%
        \label{fig:top_all_reference}%
    }%
\end{figure}%

In \cref{fig:top_all_reference} we have plots of $P(Q)$ for all four reference systems. We see that $P(Q)$ is almost identical for all four reference systems, at all distances from the silica matrix.

%
\begin{figure}[!p]%
    \centering% 
    \includesvg[%
        width=\textwidth,% choose which fits best, width or height (regular includegraphics can use both and add "keepaspectratio", but includesvg can't)
%         height=\textheight,%
        svgpath = ./images/tetrahedral_order_parameter_new/%
    ]{figure02_all_rough}%
    \captionsetup{width=\textwidth}%
    \caption{%
        Plots of $P(Q)$ for all all rough fracture systems, for different distances to the silica matrix $r$. The solid lines are random fractures generated from two different surfaces, and the dashed lines are random fractures of uniform width, generated from the same random surface repeated for the top and bottom of the fracture, with a distance of respectively 14.4 \AA\ (red, \#3) and 28.8 \AA\ (teal, \#4) between the surfaces.%
        \label{fig:top_all_rough}%
    }%
\end{figure}%

In \cref{fig:top_all_rough} we have plotted $P(Q)$ for all four rough fracture systems. We see that all four systems have very similar $P(Q)$ from 3.0 to 5.0 \AA, but at 5.5 \AA\ and further out the right near $Q = 0.75$ peak seems to rise faster for the two narrow fractures of uniform width (\#3 and \#4, the two dashed lines) than for the fractures with random width.

%
\begin{figure}[!p]%
    \centering% 
    \includesvg[%
        width=\textwidth,% choose which fits best, width or height (regular includegraphics can use both and add "keepaspectratio", but includesvg can't)
%         height=\textheight,%
        svgpath = ./images/tetrahedral_order_parameter_new/%
    ]{figure02_normal_and_reference}%
    \captionsetup{width=\textwidth}%
    \caption{%
        Plots of $P(Q)$ for the two random rough fractures (``rough'' \#1 and \#2, solid lines), and all four reference systems (dashed lines).%
        \label{fig:top_normal_and_reference}%
    }%
\end{figure}%

In \cref{fig:top_normal_and_reference} we have plots of $P(Q)$ for the two random rough fractures with varying pore width, and for all four reference systems. We see that the two peaks in $P(Q)$ generally lie lower for the rough fractures than the reference systems, for distances smaller than 5.5 \AA, but that this difference disappears at 6.0 and 6.5 \AA, where they all have bulk-like appearance. There are no clear differences between the two rough fracture systems.

%
\begin{figure}[!p]%
    \centering% 
    \includesvg[%
        width=\textwidth,% choose which fits best, width or height (regular includegraphics can use both and add "keepaspectratio", but includesvg can't)
%         height=\textheight,%
        svgpath = ./images/tetrahedral_order_parameter_new/%
    ]{figure02_narrow_and_reference}%
    \captionsetup{width=\textwidth}%
    \caption{%
        Plots of $P(Q)$ for the two random rough fractures of uniform width (solid lines), with fractures respectively 14.4 \AA\ (``rough \#3'', blue solid line) and 28.8 \AA\ (``rough \#4'', green solid line) wide, and all four reference systems (the dashed lines).%
        \label{fig:top_narrow_and_reference}%
    }%
\end{figure}%

In \cref{fig:top_narrow_and_reference} we have plots of $P(Q)$ for the two random fractures with uniform width, and for all four reference systems. We again see that the two peaks near $Q = 0.5$ and $Q = 0.75$ seems to be lower for the rough fractures than the flat pores, at least up to 5.5 \AA. As for the random rough fractures, there are again no clear differences between the two rough fracture systems with uniform width.

