\FloatBarrier
\chapter{Results}
\todoao{See \cite{bonnaud2010molecular} for references on density}%
\todoco{In all results that use distance to matrix: refer to \cref{fig:distance_to_matrix_illustration} when comparing to other results.}%
\todoco{Maybe also note that oxygen atoms closer than $\sim 3$~Å are probably bound to the silicon atom, which may come from our passivation, or from natural chemical reactions in the system.}%
Here we present the result of all the measurements we have done. Most of the measurements have been done for 200 different states for each system, with 100 timesteps of 0.050 picoseconds between each state (5 picoseconds between each state). Most measurements have been measured as function of distance to the silica matrix to study the effects of the interactions between water and silia, and we have tried to find the bulk-like behaviour of all measurements for comparison.

We will compare results between the random rough fractures to the results for bulk like-water in reference system \#1 and \#2, and against the two narrow flat reference systems \#3 and \#4, to see if the changes from flat pores to rough fractures have an effect on the water. The results for the rough fracture systems will also be compared against each other to see if the different fracture geometries have any effect.

We will first present and analyze the results from each measurement individually in \crefrange{sec:results_first_sec}{sec:results_last_sec}, and then later in \cref{chap:ending} we will give a summary of the results, and compare the results from the different measurements against each other to see if there are any further conclusions to be drawn.

% \todoa{Find diffusion for one atom?}

% \section{Density of water\label{sec:results_density}}
% We measure the density as function of distance to the matrix in all our systems. We ended up with ``bulk'' densities in the systems ranging from around 1000 kg/m$^3$ to 1150 kg/m$^3$, so for easier comparison we have normalized the densities against the bulk density in each system. Here we have defined the bulk density as the average density of water molecules more than $\sim 10\text{ \AA}$ from the nearest silicon atom. 

We have measured the density of water as function of distance to the silica matrix in all of our systems, averaged over 200 states for each system, with 100 timesteps of 0.050 picoseconds between each state (5 picoseconds between each state). The results are plotted in \crefrange{fig:first_density_fig}{fig:last_density_fig}. We measured for distances ranging from 0 to 10 \AA\ from the silica matrix, in steps of 0.25 \AA. We have also measured the bulk density in the systems where this was possible, the results of which are listed in \cref{tab:bulk_water_density}.%
%
% ---- Subcaption/subfigure version ---- %
\begin{figure}[!htb]%
    % \centering%
    \setlength{\myfigwidth}{0.58\textwidth}%  % change "b" to "t" to anchor top instead of bottom
    \makebox[\textwidth][c]{ % to center figures below that are wider than \textwidth
        \begin{minipage}[t]{\myfigwidth}%
            \captionsetup[subfigure]{width=0.9\textwidth}%
            \centering%
            \includesvg[width=\textwidth, svgpath=./images/density/]{density_water04_reference}%
            \subcaption{%
                Dashed lines are (red, \#3) 14.4 \AA\ and (teal, \#4) 28.8 \AA\ narrow flat pores, solid lines are 86 \AA\ wide flat pores.%
                \label{fig:density_reference_systems}%
            }%
        \end{minipage}%
        \hfill%
        \begin{minipage}[t]{\myfigwidth}%
            \captionsetup[subfigure]{width=0.9\textwidth}%
            \centering%
            \includesvg[width=\textwidth, svgpath=./images/density/]{density_water04_rough}%
            \subcaption{%
                Dashed lines are (red, \#3) 14.4 \AA\ and (teal, \#4) 28.8 \AA\ narrow fractures, solid lines are random fractures.%
                \label{fig:density_rough_systems}%
            }%
        \end{minipage}%
    }%
    \vspace{8pt}%
    \caption{%
        Water density ($\rho$) as function of distance to silica matrix ($r$) in \textbf{(a)} all four reference systems (flat pores) and \textbf{(b)} rough fracture systems.%
        %In \textbf{(a)} the dashed lines are 14.4 (red dashed line, \#3) and 28.8 (teal dashed line, \#4) \AA\ narrow flat pores \textbf{a)} flat pores and \textbf{b)} narrow, uniform width fractures. \hl{FINISH CAPTION}.%
        \label{fig:first_density_fig}%
    }%
\end{figure}%

The most significant trend we notice in \cref{fig:first_density_fig} is that the water density seems stable at 9-10 \AA\ from the silica matrix, but as we move closer than this we see a clear reduction in density. When we go from 10 \AA\ and move closer to the matrix we see in all systems that the density falls off at around 7-8 \AA\, and is reduced by almost 10\% at around 5 \AA. The density then increases to densities higher than the initial densities (the ones at 10 \AA) when we move towards 3 \AA. The relative reduction and then increase in density seems to be similar in all systems with similar overall densities, but in the systems with the highest overall density the relative change seems to be a bit smaller than in the other systems.

We now look at the density in the four reference systems, which is plotted in \cref{fig:density_reference_systems}. We see that the densities all have the same quantitative behaviour as we move further from the silica matrix, even though the net density is very different in the four systems, with the densities stabilizing at values ranging from 1050 to almost 1150 kg/m$^3$ at 8-10 \AA. One trend we notice is that the point where the density stabilizes, at around 7 \AA, seems to appear closer to the matrix when we increase the overall density. The minimum point seems to appear at approximately the same distance for the systems with a 86 \AA\ wide flat pore (system \#1 and \#2), but a bit closer to the matrix for the narrow flat pores (system \#4 and \#3).

In \cref{fig:density_rough_systems} we have plotted the density in the four random fracture systems. We see that the density in all four systems are very similar at all distances to the matrix, and that the density seems to stabilize between 1010 and 1025 kg/m$^3$ at 8-10 \AA\ from the silica matrix. We see that even though these four fractures have different geometries (especially system \#1 and \#2 are very different from system \#3 and \#4), the behaviour of the density as we go from 8 to 3 \AA\ from the silica matrix seems to be the same in all four systems. We also see that the density seems peak at around 8 \AA, and stabilize at a somewhat lower value as we go further from the silica matrix. This peak is similar to a peak observed around 5 \AA\ from the silica matrix by Bonnaud et al. in \cite{bonnaud2010molecular} (see Figure 6 in the article). In the article they use a different measure for the distance to the silica matrix, which might explain the difference in the distance at which the peak is observed.

We have estimated the bulk water density by the same technique we use for measuring water density as function of distance to the silica matrix, but now by averaging the density for all water-oxygen atoms further away from the silica matrix than a certain distance (typically 10 \AA\ or more), instead of using bins. This measurement requires that we have a fracture at least twice as wide as this distance to have any atoms we can measure the density of, so estimating the bulk density was only possible in reference system \#1 and \#2 with a flat, constant fracture of 86 \AA, and in the two random fractured systems \#1 and \#2, which have fractures with varying width. %We measured the bulk distance using a minimum distance to the matrix of 10 and 30 \AA\, to see if there was a difference between these two limits.%
%
\begin{table}[!htb]%
    \centering%
    \begin{tabular}{l|cc|cc}%
    ~               & \multicolumn{2}{c|}{$r>10$ \AA}& \multicolumn{2}{c}{$r>30$ \AA}    \\
    \textit{System} & Density [kg/m$^3$]    & N     & Density [kg/m$^3$]    & N   \\ \hline
    Reference \#1   & 1038.4                & 49k   & 1038.5                & 20k \\
    Reference \#2   & 1092.7                & 52k   & 1090.4                & 21k \\
    Rough \#1       & 1017.9                & 5.0k  & -                     & 0   \\
    Rough \#2       & 1002.8                & 4.2k  & -                     & 0   \\
    \end{tabular}%
    \vspace{8pt}%
    \caption{%
        Estimated bulk water densities, and the number of water molecules used in the calculations. Estimated using Voronoi tesselation, averaged over voronoi volumes for all water-oxygen atoms further away from silica matrix than 10 and 30 \AA\ (hydrogen atoms were removed before Voronoi tesselation). %
        \label{tab:bulk_water_density}%
    }%
\end{table}%
% bulk density (r>10.0) rough_fracture01_abel = 1018
% bulk density (r>10.0) rough_fracture03 = 1003
% bulk density (r>10.0) flat_square_02 = 1038
% bulk density (r>10.0) flat_square_03 = 1094

The estimated bulk densities are listed in \cref{tab:bulk_water_density}, where we have measured the bulk density for water atoms at least 10 \AA\ and at least 30 \AA\ from the silica matrix. We see that the estimated bulk density does not change if we use 10 or 30 \AA\ as the minimum distance from the matrix, indicating that a limit of 10 \AA\ is probably safe. If we compare the bulk densities to the plots of the density as function of distance from the matrix in \cref{fig:density_reference_systems,fig:density_rough_systems}, we see that the density seems to be stable close to the bulk density at around 10 \AA.

\todoa{Write about normalized densities, why we normalize, vacuum problems..}
\todoa{Compare reference systems to rough systems}

When we compare rough and flat we see that the peak/fall-off appears further from the silica in rough than flat.

%
\begin{figure}[!p]%
    \centering%
    \setlength{\myfigwidth}{0.88\textwidth}
    \setlength{\mycaptionwidth}{0.09999\textwidth}
%         \includesvg[width=0.7\textwidth, svgpath=./images/density/]{density_water04_all}%
%         \subcaption{}%
%     \vspace{8pt}%
    \begin{minipage}[c]{\myfigwidth}%
        \includesvg[width=\textwidth, svgpath=./images/density/]{density_water04_all}%
    \end{minipage}%
    \begin{minipage}[c]{\mycaptionwidth}%
        \subcaption{}%
%         \label{fig:simple_square_step}%
    \end{minipage}%
    \\%
    \begin{minipage}[c]{\myfigwidth}%
        \includesvg[width=\textwidth, svgpath=./images/density/]{density_water04_all_normalized_divide}%
    \end{minipage}%
    \begin{minipage}[c]{\mycaptionwidth}%
        \subcaption{}%
%         \label{fig:simple_square_step}%
    \end{minipage}%
    %
%         \includesvg[width=0.7\textwidth, svgpath=./images/density/]{density_water04_all_normalized_divide}%
%         \subcaption{}%
    \captionsetup{width=\textwidth}% since it's a floatpage figure
    \caption{%
        Density of water in all systems, as function of distance from the silica matrix. Dashed lines are reference systems, and solid lines rough fracture systems. In \textbf{(b)} the density has been normalized against the approximate bulk density, estimated from the density at 10 \AA\ from the silica matrix. %
%         \hl{reference figure or remove} \hl{FINISH CAPTION}. %
%         \label{fig:density_flat_and_wide_not_normalized}%
    }%
\end{figure}%
% \FloatBarrier
% \section{Diffusion/mean square displacement}
\todod{Diffusion normal to and parallel to surface?}
\todob{Plot of $r^2$ for 200 and 40 states to argue for origo move?}

% bulk diff in Ref \#1 = 0.202034846318
%   N = mean(47577, 47524, 47555, 47552, 47559) = 
% bulk diff in Ref \#2 = 0.178684448075
%   N = mean(50280, 50284, 50291, 50275, 50321)

% bulk diff in Rough \#1 = 0.197721045372      
%   N = mean(4271, 4279, 4252, 4294, 4284)
% bulk diff in Rough \#2 = 0.208658027239  
%   N = mean(3414, 3396, 3419, 3431, 3393)


To produce these results we have averaged over 200 states, with 100 timesteps of 20.67 md units between each timestep ($\sim 0.5$ picoseconds\hl{???}), divided into 5 non-overlapping origos, with 40 states per origo.

We first look at the diffusion as function of distance to the silica matrix in the four reference systems, which we have plotted in \cref{fig:diffusion_reference_systems}. We see that the diffusion constant, $D$, has similar quantitative behaviour as we move further from the silica matrix in all four reference systems, but that the actual $D$ is different in each system. The two systems with 86 \AA\ wide flat pores have diffusion constants that differ by around 10-30\%, even though those two systems should be pretty similar. The main difference between those two systems is the density, as we saw in \cref{sec:results_density}. We have listed the bulk $D$ in \cref{tab:bulk_water_diffusion}, and we see that the bulk diffusion for the two reference systems with 86 \AA\ wide flat pores is {0.202 \AA$^2$/ps}, while system \#2 has {0.179 \AA$^2$/ps}, with a difference of {13 \%}.
%
%
% \begin{figure}[htpb]%
%     \centering%
%     \includesvg[width=0.6\textwidth, svgpath=./images/diffusion/]{diffusion_constant_move_origin_reference01}%
%     \caption{%
%         Diffusion constant as function of distance to silica matrix for all four reference systems (flat fractures). The solid lines are for the two systems with 86 \AA\ wide fractures, and the dashed lines for narrow fractures of 14.4 and 28.8 \AA. \hl{FINISH CAPTION}. %
%         \label{fig:diffusion_reference_systems}%
%     }%
% \end{figure}%

In \cref{fig:diffusion_rough_systems} we have plotted the diffusion constant %as function of distance from the silica matrix 
for the four random fracture systems, ``rough \#1'' through 4. We again see a quantitative similar behaviour, and we see that all systems except the 14.4 \AA\ narrow fracture (``rough \#3'') have very similar diffusion constants. In \cref{tab:bulk_water_diffusion} see that the bulk diffusion constants in the two regular random fractures (\#1 and \#2) are similar, with a difference of {6 \%}. We also see that the $D$ for water at 7-10 \AA\ from the silica matrix is close to the bulk constant for these two systems.
\todoao{More about diffusion?}
%
% \begin{figure}[htpb]%
%     \centering%
%     \includesvg[width=0.6\textwidth, svgpath=./images/diffusion/]{diffusion_constant_move_origin_rough01}%
%     \caption{%
%         Diffusion constant as function of distance to silica matrix for all four random/rough fractures. \hl{FINISH CAPTION}. %
%         \label{fig:diffusion_rough_systems}%
%     }%
% \end{figure}%
%
%
\begin{figure}[htpb]%
% \centering%
\setlength{\myfigwidth}{0.58\textwidth}%
\makebox[\textwidth][c]{ % to center figures below that are wider than \textwidth
    \begin{minipage}[t]{\myfigwidth}%
        \captionsetup{width=0.925\textwidth}%
        \centering%
        \includesvg[width=\textwidth, svgpath=./images/diffusion/]{diffusion_constant_move_origin_reference01}%
        \caption{%
            Diffusion constant as function of distance to silica matrix for all four reference systems (flat pores). The solid lines are for the two systems with 86 \AA\ wide pores, and the dashed lines for narrow fractures of 14.4 and 28.8 \AA. \hl{FINISH CAPTION}. %
            \label{fig:diffusion_reference_systems}%
        }%
    \end{minipage}%
    \hfill%
    \begin{minipage}[t]{\myfigwidth}% % change "b" to "t" to anchor top instead of bottom
        \captionsetup{width=0.925\textwidth}% % minipage defines a \textwidth for it's own, so we have to repeat this command inside the minipage
        \centering%
        \includesvg[width=\textwidth, svgpath=./images/diffusion/]{diffusion_constant_move_origin_rough01}%
        \caption{%
            Diffusion constant as function of distance to silica matrix for all four random rough fractures. Solid lines are for regular random fractures, and dashed lines for narrow fractures with uniform width of 14.4 and 28.8 \AA. \hl{FINISH CAPTION}. %
            \label{fig:diffusion_rough_systems}%
        }%
    \end{minipage}%
}
\end{figure}%

% \begin{figure}[htpb]%
%     \centering%
%     {
%         \newcommand{\f}{\footnotesize}
%         \includesvg[width=0.7\textwidth, svgpath=./images/diffusion/]{diffusion01}%
%     }
%     \caption{%
%         Diffusion. \hl{Make new figure using new diffusion program}. %
% %         \label{fig:cell_lists}%
%     }%
% \end{figure}%

% \begin{figure}[htpb]%
%     \centering%
%     {
%         \newcommand{\f}{\footnotesize}%
%         \includesvg[width=0.7\textwidth, svgpath=./images/diffusion/]{diffusion_constant02}%
%     }
%     \caption{%
%         Diffusion. \hl{Make new figure using new diffusion program} \hl{FINISH CAPTION}. %
% %         \label{fig:cell_lists}%
%     }%
% \end{figure}%

The bulk diffusion constants have been estimated in the two reference systems with 86 \AA\ wide flat pores (reference \#1 and \#2), and in rough system \#1 and \#2, by measuring $D$ for all water molecules further away from the silica matrix than 10 \AA. The results can be seen in \cref{tab:bulk_water_diffusion}. The other four systems have very narrow pores and fractures, with most of the water molecules closer to the silica matrix than 10 \AA\, so we haven't measured the bulk diffusion in those systems, and we don't expect much of the water to have bulk behaviour\todobo{explain this better?}.
%
\begin{table}[!htb]%
    \centering%
    \begin{tabular}{l|cc}%
        \textit{System} & Bulk $D$ [\AA$^2$/ps] & N    \\\hline
        Reference \#1   & 0.202                 & 48k  \\ % 0.202034846318
        Reference \#2   & 0.179                 & 50k  \\ % 0.178684448075
        Rough \#1       & 0.198                 & 4.3k \\ % 0.197721045372
        Rough \#2       & 0.209                 & 3.4k \\ % 0.208658027239
    \end{tabular}%
    \vspace{8pt}%
    \caption{%
        Bulk diffusion constant for water (for $r>10$ \AA), and the number of water molecules used in the calculations. %
        \label{tab:bulk_water_diffusion}%
    }%
\end{table}

\begin{figure}[htpb]%
% \centering%
\setlength{\myfigwidth}{0.58\textwidth}%
\makebox[\textwidth][c]{ % to center figures below that are wider than \textwidth
    \begin{minipage}[t]{\myfigwidth}%
        \captionsetup{width=0.925\textwidth}%
        \centering%
        \includesvg[width=\textwidth, svgpath=./images/diffusion/]{diffusion_constant_move_origin_normal01}%
        \caption{%
            Diffusion for random rough fractures and reference systems. \hl{reference figure or remove} \hl{FINISH CAPTION}. %
    %         \label{fig:cell_lists}%
        }%
    \end{minipage}%
    \hfill%
    \begin{minipage}[t]{\myfigwidth}% % change "b" to "t" to anchor top instead of bottom
        \captionsetup{width=0.925\textwidth}% % minipage defines a \textwidth for it's own, so we have to repeat this command inside the minipage
        \centering%
        \includesvg[width=\textwidth, svgpath=./images/diffusion/]{diffusion_constant_move_origin_narrow01}%
        \caption{%
            Diffusion for random narrow fractures and reference systems. \hl{reference figure or remove} \hl{FINISH CAPTION}. %
    %         \label{fig:cell_lists}%
        }%
    \end{minipage}%
}
\end{figure}%

% \begin{figure}[htpb]%
%     \centering%
%     \includesvg[width=0.8\textwidth, svgpath=./images/diffusion/]{diffusion_constant_move_origin_normal01}%
%     \caption{%
%         Diffusion for random rough fractures and reference systems. \hl{reference figure or remove} \hl{FINISH CAPTION}. %
% %         \label{fig:cell_lists}%
%     }%
% \end{figure}%
% 
% \begin{figure}[htpb]%
%     \centering%
%     \includesvg[width=0.8\textwidth, svgpath=./images/diffusion/]{diffusion_constant_move_origin_narrow01}%
%     \caption{%
%         Diffusion for random narrow fractures and reference systems. \hl{reference figure or remove} \hl{FINISH CAPTION}. %
% %         \label{fig:cell_lists}%
%     }%
% \end{figure}%

% \begin{figure}[htpb]%
%     \centering%
%     \includegraphics[width=0.5\textwidth]{images/diffusion/mean_square_displacement_interesting.png}%
%     \caption{%
%         Something interesting (msd stops increasing for a couple of angstrom near 4.5-5, 5-5.5, 5.5-6.0) \hl{reference figure or remove} \hl{FINISH CAPTION}. %
% %         \label{fig:cell_lists}%
%     }%
% \end{figure}%

% \begin{figure}[htpb]%
%     \centering%
%     \setlength{\myfigwidth}{0.49\textwidth}%
% %     \setlength{\mycaptionwidth}{0.3\textwidth}%
% %
%     \begin{subfigure}[b]{\myfigwidth}%
%         \includesvg[width=\textwidth, svgpath=./images/diffusion/]{diffusion_constant_move_origin01}%
%         \caption{%
%             Diffusion. \hl{FINISH CAPTION}. %
%     %         \label{fig:cell_lists}%
%         }%
%     \end{subfigure}%
%     \hfill%
%     \begin{subfigure}[b]{\myfigwidth}%
%         \includesvg[width=\textwidth, svgpath=./images/diffusion/]{diffusion_constant_move_origin02}%
%         \caption{%
%             Diffusion. \hl{FINISH CAPTION}. %
%     %         \label{fig:cell_lists}%
%         }%
%     \end{subfigure}%
%     \caption{%
%         rough\_fracture\_01\_abel - ``Rough fracture \#1'' \hl{Caption} %
%         \label{fig:renderings_rough_fracture01_abel}%
%     }%
% \end{figure}%
% \FloatBarrier
% \section{Tetrahedral order parameter}
% \todoa{Make two separate TOP figures, one for regular rough, and one for rough with same surfac x2 ?}
\todoa{Update bulk TOP figure with results from long analysis run}
% %
% \begin{figure}[htpb]%
%     \centering%
%     \includesvg[width=1.0\textwidth, svgpath = ./images/tetrahedral_order_parameter/]{fancyfig04}%
%     \caption{}%
% %     \label{fig:distance_to_atom_r20}%
% \end{figure}%
%
% \begin{figure}[htpb]%
%     \centering%
%     \includesvg[%
% %         width=\textwidth,% choose which fits best, width or height (regular includegraphics can use both and add "keepaspectratio", but includesvg can't)
%         height=\textheight,%
%         svgpath = ./images/tetrahedral_order_parameter_new/]{figure01}%
%     \caption{\hl{Caption, add legends!}}%
% %     \label{fig:distance_to_atom_r20}%
% \end{figure}%
%
\begin{figure}[htpb]%
    \centering% 
    \includesvg[%
        width=0.7\textwidth,
        svgpath = ./images/tetrahedral_order_parameter_new/%
    ]{figure01_bulk}%
    \caption{%
        Plot of tetrahedral order parameter for bulk water (water molecules further than 10 \AA\ from any silicon atoms), in the systems in \cref{fig:renderings_flat_square_fracture02,fig:renderings_flat_square_fracture03}.%
    }%
%     \label{fig:distance_to_atom_r20}%
\end{figure}%
%
\begin{figure}[!p]%
    \centering%
    \includesvg[%
        width=\textwidth,% choose which fits best, width or height (regular includegraphics can use both and add "keepaspectratio", but includesvg can't)
%         height=\textheight,%
        svgpath = ./images/tetrahedral_order_parameter_new/%
    ]{figure02_regular_and_wide}%
    {
        \captionsetup{width=\textwidth} 
        \caption{%
            Plot of tetrahedral order parameter for two regular rough fractures, and a flat reference system with a $\sim 86 \AA$ wide fracture. See \cref{fig:renderings_rough_fracture01_abel,fig:renderings_rough_fracture03,fig:renderings_flat_square_fracture03}.%
        }%
    }
%     \label{fig:distance_to_atom_r20}%
\end{figure}%
%
\begin{figure}[!p]%
    \centering% 
    \includesvg[%
        width=\textwidth,% choose which fits best, width or height (regular includegraphics can use both and add "keepaspectratio", but includesvg can't)
%         height=\textheight,%
        svgpath = ./images/tetrahedral_order_parameter_new/%
    ]{figure02_flat_and_wide01}%
    {
        \captionsetup{width=\textwidth} 
        \caption{%
            Plot of tetrahedral order parameter for a rough fracture with uniform width of $\sim 14.4\AA$, a flat reference system with a $14.4\AA$ wide fracture, and a flat reference system with a $\sim 86 \AA$ wide fracture. See \cref{fig:renderings_rough_fracture04_same_distance,fig:renderings_flat_fracture03,fig:renderings_flat_square_fracture03}.%
        }%
    }
%     \label{fig:distance_to_atom_r20}%
\end{figure}%
%
\begin{figure}[!p]%
    \centering% 
    \includesvg[%
        width=\textwidth,% choose which fits best, width or height (regular includegraphics can use both and add "keepaspectratio", but includesvg can't)
%         height=\textheight,%
        svgpath = ./images/tetrahedral_order_parameter_new/%
    ]{figure02_flat_and_wide02}%
    {
        \captionsetup{width=\textwidth} 
        \caption{%
            Plot of tetrahedral order parameter for a rough fracture with uniform width of $\sim 28.8\AA$, a flat reference system with a $28.8\AA$ wide fracture, and a flat reference system with a $\sim 86 \AA$ wide fracture. See \cref{fig:renderings_rough_fracture05,fig:renderings_flat_fracture03,fig:renderings_flat_square_fracture03}.%
        }%
    }
%     \label{fig:distance_to_atom_r20}%
\end{figure}%
% \FloatBarrier
% \section{Distance to nearest atom}
% \todoa{Mention that images are made with no water molecules (only passivation atoms) in system}
%
We have made 3d maps of the system labelled ``rough fracture \#2'', using the method from \cref{sec:distance_to_atom}, which finds the distance to the nearest atom on a grid of points. The results can be seen in \cref{fig:distance_to_atom_r05,fig:distance_to_atom_r20}, where we show slices of the maps in the $yz$- and $xy$-plane. For the first figure we used a max distance of 5 \AA, while for the second one we used a max distance of 20 \AA. The maps were made from the molecular system after cutting out the atoms to make the pore and passivating the dangling ends, but before filling the pore with water. A similar map can me made by not including the water molecules in the calculations.

In \cref{fig:distance_to_atom_r05} we can clearly see the positions of the atoms in the silica matrix as the dark blue dots, while the pores light up as red areas.

In \cref{fig:distance_to_atom_r20} we still see the positions of the atoms, but they are less visible now since we have a bigger range for the colormap. We can still see the pores easily, colored white, and we now also see some characteristics of the pore itself, where it has a darker red color.%
%
\begin{figure}[!p]%
    \centering%
    \setlength{\myfigwidth}{\textwidth}%
    \begin{subfigure}[b]{\myfigwidth}%
        \includesvg[pretex=\normalsize, width=\textwidth, svgpath = ./images/distance_to_atom/rough_fracture_03/]{05_r05_n256}%
        \caption{Max distance $r_\text{max}=5.0$ \AA.%
        \label{fig:distance_to_atom_r05}}%
    \end{subfigure}%
    \vspace{10pt}
    \begin{subfigure}[b]{\myfigwidth}%
        \includesvg[pretex=\normalsize, width=\textwidth, svgpath = ./images/distance_to_atom/rough_fracture_03/]{05_r20_n256}%
        \caption{Max distance $r_\text{max}=20$ \AA.%
        \label{fig:distance_to_atom_r20}}%
    \end{subfigure}%
    \captionsetup{width=\textwidth}%
    \caption{%
        Slices of 3d maps of the distance to the nearest atom in the system labelled ``rough fracture \#2'', generated using method from \cref{sec:distance_to_atom}, using a colormap that goes from \textbf{(a)} 0 to 5 \AA, and \textbf{(b)} 0 to 20 \AA.%
    }%
\end{figure}%
% % \FloatBarrier
% \section{Manhattan distance to nearest atom}
% \todo{replace these with results from actual fracture system?}
%Not very useful. Much of the same as distance to atom, only worse (but faster). Quantitatively much the same as distance to atom?
%
We have made 3d maps of the system labelled ``rough fracture \#2'', of the Manhattan distance to the nearest atom to each point on a grid, using the method from \cref{sec:generation_matrix}. See \cref{fig:generation_matrix_r05,fig:generation_matrix_r11} for the results, where we show slices of the maps in the $yz$- and $xy$-plane.

We see that the 3d maps made using this method shows a lot of the same details as the maps in \cref{fig:distance_to_atom_r05,fig:distance_to_atom_r20}, but we see that using the Manhattan distance can make the results harder to interpret, since the Manhattan distance between two points is generally shorter than the Euclidean distance, and we are not used to interpreting the Manhattan distance. In molecular dynamics simulations it is also unusual to use the Manhattan distance.%
%
\begin{figure}[!p]%
    \centering%
    \setlength{\myfigwidth}{\textwidth}%
    \begin{subfigure}[b]{\myfigwidth}%
        \includesvg[pretex=\normalsize, width=\myfigwidth, svgpath = ./images/generation_matrix/rough_fracture03/]{05_r0745_n256}%
        \caption{Max distance $r_\text{max}=5.0$ \AA.%
        \label{fig:generation_matrix_r05}}%
    \end{subfigure}%
    \vspace{10pt}
    \begin{subfigure}[b]{\myfigwidth}%
        \includesvg[pretex=\normalsize, width=\myfigwidth, svgpath = ./images/generation_matrix/rough_fracture03/]{05_r308_n256}%
        \caption{Max distance $r_\text{max}=20$ \AA.%
        \label{fig:generation_matrix_r11}}%
    \end{subfigure}%
    \captionsetup{width=\textwidth}%
    \caption{%
        Slices of 3d maps of the system labelled ``rough fracture \#2'', made using the method from \cref{sec:generation_matrix}. The method generates a 3d map of the Manhattan distance to the nearest atom, in each point on a grid. We have used colormaps from 0 to $r_\text{max}$, with \textbf{(a)} $r_\text{max} = 8$ \AA, and \textbf{(b)} $r_\text{max} = 30$ \AA.%
    }%
\end{figure}%
% \FloatBarrier

% \section{Area}
% \section{Volume?}

