\chapter{Re}
\section{Density}
\begin{figure}[htpb]%
    \centering%
    \includesvg[width=0.7\textwidth, svgpath=./images/density/]{density_water02}%
    \caption{%
        Density of water. \hl{FINISH CAPTION}. %
%         \label{fig:cell_lists}%
    }%
\end{figure}%
\begin{figure}[htpb]%
    \centering%
    \includesvg[width=0.7\textwidth, svgpath=./images/density/]{number_of_molecules02}%
    \caption{%
        Number of molecules. Bin width = 0.2 \AA?. \hl{FINISH CAPTION}. %
%         \label{fig:cell_lists}%
    }%
\end{figure}%

\FloatBarrier
\section{Diffusion/mean square displacement}
\todo[inline]{Diffusion normal to and parallel to surface?}
\begin{figure}[htpb]%
    \centering%
    {
        \newcommand{\f}{\footnotesize}
        \includesvg[width=0.7\textwidth, svgpath=./images/diffusion/]{diffusion01}%
    }
    \caption{%
        Diffusion. \hl{FINISH CAPTION}. %
%         \label{fig:cell_lists}%
    }%
\end{figure}%

\begin{figure}[htpb]%
    \centering%
    {
        \newcommand{\f}{\footnotesize}%
        \includesvg[width=0.7\textwidth, svgpath=./images/diffusion/]{diffusion_constant02}%
    }
    \caption{%
        Diffusion. \hl{Consider fixing label background...} \hl{FINISH CAPTION}. %
%         \label{fig:cell_lists}%
    }%
\end{figure}%

\FloatBarrier
\section{Distance to atom}
\todo{replace these with results from actual fracture system?}
We developed a program that finds the distance to the nearest atom, in all points of the 
%
\setlength{\myfigwidth}{0.90\textwidth}%
\begin{figure}[htpb]%
    \centering%
    \includesvg[width=\myfigwidth, svgpath = ./images/distance_to_atom/]{SiO2_06_slice_r05_n256}%
    \caption{$r = 5$ \Ang}%
    \label{fig:distance_to_atom_r05}%
\end{figure}%
%
\begin{figure}[htpb]%
    \centering%
    \includesvg[width=\myfigwidth, svgpath = ./images/distance_to_atom/]{SiO2_06_slice_r20_n256}%
    \caption{$r = 20$ \Ang}%
    \label{fig:distance_to_atom_r20}%
\end{figure}%

\FloatBarrier
\section{``Generation matrix''}
\todo{replace these with results from actual fracture system?}
Not very useful. Much of the same as distance to atom, only worse (but faster).
%
\begin{figure}[htpb]%
    \centering%
    \includesvg[width=\myfigwidth, svgpath = ./images/generation_matrix/]{SiO2_06_slice_r05_n256}%
    \caption{5 generations}%
    \label{fig:generation_matrix_r05}%
\end{figure}%
%
\begin{figure}[htpb]%
    \centering%
    \includesvg[width=\myfigwidth, svgpath = ./images/generation_matrix/]{SiO2_06_slice_r11_n256}%
    \caption{11 generations}%
    \label{fig:generation_matrix_r11}%
\end{figure}%
%
\begin{figure}[htpb]%
    \centering%
    \includesvg[width=\myfigwidth, svgpath = ./images/generation_matrix/]{SiO2_06_slice_r40_n256}%
    \caption{40 generations}%
    \label{fig:generation_matrix_r40}%
\end{figure}%

\FloatBarrier
\section{Tetrahedral order parameter}


% \setlength{\myfigwidth}{0.499\textwidth}%
% \begin{figure}[htpb]%
%     \centering%
%     \begin{subfigure}[b]{\myfigwidth}%
%         \includesvg[width=\textwidth, svgpath=./images/tetrahedral_order_parameter/]{figure04}%
%         \caption{Caption}%
% %         \label{fig:pass_tet01}%
%     \end{subfigure}%
% %     \hspace{0.05\textwidth}%
%     \begin{subfigure}[b]{\myfigwidth}%
%         \includesvg[width=\textwidth, svgpath=./images/tetrahedral_order_parameter/]{figure07}%
%         \caption{Caption}%
% %         \label{fig:pass_tet02}%
%     \end{subfigure}%
%     
%     \begin{subfigure}[b]{\myfigwidth}%
%         \includesvg[width=\textwidth, svgpath=./images/tetrahedral_order_parameter/]{figure09}%
%         \caption{Caption}%
% %         \label{fig:pass_tet02}%
%     \end{subfigure}%
%     \begin{subfigure}[b]{\myfigwidth}%
%         \includesvg[width=\textwidth, svgpath=./images/tetrahedral_order_parameter/]{figure10}%
%         \caption{Caption}%
% %         \label{fig:pass_tet02}%
%     \end{subfigure}%
%     
%     \begin{subfigure}[b]{\myfigwidth}%
%         \includesvg[width=\textwidth, svgpath=./images/tetrahedral_order_parameter/]{figure11}%
%         \caption{Caption}%
% %         \label{fig:pass_tet02}%
%     \end{subfigure}%
%     \begin{subfigure}[b]{\myfigwidth}%
%         \includesvg[width=\textwidth, svgpath=./images/tetrahedral_order_parameter/]{figure12}%
%         \caption{Caption}%
% %         \label{fig:pass_tet02}%
%     \end{subfigure}%
%     %
%     \caption{%
%         \hl{Caption}%
%     }%
% %     \label{fig:inject_empty_voxel}%
% \end{figure}%

%
\begin{figure}[htpb]%
    \centering%
    \includesvg[width=1.0\textwidth, svgpath = ./images/tetrahedral_order_parameter/]{fancyfig04}%
    \caption{}%
%     \label{fig:distance_to_atom_r20}%
\end{figure}%

\FloatBarrier
\section{Area}
\section{Volume?}

