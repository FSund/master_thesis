\section{Density of water\label{sec:results_density}}
% We measure the density as function of distance to the matrix in all our systems. We ended up with ``bulk'' densities in the systems ranging from around 1000 kg/m$^3$ to 1150 kg/m$^3$, so for easier comparison we have normalized the densities against the bulk density in each system. Here we have defined the bulk density as the average density of water molecules more than $\sim 10\text{ \AA}$ from the nearest silicon atom. 

We have measured the density of water as function of distance to the silica matrix in all of our systems, averaged over 200 states for each system, with 100 timesteps of 0.050 picoseconds between each state (5 picoseconds between each state). The results are plotted in \crefrange{fig:first_density_fig}{fig:last_density_fig}. We measured for distances ranging from 0 to 10 \AA\ from the silica matrix, in steps of 0.25 \AA. We have also measured the bulk density in the systems where this was possible, the results of which are listed in \cref{tab:bulk_water_density}. The plots start at 3.0 \AA, since water molecules closer to the silica matrix than this are most likely bound to silica atoms, and are part of the passivating silanol groups, as we saw in \cref{sec:distance_to_matrix_issues}.%
%
% ---- Subcaption/subfigure version ---- %
\begin{figure}[!htb]%
    % \centering%
    \setlength{\myfigwidth}{0.58\textwidth}%  % change "b" to "t" to anchor top instead of bottom
    \makebox[\textwidth][c]{ % to center figures below that are wider than \textwidth
        \begin{minipage}[t]{\myfigwidth}%
            \captionsetup[subfigure]{width=0.9\textwidth}%
            \centering%
            \includesvg[width=\textwidth, svgpath=./images/density/]{density_water04_reference}%
            \subcaption{%
                Dashed lines are (red, \#3) 14.4 \AA\ and (teal, \#4) 28.8 \AA\ narrow flat pores, solid lines are 86 \AA\ wide flat pores.%
                \label{fig:density_reference_systems}%
            }%
        \end{minipage}%
        \hfill%
        \begin{minipage}[t]{\myfigwidth}%
            \captionsetup[subfigure]{width=0.9\textwidth}%
            \centering%
            \includesvg[width=\textwidth, svgpath=./images/density/]{density_water04_rough}%
            \subcaption{%
                Dashed lines are (red, \#3) 14.4 \AA\ and (teal, \#4) 28.8 \AA\ narrow fractures, solid lines are random fractures.%
                \label{fig:density_rough_systems}%
            }%
        \end{minipage}%
    }%
    \vspace{8pt}%
    \caption{%
        Water density ($\rho$) as function of distance to silica matrix ($r$) in \textbf{(a)} all four reference systems (flat pores) and \textbf{(b)} rough fracture systems.%
        %In \textbf{(a)} the dashed lines are 14.4 (red dashed line, \#3) and 28.8 (teal dashed line, \#4) \AA\ narrow flat pores \textbf{a)} flat pores and \textbf{b)} narrow, uniform width fractures. \hl{FINISH CAPTION}.%
        \label{fig:first_density_fig}%
    }%
\end{figure}%

The most significant trend we notice in \cref{fig:first_density_fig} is that the water density seems stable at 9-10 \AA\ from the silica matrix, but as we move closer than this we see a clear reduction in density. When we go from 10 \AA\ and move closer to the matrix we see in all systems that the density falls off at around 7-8 \AA\, and is reduced by almost 10\% at around 5 \AA. The density then increases to densities higher than the initial densities (the ones at 10 \AA) when we move towards 3 \AA. The relative reduction and then increase in density seems to be similar in all systems with similar overall densities, but in the systems with the highest overall density the relative change seems to be a bit smaller than in the other systems.

We now look at the density in the four reference systems, which is plotted in \cref{fig:density_reference_systems}. We see that the densities all have the same quantitative behaviour as we move further from the silica matrix, even though the net density is very different in the four systems, with the densities stabilizing at values ranging from 1050 to almost 1150 kg/m$^3$ at 8-10 \AA. One trend we notice is that the point where the density stabilizes, at around 7 \AA, seems to appear closer to the matrix when we increase the overall density. The minimum point seems to appear at approximately the same distance for the systems with a 86 \AA\ wide flat pore (system \#1 and \#2), but a bit closer to the matrix for the narrow flat pores (system \#4 and \#3).

In \cref{fig:density_rough_systems} we have plotted the density in the four random fracture systems. We see that the density in all four systems are very similar at all distances to the matrix, and that the density seems to stabilize between 1010 and 1025 kg/m$^3$ at 8-10 \AA\ from the silica matrix. We see that even though these four fractures have different geometries (especially system \#1 and \#2 are very different from system \#3 and \#4), the behaviour of the density as we go from 8 to 3 \AA\ from the silica matrix seems to be the same in all four systems. We also see that the density seems peak at around 8 \AA, and stabilize at a somewhat lower value as we go further from the silica matrix. This peak is similar to a peak observed around 5 \AA\ from the silica matrix by Bonnaud et al. in \cite{bonnaud2010molecular} (see Figure 6 in the article). In the article they use a different measure for the distance to the silica matrix, which might explain the difference in the distance at which the peak is observed.

We have estimated the bulk water density by the same technique we use for measuring water density as function of distance to the silica matrix, but now by averaging the density for all water-oxygen atoms further away from the silica matrix than a certain distance (typically 10 \AA\ or more), instead of using bins. This measurement requires that we have a fracture at least twice as wide as this distance to have any atoms we can measure the density of, so estimating the bulk density was only possible in reference system \#1 and \#2 with a flat, constant fracture of 86 \AA, and in the two random fractured systems \#1 and \#2, which have fractures with varying width. %We measured the bulk distance using a minimum distance to the matrix of 10 and 30 \AA\, to see if there was a difference between these two limits.%
%
\begin{table}[!htb]%
    \centering%
    \begin{tabular}{l|cc|cc}%
    ~               & \multicolumn{2}{c|}{$r>10$ \AA}& \multicolumn{2}{c}{$r>30$ \AA}    \\
    \textit{System} & Density [kg/m$^3$]    & N     & Density [kg/m$^3$]    & N   \\ \hline
    Reference \#1   & 1038.4                & 49k   & 1038.5                & 20k \\
    Reference \#2   & 1092.7                & 52k   & 1090.4                & 21k \\
    Rough \#1       & 1017.9                & 5.0k  & -                     & 0   \\
    Rough \#2       & 1002.8                & 4.2k  & -                     & 0   \\
    \end{tabular}%
    \vspace{8pt}%
    \caption{%
        Estimated bulk water densities, and the number of water molecules used in the calculations. Estimated using Voronoi tesselation, averaged over voronoi volumes for all water-oxygen atoms further away from silica matrix than 10 and 30 \AA\ (hydrogen atoms were removed before Voronoi tesselation). %
        \label{tab:bulk_water_density}%
    }%
\end{table}%
% bulk density (r>10.0) rough_fracture01_abel = 1018
% bulk density (r>10.0) rough_fracture03 = 1003
% bulk density (r>10.0) flat_square_02 = 1038
% bulk density (r>10.0) flat_square_03 = 1094

The estimated bulk densities are listed in \cref{tab:bulk_water_density}, where we have measured the bulk density for water atoms at least 10 \AA\ and at least 30 \AA\ from the silica matrix. We see that the estimated bulk density does not change if we use 10 or 30 \AA\ as the minimum distance from the matrix, indicating that a limit of 10 \AA\ is probably safe. If we compare the bulk densities to the plots of the density as function of distance from the matrix in \cref{fig:density_reference_systems,fig:density_rough_systems}, we see that the density seems to be stable close to the bulk density at around 10 \AA.

In \cref{fig:density_all} we have plotted the density in all eight systems we have simulated. For easier comparisons we have also tried normalizing the density against an estimated bulk density, plotted in \cref{fig:density_all_normalized}. In both figures we see the same trend as before, where the falloff for the density seems to move closer to the matrix as we increase the overall density. We see a hint of a similar trend for the minimum point of the density, but not nearly as clear as for the falloff point. We also notice that the falloff seems to be closer to the silica matrix for the rough fracture systems than for the flat pores, with the falloff being near 8 \AA\ in the rough fracture systems, but from 7 to 8 \AA\ for the flat pores, depending on the density.
\todoa{More about these two figures}
%
\begin{figure}[!p]%
    \centering%
    \setlength{\myfigwidth}{0.88\textwidth}%
    \setlength{\mycaptionwidth}{0.09999\textwidth}%
    %
    \begin{minipage}[c]{\myfigwidth}%
%         \includesvg[width=\textwidth, svgpath=./images/density/]{density_water04_all}%
        \includesvg[width=\textwidth, svgpath=./images/density/]{density_water04_all_trends}%
    \end{minipage}%
    \begin{minipage}[c]{\mycaptionwidth}%
        \subcaption{\label{fig:density_all}}%
    \end{minipage}%
    \\%
    \begin{minipage}[c]{\myfigwidth}%
        \includesvg[width=\textwidth, svgpath=./images/density/]{density_water04_all_normalized_divide}%
    \end{minipage}%
    \begin{minipage}[c]{\mycaptionwidth}%
        \subcaption{\label{fig:density_all_normalized}}%
    \end{minipage}%
    %
    \captionsetup{width=\textwidth}% since it's a floatpage figure
    \caption{%
        Density of water in all systems, as function of distance from the silica matrix. Dashed lines are reference systems, and solid lines rough fracture systems. In \textbf{(b)} the density has been normalized against the approximate bulk density, estimated from the density at 10 \AA\ from the silica matrix. %
        \label{fig:last_density_fig}%
    }%
\end{figure}%

\subsubsection*{Injection density vs. actual density}
To check the accuracy of the method we use for filling the fractures and pores with water, we want to compare the input densities used when filling the pores to the actual resulting density in the systems. We have already found the bulk density in four of the systems, but in the other systems we are not able to measure the density in the same way. What we do in these systems (``rough fracture'' \#3 and \#4, and reference system \#3 and \#4) is to approximate the bulk density by looking at the value of the density near 10 \AA\ from the silica matrix. This should give us a somewhat useable approximation to the bulk density in the system\footnote{In rough fracture system \#3 and reference system \#3 we do not have any measurements for water molecules around 10 \AA\ from the matrix, since these systems have pores that are narrower than 20 \AA. For these systems we extrapolate the the density at 10 \AA\ from the values we have.}.

The results can be seen in \cref{tab:estimate_bulk_density}. We see that using the value at 10 \AA\ gives a good approximation to the measured bulk density for the system where we are able to measure the bulk density, and that the method for filling the fractures and pores with water performs well when the pores and fracures are very large. But we also see that in the systems with very narrow pores, the method for filling the pore with water seems to struggle with acheving the wanted density. This is cause by the voxelation method we use when finding room for water molecules, as expected and noted previously.
%
\begin{table}[!htb]%
    \centering%
    \begin{tabular}{l|ccc}%
    \textit{System} & Input $D$ [kg/m$^3$]  & Measured $D$ [kg/m$^3$]   & Approx. $D$ at 10 \AA \\ \hline
    Reference \#1   & 1050                  & 1038                      & 1045 \\
    Reference \#2   & 1126                  & 1093                      & 1098 \\
    Reference \#3   & 1273                  & -                         & 1085 \\
    Reference \#4   & 1273                  & -                         & 1135 \\
    Rough \#1       & 1050                  & 1018                      & 1022 \\
    Rough \#2       & 1050                  & 1003                      & 1010 \\
    Rough \#3       & 1273                  & -                         & 1025 \\
    Rough \#4       & 1273                  & -                         & 1016 \\
    \end{tabular}%
    \vspace{8pt}%
    \caption{%
        A comparison of the estimated and measured bulk water density in the systems we have simulated, with the input density we used when filling the fractures with water.
        \label{tab:estimate_bulk_density}%
    }%
\end{table}%