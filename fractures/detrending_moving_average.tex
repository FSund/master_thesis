\subsection{Detrending moving average}
To estimate the Hurst exponent of a surface we use a method called detrending moving average (DMA) first described by E. Alessio, A. Carbone et al.\cite{alessio2002dma}, and later generalized to higher dimensions by A. Carbone \cite{carbone2007algorithm}. We use this method because ??? 

\todo[inline]{good results, easy to implement? }
\todo[inline]{everything below is for 2/3 D, not always obvious}

We define a \hl{(self-affine)} surface $f(i,j)$, where $f$ is the height in the point $(i,j)$, defined in a discrete 2-dimensional domain with size $N\times N$, for $i,j = 1,\dots,N$. We divide the surface into square \hl{OVERLAPPING!} \hl{subsurfaces} of size $n \times n$, and find the average $\tilde f_n$ of each subsurface by\footnote{\Cref{eq:carbone_average} is how the average $\tilde f_n$ is stated in the article\cite{carbone2007algorithm}, but we rewrite it to \cref{eq:carbone_average_rewritten} so it's easier to understand. The two forms are equivalent.} \todo{unclear what $i$ and $j$ is in $\tilde f_n(i,j)$?}
\begin{align}
    \tilde f_n(i,j) 
    &= \frac{1}{n^2}\sum_{k=-m}^{n-1-m} ~ \sum_{l=-m}^{n-1-m} f(i-k, j-l) \label{eq:carbone_average}\\
%     &= \frac{1}{n^2}\sum_{k=i-m}^{i-n+1+m}\sum_{l=j-m}^{j-n+1+m} f(i-k, j-l) \\
    &= \frac{1}{n^2} \sum_{k=(i+m)-n+1}^{i+m} ~ \sum_{l=(i+m)-n+1}^{i+m} f(k, l), \label{eq:carbone_average_rewritten}
\end{align}
where
\begin{align*}
    &m = \left \lfloor n\theta \right \rfloor &\text{for }0 \leq \theta < 1,
\end{align*}
which means that $0 \leq m \leq (n-1)$. If we study the equation above we see that $\theta$ is a parameter that controls the position of the \hl{subsurface/square} relative to the point $(i,j)$%
%in $\tilde f_n(i,j)$
. There are three \hl{extreme} cases for $\theta$, which are listed below and illustrated in \cref{fig:DMA_theta}.%
%
% \begin{itemize}
%     \item For $\theta = 0$ we have $m=0$ and we average over a square with the point $(i,j)$ in the square's upper right corner (($i-n+1) \leq k,l \leq i$). See \cref{fig:DMA_theta_a}.
%     \item For $\theta = 1/2$ we average over a square centered on the point $(i,j)$. See \cref{fig:DMA_theta_b}.
%     \item For $\theta = (n-1)/n$ we have the maximum value for $m$, $m=n-1$, and average over a square with the point $(i,j)$ in the square's lower left corner \todo{THESE LIMITS ARE WRONG??}($i \leq k,l \leq (i+n-1)$). See \cref{fig:DMA_theta_c}.
% \end{itemize}
%
\begin{description}
    \item[$\bm{\theta = 0}$:] \hfill\\ 
        We have $m=0$ and we average over a \hl{subsurface/square} with the point $(i,j)$ in the \hl{subsurface/square}'s upper right corner \hl{(($i-n+1) \leq k,l \leq i$)}. See \cref{fig:DMA_theta_a}.
    \item[$\bm{\theta = 1/2}$:] \hfill\\ 
        We average over a \hl{subsurface/square} centered on the point $(i,j)$. See \cref{fig:DMA_theta_b}.
    \item[$\bm{\theta = (n-1)/n}$:] \hfill\\ 
        {\sloppy 
        We have the maximum value for $m$, $m=n-1$, and average over a \hl{subsurface/square} with the point $(i,j)$ in the \hl{subsurface/square}'s lower left corner \todo{\st{THESE LIMITS ARE WRONG??} fixed} \hl{(${i \leq k,l \leq (i+n-1)}$)}. See \cref{fig:DMA_theta_c}.
        }
\end{description}%
%
\begin{figure}[htpb]%
    \centering%
    \begin{subfigure}[b]{0.25\textwidth}%
        \includesvg[width=\textwidth, svgpath=./images/Hurst/]{2DDMA_theta04_a}%
%         \caption{Illustration of how to divide a convex hexahedron into five tetraheda.}%
        \caption{$\theta = 0$}%
        \label{fig:DMA_theta_a}%
    \end{subfigure}%
    \hspace{0.1\textwidth}%
    \begin{subfigure}[b]{0.25\textwidth}%
        \includesvg[width=\textwidth, svgpath=./images/Hurst/]{2DDMA_theta04_b}%
%         \caption{A random fracture made from two periodic heightmaps.}%
        \caption{$\theta = 1/2$}%
        \label{fig:DMA_theta_b}%
    \end{subfigure}%
    \hspace{0.1\textwidth}%
    \begin{subfigure}[b]{0.25\textwidth}%
        \includesvg[width=\textwidth, svgpath=./images/Hurst/]{2DDMA_theta04_c}%
%         \caption{A random fracture made from two periodic heightmaps.}%
%         \caption{$\theta \rightarrow 1$ \\ ($\theta = (n-1)/n$)}%
%         \caption{$\theta \rightarrow 1$}%
        \caption{$\theta = (n-1)/n$}%
        \label{fig:DMA_theta_c}%
    \end{subfigure}%
        \caption{%
        Illustration of what the parameter $\theta$ controls in the detrending moving average method in \hl{2/3} dimensions. The circles are points where the surface is defined, the red star is the point $(i,j)$, and the black square encompasses the points averaged over to calculate $\tilde f_n(i,j)$ in \cref{eq:carbone_average_rewritten}. The illustration uses $n = 3$.%
        \label{fig:DMA_theta}%
    }%
\end{figure}%

We then define the generalized variance $\sigma_\text{DMA}^2$
\begin{align}
    \sigma_\text{DMA}^2 
    = \frac{1}{(N-n)^2}\sum_{i=n-m}^{N-m} ~ \sum_{j=n-m}^{N-m} 
    \big(
        f(i,j) - \tilde f_n(i,j)
    \big)^2, \label{eq:dma_variance}
\end{align}
where we see that $f(i,j) - f_n(i,j)$ is the difference between the point $(i,j)$ and the average of a \hl{subsurface/square} of points (including the point itself) of size $n \times n$, as explained above (see also \cref{fig:DMA_theta}).
% (the position of the \hl{subsurface/square} is controlled by the parameter $\theta$, as explained above, and illustrated \cref{fig:DMA_theta}). 
The summation limits $(n-m) \leq i,j \leq N-m$ are set so that the averages $f_n(i,j)$ don't exceed the domain with size $N \times N$.

The generalized variance goes as \hl{source?}
\begin{align*}
    \sigma_\text{DMA}^2 \sim \left(2n^2\right)^H,
\end{align*}
which we can use to find the Hurst exponent $H$, by calculating $\sigma_\text{DMA}^2$ for different sizes of the squares, $n$. We estimate $H$ by a linear fit of $\log \left(\sigma_\text{DMA}\right)^2$ against $\log \left(2n^2 \right)$, where $H$ is the slope of this fit.

In the paper that generalizes DMA to higher dimensions\cite{carbone2007algorithm} they use different parameters for each spatial dimension $d$, $\bvec\theta = \theta_1, \dots, \theta_d$ and $\bvec{n} = n_1, \dots, n_d$, but for simplicity and to avoid \hl{strange/spurious?} results, we use $\theta_1 = \theta_2 = \theta$ and $n_1 = n_2 = n$.

A modification mentioned mentioned in \cite{carbone2007algorithm} is replacing $\tilde f_n(i,j)$ in \cref{eq:dma_variance} with
\begin{align*}
    \tilde f_n^*(i,j) = (1-\alpha) f_n(i,j) + \alpha \tilde f_n(i-1,j-1),
\end{align*}
where $\alpha = n^2/(n+1)^2$, and $\tilde f_n(i,j)$ has the same form as in \cref{eq:carbone_average_rewritten}. This modification \hl{can give/gives} better results, so we have  implemented both the regular, and the modified method.

% With $\theta = 0$ we have $m = 0$ and average over all points $(\leq k \leq i,l \leq j)$. 

% With $\theta = n/(n-1)$ we have $m = n-1$, and average over all points $(k \geq i,l \geq j)$.

% We define the generalized variance $\sigma_\text{DMA}^2$ as the variance 
% \begin{align*}
%     \sigma_\text{DMA}^2 
%     = \frac{1}{(N-n)^2}\sum_{i=n-m}^{N-m}\sum_{j=n-m}^{N-m} 
%     \big(
%         f(i,j) - \tilde f_n(i,j)
%     \big)^2,
% \end{align*}
% where\footnote{We rewrite \cref{eq:carbone_average} for easier understanding}
% \begin{align}
%     \tilde f_n(i,j) 
%     &= \frac{1}{n^2}\sum_{k=-m}^{n-1-m}\sum_{l=-m}^{n-1-m} f(i-k, j-l) \label{eq:carbone_average}\\
% %     &= \frac{1}{n^2}\sum_{k=i-m}^{i-n+1+m}\sum_{l=j-m}^{j-n+1+m} f(i-k, j-l) \\
%     &= \frac{1}{n^2} \sum_{k=(i+m)-n+1}^{i+m} ~ \sum_{l=(i+m)-n+1}^{i+m} f(k, l).\nonumber %\label{eq:carbone_average_rewritten}
% \end{align}
% \begin{align*}
%     m = \left \lfloor n\theta \right \rfloor
% \end{align*}
% $\theta = 1$ or $\theta = 0$ (directed implementation, for fractals with preferential growth direction) \\
% $\theta = 0.5$ (isotropic implementation, ``uniformity in all orientations'') \\



% \begin{figure}[htpb]%
%     \centering%
%     \includesvg[width=0.99\textwidth, svgpath=./images/Hurst/]{2DDMA_theta04}%
%     \caption{%
%         Illustration of what the parameter $\theta$ controls in the detrending moving average method in 2 dimensions. The circles are points where a surface is defined, the red star marks the point $(i,j)$, and the black square marks the points averaged over to calculate $\tilde f_n(i,j)$ in \cref{eq:carbone_average_rewritten}.
%         \label{fig:DMA_theta}%
%     }%
% \end{figure}%



% \subsubsection{Detrended fluctuation analysis in higher dimensions}
% original DFA\cite{peng1994mosaic}
% HDDMA\cite{gu2006detrended}
% 
% To estimate the Hurst exponent of a fracture we use a generalized version of a method called \emph{detrended fluctuation analysis}\cite{peng1994mosaic}, which was generalized to higher dimensions by Gu and Zhou\cite{gu2006detrended}.
% 
% The two-dimensional DFA:
% 
% We have a self-affine surface stored in the matrix $f(i,j)$, where $f$ \hl{is} the height of the surface in the point $(i,j)$, with $i,j = 1,\dots,N$, and $N^2$ is the number of points where the surface is defined. The surface is divided into $n\times n$ disjoint (non-overlapping) \hl{and neighboring} surfaces \todo{sub-surface?} of the same size $s\times s$, where $n = N/s$. Each of these surfaces is denoted by $f_{u,v}$ such that $f_{u,v}(i,j) = f(i+k, j+l)$ for $i,j = 1, \dots, s$, where $k = (u-1)s$ and $l = (v-1)s$.
% 
% For a surface $f_{u,v}$ the cumulative sum $F_{u,v}(i,j)$ is calculated as follows
% \begin{align*}
%     F_{u,v}(i,j) = \sum_{m=1}^i\sum_{n=1}^j f_{u,v}(m,n),
% \end{align*}
% where $i,j = 1,\dots,s$. Note that F_{u,v}(i,j) itself is a surface.
% 
% The trend of the constructed surface $F_{u,v}$ can be determined by fitting it with a polynomial $\tilde F$



\todo[inline]{implemented in Matlab/Octave}

\section{How well does diamondSquare and DMA work?}
\orangebox{
    \begin{itemize}
        \item 1D: Plot of H from DMA vs. input H for wfbm (Matlab built in), and perhaps a FracLab variant?
        \item 1D: Plot of H from DMA vs. H from FracLab measure method?
        \item 2D: Plot of H from DMA vs. input DiamondSquare H and vs. synth2? (this could fit better in diamondsquare part).
    \end{itemize}
}
