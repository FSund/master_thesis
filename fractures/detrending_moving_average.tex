\section{Detrending moving average\label{sec:dma}}
\todoao{Plot of $\sigma_\text{DMA}^2$ as function of $n$ to determine $H$}%
% \todobo{Transition from rescaled range to surface?}%
Estimating the Hurst exponent of a surface is not trivial, so to measure this we use a method called detrending moving average (DMA), developed for 1-dimensional data by E. Alessio, A. Carbone et al.\cite{alessio2002dma}, and later generalized to higher dimensions by A. Carbone \cite{carbone2007algorithm}. 

After trying out some different methods for estimating the Hurst exponent, we ended up choosing this method both because it is easy to understand and implement, and because it has been shown to give good results, as we will also confirm later. A more detailed comparison of different methods for estimating the Hurst exponent can be seen in\cite{shao2012comparing}, where they find that DMA and DFA (detrended fluctuation analysis) overall perform better than FA (fluctuation analysis), also being less sensitive to the choice of scaling range.%
% \todo{good results, easy to implement? }%
% \todoao{everything below is for 2/3 D, not always obvious}%

\subsection{Detrending moving average in 2 dimensions}
% We define a \hl{(self-affine)} surface $f(i,j)$, where $f$ is the height in the point $(i,j)$, defined in a discrete 2-dimensional domain with size $N\times N$, for $i,j = 1,\dots,N$. We divide this surface into \emph{overlapping} subsurfaces of size $n \times n$, spanning the whole surface. The \hl{lower left} corners of the surfaces are located in the points $(i,j)$ for $i,j = 1,\dots, N-n+1$ (the limits are set so that the subsurfaces stay within the domain of the surface). In each surface we find a point $(k,l) = (n-m,n-m)$ (relative to lower bottom corner of subsurface)

We define a (self-affine) surface $f(i,j)$ of size $N,N$, with $i,j \in [1,N]$. For each point $i,j \in [1,N-n+1]$ in this surface we define a subsurface of size $n\times n$, where each subsurface consists if the points
\begin{align*}
%     (k,l) = \left([i, \dots, i+n], [j, \dots, j+n]\right)%
%     (k,l) = (i,j) + ([1\dots,n], [1,\dots,n])%
%     (k,l) = (i,j) + \left([1\dots,n], [1,\dots,n]\right) \\
    (k,l) \in \left\{ (i,j) + \left([1\dots,n], [1,\dots,n]\right)\right\}
%     (k,l) \in \left( i + [1\dots,n], j + [1,\dots,n]\right)%
\end{align*}
\begin{align*}
    \begin{pmatrix}
        k \\
        l
    \end{pmatrix}
    =
    \begin{pmatrix}
        i \\
        j
    \end{pmatrix}
    +
    \begin{pmatrix}
        1,\dots,n \\
        1,\dots,n
    \end{pmatrix}
\end{align*}%
\todoao{Decide on which vector/matrix thing}%
in the main surface. This means that the point $(i,j)$ is located in the \hl{``lower left''}\todoao{replace lower left with something smart} corner of the subsurface, that the subsurfaces overlap, and that they together span the whole main surface. The limits $i,j \in [1,N-n+1]$ are set so that all subsurfaces are \emph{inside} the main surface.

For each subsurface located at $(i,j)$ we find a point $(k_m, l_m)$ in the subsurface, which can be written as
\begin{align}
%     (k^*, l^*) = (i,j) + (n-m, n-m) %
%     (k_0, l_0) = (i,j) + (n-m, n-m) %
    (k_m, l_m) = (i,j) + (n-m, n-m), %
%     \\
%     \big(i+(n-m), j+(n-m)\big)
    \label{eq:dma_point_in_subsurface}
\end{align}
where $m$ is defined as
\begin{align*}
    &m = \lfloor n\theta \rfloor &\text{for } \theta \in [0,1).
\end{align*}
% With $\theta = 0$ the point is in the upper right corner of the subsurfaces, at $(i+n,i+n)$, for $\theta = 1/2$ the point is in the center, at $(i+n/2, i+n/2)$, and for $\theta \rightarrow 1 \rightarrow (n-1)/n$ the point is in the lower left corner, at $(i,j)$.
The parameter $\theta$ controls the position of this point inside the subsurface, and we have three extreme cases, listed below, and illustrated in \cref{fig:DMA_theta}.%
%
\begin{description}[%
    labelindent=\oldparindent,%
%     leftmargin=12em%
    leftmargin=2.0\oldparindent%
]%
    \item[$\bm{\theta = 0}$:] the point $(k_m, l_m)$ is in the upper right corner of the subsurface, \\at ${(k_m, l_m) = (i+n,i+n)}$
    \item[$\bm{\theta = 1/2}$:] the point $(k_m, l_m)$ is in the center of the subsurface, \\at ${(k_m, l_m) = (i+n/2, i+n/2)}$
    \item[$\bm{\theta \rightarrow 1 \rightarrow (n-1)/n}$:] the point $(k_m, l_m)$ is in the lower left corner of the subsurface, at ${(k_m, l_m) = (i,j)}$
\end{description}%
% \begin{itemize}[label={}]
%     \item $\bm{\theta = 0:}$ the point $(k_m, l_m)$ is in the upper right corner of the subsurface, at $(i+n,j+n)$
%     \item $\bm{\theta = 1/2:}$ the point $(k_m, l_m)$ is in the center, at $(i+n/2, j+n/2)$
%     \item $\bm{\theta \rightarrow 1 \rightarrow (n-1)/n:}$ the point $(k_m, l_m)$ is in the lower left corner, at $(i,j)$
% \end{itemize}
%
\begin{figure}[htpb]%
    \centering%
    \vspace{1em}% Add some space so the figure isn't as close to the content above
    \begin{subfigure}[b]{0.25\textwidth}%
        \includesvg[width=\textwidth, svgpath=./images/Hurst/]{2DDMA_theta04_a}%
%         \caption{Illustration of how to divide a convex hexahedron into five tetraheda.}%
        \caption{$\theta = 0$}%
        \label{fig:DMA_theta_a}%
    \end{subfigure}%
    \hspace{0.1\textwidth}%
    \begin{subfigure}[b]{0.25\textwidth}%
        \includesvg[width=\textwidth, svgpath=./images/Hurst/]{2DDMA_theta04_b}%
%         \caption{A random fracture made from two periodic heightmaps.}%
        \caption{$\theta = 1/2$}%
        \label{fig:DMA_theta_b}%
    \end{subfigure}%
    \hspace{0.1\textwidth}%
    \begin{subfigure}[b]{0.25\textwidth}%
        \includesvg[width=\textwidth, svgpath=./images/Hurst/]{2DDMA_theta04_c}%
%         \caption{A random fracture made from two periodic heightmaps.}%
%         \caption{$\theta \rightarrow 1$ \\ ($\theta = (n-1)/n$)}%
%         \caption{$\theta \rightarrow 1$}%
        \caption{$\theta = (n-1)/n$}%
        \label{fig:DMA_theta_c}%
    \end{subfigure}%
        \caption{%
            Illustration of three extreme cases for the parameter $\theta$ in DMA, on a surface. The dots are points where the main surface is defined, the red star is the point $(k_m,j_m)$, and the black square marks the subsurface, and the points averaged over to calculate $\bar {f_n}(i,j)$ in \cref{eq:carbone_average}. The illustrations use $n = 3$.%
        \label{fig:DMA_theta}%
    }%
\end{figure}%

We find the average $\bar {f_n}$ of each subsurface using
\begin{align}
    \bar {f_n}(i,j) = \frac{1}{n^2} \sum_{k = i}^{i+n} \sum_{l = j}^{j+n} f(k,l),
    \label{eq:carbone_average}
\end{align}
and we define the \emph{generalized variance}, $\sigma_\text{DMA}^2$, as the sum of the squared differences between the value in the point $f(k_m,l_m)$ minus the average $\bar {f_n}(i,j)$, for each subsurface. This can be written as
\begin{align}
    \sigma_\text{DMA}^2 
%     &= \frac{1}{(N-n)^2}\sum_{i=n-m}^{N-m} ~ \sum_{j=n-m}^{N-m} 
%     \big(
%         f(i,j) - \tilde f_n(i,j)
%     \big)^2,% \label{eq:dma_variance}
%     \\
    &= \frac{1}{(N-n)^2}\sum_{i=1}^{N-n+1} ~ \sum_{j=1}^{N-n+1} 
    \big(
        f(k_m,j_m) - \bar f_n(i,j)
    \big)^2%
%     \qquad\text{for } k_m, j_m \text{as in \cref{eq:dma_point_in_subsurface}}%
    \nonumber\\%
    &= \frac{1}{(N-n)^2}\sum_{i=1}^{N-n+1} ~ \sum_{j=1}^{N-n+1} 
    \big(
        f(i+n-m,j+n-m) - \bar f_n(i,j)
    \big)^2.
    \label{eq:dma_variance}
\end{align}

It can be shown that this generalized variance has a power-law dependence on $n$ \cite{alessio2002dma,carbone2007algorithm}, which goes as
\begin{align*}
    \sigma_\text{DMA}^2 \sim \left(2n^2\right)^H.
\end{align*}
We can use this dependence to estimate the Hurst exponent $H$, by calculating $\sigma_\text{DMA}^2$ for different sizes of the subsurfaces, $n$. We estimate $H$ by a linear fit of $\log \left(\sigma_\text{DMA}\right)^2$ against $\log \left(2n^2 \right)$, where $H$ is the slope of this fit.

In the paper by Anna Carbone that generalizes DMA to higher dimensions\cite{carbone2007algorithm} they use different parameters for each spatial dimension $d$, $\bvec\theta = \theta_1, \dots, \theta_d$ and $\bvec{n} = n_1, \dots, n_d$, but for simplicity and to avoid spurious results, we use $\theta_1 = \theta_2 = \theta$ and $n_1 = n_2 = n$.

A modification of the method mentioned is replacing $\bar f_n(i,j)$ in \cref{eq:dma_variance} with
\begin{align*}
    \bar {f_n}^*(i,j) = (1-\alpha) f_n(i,j) + \alpha \bar {f_n}(i-1,j-1),
\end{align*}
where
\[
    \alpha = n^2/(n+1)^2,
\]
and $\tilde f_n(i,j)$ has the same form as before (see \cref{eq:carbone_average}). To use this modification we also have to change the limits in \cref{eq:dma_variance} from ${i,j\in [1,N-n+1]}$ to $i,j\in [2,N-n+1]$. This modification has been shown to give better results for small systems\cite{carbone2007algorithm}, and since we are usually generating surfaces with a resolution of 100-200 points in each direction, we use this modified method in our implementation of the method.

% \todob{The method is implemented in Matlab/Octave}

% \subsection{The performance of detrending moving average}
\subsection{Validation}
\todoao{Read through DMA performance, make some conclusions from plot}
To verify that the method we used for estimating the Hurst exponent worked as intended, and gave good results, we ran a series of tests using synthetic 1-dimensional timeseries and 2-dimensional surfaces of fractional Brownian motion, with a known Hurst exponent. When doing these tests we soon realized that a big problem with synthesizing time series and surfaces with a given Hurst exponent is that it is both hard to accurately measure the exponent, and it is hard to synthesize data with a given exponent. %\hl{This means that when testing methods for analysing and synthesizing signals, you never know if it's you measuring method, or your synthesizing method that is causing problems if things aren't working as intended.}

To test the DMA method we synthesized data with a given Hurst exponent using 4 different programs, and measured the exponent using the detrending moving average method for each of these methods. 

For synthesizing 1D data we used the built-in Matlab-function \Verb!wfbm! which uses a wavelet-based synthesis method described by Abry and Sellan \cite{abry1996wavelet}, and two methods from the Matlab-toolbox FracLab\cite{fraclab_toolbox}, \Verb!fbmwoodchan! which uses a method proposed by Wood and Chan in \cite{wood1994simulation}, and \Verb!fbmlevinson! which uses Cholesky/Levinson factorization from \cite{levinson1947wiener}. 

There are many methods and algorithms for generating surfaces data with a given Hurst exponent, but we had problems finding working implementations of any of them. We will later implement a midpoint displacement method for this (see \cref{chap:generating_surfaces,subsec:SRA_implementation}), but having external reference is very useful, so we used a function from FracLab called \Verb!synth2!, which, according to the documentation of the function:\todoao{What to do about this? no sources...}
\begin{quote}
    \textit{Generates a 2D Fractional Brownian Motion (fBm) using an incremental Fourier Method for processes with stationary increments of order (0,1) and (1,0).}
\end{quote}
No further information is given in the FracLab documentation for how this method works.
% We will later also test our own implementation of a method for generating \hl{2d/3d} data against \hl{DMA}.
% ``Generates a 2D Fractional Brownian Motion (fBm) using an incremental Fourier Method for processes with stationary increments of order (0,1) and (1,0)''.

See \cref{fig:dma_performace} for a plot of Hurst exponent measured using DMA as a function of the exponent used as input for the four different methods for generating synthetic data above, for three different values of the parameter $\theta$.

From \cref{fig:dma_performace} we conclude that $\theta = 0.0$ seems to give the best and most consistent results, as also noted by Gao-Feng Gu and Wei-Xing Zhou in \cite{gu2010detrending}, where they further develop the DMA method to analyse multifractals. %
% \todobo{We would liked to have more 2D data references.}%
% \todob{Fix newcommand stuff in \cref{fig:dma_performace}, and maybe legend texts (can remove stuff now that we know have smaller text in figures)} %
%
\begin{figure}[htpb]%
    \centering%
    {
        \newcommand{\f}{\footnotesize}%
        \newcommand{\x}{\text}%
        \newcommand{\thislabelaaaaaa}{{\f $H_\x{in}=H_\x{out}$}}%
        \includesvg[width=1.0\textwidth, svgpath=./images/diamond_square_Hurst/test_HDDMA/]{fig03}%
    }
%     \caption[
%         Plot of the Hurst exponent as estimated by the detrending moving average method, used on data from four different synthetic signals, against the exponent used as input when generating the signals, and for three different values of the parameter $\theta$ used in \hl{DMA}. All methods except synth2 generate 1-dimensional signals, while synth2 generates a 2-dimensional signal. \hl{FINISH CAPTION}. %
%     ]{%
%         Plot of the Hurst exponent as estimated by the detrending moving average method, used on data from four different synthetic signals, against the exponent used as input when generating the signals, and for three different values of the parameter $\theta$ used in \hl{DMA}. All methods except \Verb!synth2! generate 1-dimensional signals, while \mono{synth2} generates a 2-dimensional signal. \hl{FINISH CAPTION}. %
%         \label{fig:dma_performace}%
%     }%
    \caption{%
        Plot of the Hurst exponent against the exponent used as input when generating the signals, as estimated by the detrending moving average method, used on data from four different synthetic signals, and for three different values of the parameter $\theta$ used in \hl{DMA}. For the 1d methods we have averaged over 1000 samples for each point, and for \mono{synth2} we have averaged over 100 samples, for input Hurst exponents between 0.05 and 0.95 in steps of 0.1. All methods except \mono{synth2} generate 1-dimensional signals, while \mono{synth2} generates a 2-dimensional signal. \hl{FINISH CAPTION}. %
        \label{fig:dma_performace}%
    }%
\end{figure}%
%
% \orangebox{
%     \begin{itemize}
%         \item 1D: Plot of H from DMA vs. input H for wfbm (Matlab built in), and perhaps a FracLab variant?
%         \item 1D: Plot of H from DMA vs. H from FracLab measure method?
%         \item 2D: Plot of H from DMA vs. input DiamondSquare H and vs. synth2? (this could fit better in diamondsquare part).
%     \end{itemize}
% }
