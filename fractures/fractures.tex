\chapter{Fractals and fractures}
\todoao{Finish fractals and fractures chapter thing}%
%
% \section{Surface area}
% \section{Distance to nearest atom}
% \begin{itemize}
%     \item Fractals
%     \item Fractional Brownian Motion
%     \item The Hurst Exponent
% \end{itemize}
To generate a fractal surface we use fractional Brownian motion (fBm), introduced by Mandelbrot and van Ness in 1968\cite{mandelbrot1968fractional}. Fractional Brownian motion is a generalization of Brownian motion, which is the random motion of particles suspended in a fluid, which comes from their collisions with the atoms and molecules in the fluid. 

Fractional Brownian motion is a process that generates data that is fractal, in the sense that it is self-similar. The data generated by this process can be characterized by a parameter denoted $H$, often called the Hurst exponent. $H$ is related to the autocorrelation (define?) of a data set, and is always a number between 0 and 1. It has been shown that fractures and other phenomena in nature have a Hurst exponent of around 0.75, for over eleven decades of length scales\cite{renard2013hurst}, so we will try to generate fractures with this Hurst exponent for our simulations.\todoco{No real fractals in nature, since limited resolution?}
%It is directly related to the fractal dimension $D$ via\cite{feder1988fractals}\todoao{The Hurst exponent relates the root mean square value of the change $dy$ to the distance $x$ over which it changes}
% \begin{align*}
%     D = d - H
% \end{align*}

\hl{The Hurst exponent quantifies the relative tendency of a data set to either regress to the mean, or to cluster in a direction.} $H\in[0,0.5]$ indicates a dataset with long-term switching between high and low values in adjacent pairs, meaning that a single high value will probably be followed by a low value, and vice versa. $H\in[0.5,1.0]$ indicates a \hl{time series} with long-term positive autocorrelation, meaning both that a high value will probably be followed by another high value, and that the values a long time into the future will also tend to be \hl{high (increasing?)}.\todoao{Remove this, or rewrite!}

Samples of fBm with different Hurst parameters will differ in what can quantitatively be called the ``roughness'' or the ``randomness'' of the data, as can be seen in \cref{fig:fBm_examples}, where we have plotted some samples of fBm with different Hurst exponents.

% \todo{in honor of both Harold Edwin Hurst and Ludwig Otto H\"older}

The Hurst exponent and the use of it as a means of characterizing a dataset was developed in the field of hydrology, as seen in \cite{hurst1951longterm,hurst1965longterm}, where it was used to determine the optimal dam sizing for the Nile river's, by studying the large fluctuations in the flow rate of the river, which there are extensive records of\todoco{rewrite this one-sentence paragraph}. The exponent is denoted $H$ in honor of both Harold Hurst, who was the lead researcher in these studies, and in honor of Otto H\"older.\todoco{Why holder?}
%
% \todobo{See \cite{mu1988steel} for ref. on fractal dimension of fractured steel surface}%
% \todob{No true fractals in nature, since we need infinite resolution. $H = $ Hausdorff dimension? index-$\alpha$?}%
% \todob{Decide on which Hurst figure, and maybe resize the label box in alt. 1? (just increase fig size in plot script)}%
%
\begin{figure}[htpb]%
    \centering%
    {%
%         \newcommand{\s}{\sim}%
        \newcommand{\sa}{H $\approx$ }%
        \includesvg[width=0.8\textwidth, svgpath = ./images/Hurst/]{fbm_1d_examples_grid01}%
    }%
    \caption{%
        Samples of fractional Brownian motion (fBm) with different Hurst exponents, generated using the built-in Matlab function \mono{wfbm}, which uses uses a wavelet-based synthesis method\cite{abry1996wavelet} for generating fBm.%
    }%
    \label{fig:fBm_examples}%
\end{figure}%
