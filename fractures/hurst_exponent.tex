\section{Hurst exponent}
\orangebox{
\begin{itemize}
    \item define fractal dimension
\end{itemize}
}

One common method to characterize fractals is by the Hurst exponent, usually called $H$\footnote{\todo[inline]{The name used by Hurst in his work where he first describes the exponent was actually $K$\cite{hurst1965longterm}\cite{hurst1951longterm}}}, which comes from a statistical method developed by Hurst\cite{hurst1965longterm}\cite{hurst1951longterm}. The Hurst exponent is related to the fractal dimension $D$ by
\begin{align*}
    D = d-H,
\end{align*}
where $d$ is the spatial dimension of the fractal's domain\cite{feder1988fractals}.

The statistical method developed by Hurst is called \emph{rescaled range analysis} and was designed for use on 1-dimensional time series $f(t)$. The method has been generalized to higher dimensions\cite{fan2013rescaled}, but the original 1D form is shown here.

\todo[inline]{Something about that it's hard to measure, and why.}

\subsection{Rescaled range analysis}
First the time series is divided into \todo{overlapping/non-overlapping?} intervals of length $\tau$. The average over each interval of length $\tau$ is
\begin{align*}
    \langle f \rangle_\tau = \frac{1}{\tau} \sum_{t=1}^\tau f(t).
\end{align*}
We let $F$ be the accumulated deviation from the mean
\begin{align*}
    F(t, \tau) = \sum_{t' = 1}^t \big( f(t') - \langle f \rangle_\tau \big).
\end{align*}
The difference between the maximum and minimum of the accumulated deviation from the mean is the \emph{range} R
\begin{align*}
    R(\tau) = \max_{1 \leq t \leq \tau} \big(F(t,\tau)\big) - \min_{1 \leq t \leq \tau} \big(F(t, \tau)\big).
\end{align*}
The standard deviation $S$ of the time series is estimated using
\begin{align*}
    S^2 = \frac{1}{\tau} \sum_{t=1}^\tau \big( f(t) - \langle f \rangle_\tau \big)^2.
\end{align*}
Hurst found that the observed \emph{rescaled range}, $R/S$, for many time series is described by the \hl{empirical} relation\cite{feder1988fractals}
\begin{align*}
    \frac{R}{S} = \left(\frac{\tau}{2}\right)^H \sim \tau^H.
\end{align*}
We now see that we can estimate the Hurst exponent by a linear fit of the form
\begin{align*}
    \log \left(\frac{R}{S}\right) \sim H\log\tau,
\end{align*}
where we find $H$ as the slope of the linear fit.


