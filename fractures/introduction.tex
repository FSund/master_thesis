% \chapter{Fractures}
% \chapter{Introduction}
\vspace*{\fill}
\begin{figure}[hp!]%
\thispagestyle{empty}
    \centering%
    \includegraphics[width=\textwidth]{images/fracture/large_fracture05.jpg}%
    \caption{%
        A randomly generated fracture.%
    }%
\end{figure}%
\vspace{\fill}

\chapter*{Introduction}
\addcontentsline{toc}{chapter}{Introduction}
We want to study the behaviour of water trapped in nanoscale pores and fractures in silica, so need a way to generate and characterize such structures. We choose to model a fracture as two surfaces, with the volume between the surfaces as the void fraction. This makes it easier to make fractures, since we only need to create two surfaces to get a fracture. 

To generate realistic surfaces we could have used scans of the structures we want to simulate, but the problem with this approach is that the resulting fracture will depend a lot on how we interpret the image, and that we can not easily generate a lot of samples of surfaces. To avoid this we use fractals to describe surfaces, and use this to randomly generate surfaces and fractures that are statistically similar to real fractures. Like a lot of phenomena in nature, fractures and surfaces can be very well described by the theory of fractals\cite{mandelbrot1983fractal}, so we think that this method should give good results.

What makes a \emph{fractal} fractal, or what characterizes a fractal, does not have a rigorous definition, but in general a fractal is something that looks similar to itself at different length scales. A fractal might be \emph{identical} to itself at different length-scales (self-similar), or be \emph{statistically similar} to itself (statistically self-similar). In one dimension this looks like \todoao{Examples of 1D, scale x and y equally for self-similar, different scaling for self-affine}

% Fractals and concepts similar to fractals have been discussed as early as the 17th century, but the use of fractals in natural sciences didn't really \hl{take off} before computers became \hl{readily} available and Benoit Mandelbrot gathered and developed a lot of theory on fractals in ``The Fractal Geometry of Nature''\cite{mandelbrot1983fractal}. 

% \todoa{Define fractal dimension}
\todoao{More intro about fractals, some history?}
% \tododone{Define fractal, define self-similarirt and similar terms}

% \todoao{change wording? copied from Fractals...}
% An \emph{affine transformation} transforms a point $\bvec x = (x_1, \dots, x_n)$ into new points $\bvec x' = (r_1x_1, \dots, r_n, x_n)$, where the scaling rations $r_1, \dots, r_n$ are \emph{not} all equal.
% 
% A bounded set $\mathcal{S}$ is \emph{self-affine} if $\mathcal{S}$ is the union of $N$ non-overlapping subsets $\mathcal{S}_1, \dots, \mathcal{S}_N$, each of which is congruent to the set $\bvec r(\mathcal{S})$ obtained from $\mathcal S$ by the affine transform defined by $\bvec r$. Here \emph{congruent} means that the set of points $\mathcal{S}$ is identical to the set of points $\bvec r(\mathcal{S})$ after possible translations and/or rotations of the set\cite{feder1988fractals}.
% 
% A set $\mathcal{S}$ is \emph{statistically self-affine} if $\mathcal{S}$ is the union of $N$ non-overlapping subsets each of which is scaled down by $\bvec r$ from the original, and is identical in all statistical respects to $\bvec r(\mathcal{S})$.