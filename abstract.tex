\begin{abstract}
% \tododone{Check distance before and after \AA and \cpp and similar commands - FIX: add slash \slash after command}
% \todoa{Write abstract}
% \todoa{Write acknowledgements}
% \todoa{Write introduction}
% \todoc{Refer to results sections in diffusion, TOP, density measuring parts?}
% \todob{Plot of temperature stability in simulations}
% \todob{Fix figure number crashing in lof}

In this thesis we have studied the structure and transport properties of water trapped in nanoporous fractures in silica. The studies are performed using a numerical method called molecular dynamics, using an advanced, accurate, and verified model of silica and water, based on quantum mechanical and experimental studies of these systems.

A method for generating random fractures which are statistically similar to naturally occurring fractures is developed, which is used to create nanoporous silica. To validate this method a method for characterizing random fractures and surfaces is implemented. We also develop a method for filling the fractures and pores with water, a method which we find to work well in large fractures, but the approach used breaks down in pores and fractures that are smaller than about 30 \Ang\ in one or more \hl{dimensions}.

We find that the structure and transport properties is of water is bulk-like down to around 8 \AA\ from a silica matrix, where these properties start to change. There is a region around 5 \AA\ from the silica matrix where the density is noticeably reduced, and the self-diffusion of water is also reduced. Upon studies of the intermolecular structure of water we conclude that the interactions between the hydrophilic silica structure and the water molecules causes changes in the water structure, which again leads to changes in the density and diffusive properties of water.

We also find that the actual geometry and structure of the pore or fracture in silica has little to no effect on the properties of water we studied, and the only real deviation we found from this was in a 14.4 \AA\ wide fracture, which had somewhat diffusion, and some differences in structure near the silica matrix.
\end{abstract}