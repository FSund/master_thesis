\begin{abstract}
% \tododone{Check distance before and after \AA and \cpp and similar commands - FIX: add slash \slash after command}
% \todoa{Write abstract}
% \todoa{Write acknowledgements}
% \todoa{Write introduction}
% \todoc{Refer to results sections in diffusion, TOP, density measuring parts?}
% \todob{Plot of temperature stability in simulations}
% \todob{Fix figure number crashing in lof}

In this thesis we have studied the structure and transport properties of water trapped in nanoporous fractures in silica. The studies are performed using a numerical method called molecular dynamics, using an advanced, accurate, and verified model of silica and water, based on quantum mechanical and experimental studies of these systems.

We have developed a method for generating random fractures which are statistically similar to naturally occurring fractures, which is used to create nanoporous silica. This method is validated using a fractal-based characterization method. We have also developed a method for filling the fractures and pores with water

We have studied systems of silica with nanoscale fractures of sizes ranging from 14 to 80 \Ang, with a Hurst (or roughness) exponent of approximately 0.75. We have found that approach we used for filling the fractures worked well in large fractures, but that the method breaks down in rough fractures smaller than around 20 \Ang in one or more dimensions.

We have also found that the structure and transport properties of water is bulk-like down to around 8 \AA\ from a the silica surface. We found that in a region around 7 \AA\ and closer to the silica surface, the water density was noticeably reduced, and the self-diffusion of water was also reduced. Upon studies of the intermolecular structure of water we conclude that the interactions between silica and water causes changes in the water structure, which again leads to changes in the density and diffusive properties of water near the silica surface.

We also found that the actual geometry and structure of the pore or fracture had little to no effect on the properties of water in the systems we studied, with the only real deviation from this was found in a 14.4 \AA\ wide fracture, which had some differences in diffusion and structure near the silica matrix compared to the other systems.
\end{abstract}